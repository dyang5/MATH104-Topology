\documentclass[11pt]{article}
\usepackage{../styletemplate}

\begin{document}

	\hrule
	\begin{center}
        \textbf{MATH104: Topology}\hfill \textbf{Spring 2024}\newline

		{\Large The Order Topology and the First Uncountable Ordinal}

		David Yang and Spencer Martin
	\end{center}

\hrule

\vspace{1em}

\section{Introduction}

\begin{definition*}[Order Topology]
Let $X$ be a set with a simple order relation and assume that $X$ has more than one element. Let $\mathscr{B}$ be the collection of all sets of the following types:
\begin{enumerate}
    \item All open intervals $(a, b)$ in $X$.
    \item All intervals of the form $[a_0, b)$ where $a_0$ is the smallest element (if any) of $X$.
    \item All intervals of the form $(a, b_0]$ where $b_0$ is the largest element (if any) of $X$.
\end{enumerate}

The collection $\mathscr{B}$ is a basis for a topology on $X$, known as the \textbf{order topology}. \\

\textit{If $X$ has no smallest element, there are no sets of type $2$, and if $X$ has no largest element, there are no sets of type $3$.}
\end{definition*}

\begin{lemma*}[13.1, page 80]
Let $X$ be a set; let $\mathscr{B}$ be a basis for a topology $\mathcal{T}$ on $X$. Then $\mathcal{T}$ equals the collection of all unions of elements of $\mathscr{B}$.
\end{lemma*}


\begin{lemma*}[10.2, page 66]
There exists a well-ordered set $A$ having a largest element $\Omega$, such that $S_\Omega$ of $A$ by $\Omega$
is uncountable but every other section of $A$ is uncountable.
\end{lemma*}

We can actually construct such a well-ordered set! Assuming the existence of an uncountable well-ordered set $B$ (a weaker result following from the axiom of choice), define $C$ to be the well-ordered set $\{1, 2\} \times B$ in the dictionary order; a section of $C$ is uncountable (for example, the section of $C$ by any element of the form $2 \times b$). Let $\Omega$ be the smallest element of $C$ for which the section of $C$ by $\Omega$ is uncountable, and let $A$ be this section along with $\Omega$.

\begin{definition*}[Minimal Uncountable Well-Ordered Set]
$S_\Omega$ is an uncountable well-ordered set, every section of which is countable; it is known as the \textbf{minimal uncountable well-ordered set}. \\

Alternatively, $S_\Omega$ is known as the \textbf{first uncountable ordinal}, and the closure of $\overline{S}_\Omega$ is $\overline{S}_\Omega = S_\Omega \cup \{ \Omega\}$.
\end{definition*}

\section{The Order Topology of the First Uncountable Ordinal}

\begin{itemize}
    \item Sets of type $2$ in the order topology can be thought of as sections of $S_{\Omega}$, and are countable.
    \item By problem 10.6(a) on page 67, $S_{\Omega}$ has no largest element, so no set of type $3$ exists in $S_{\Omega}$.
    \item Since $S_\Omega$ is well-ordered, any nonempty subset has to have a minimal element; consequently, all basis elements in the order topology can be written in the form $[a,b)$.  
    \item We can characterize all open sets of $S_\Omega$: they must be unions of our basis elements. These look like unions of disjoint intervals of open sets, or are of the form of our basis elements.
\end{itemize}



\end{document}
