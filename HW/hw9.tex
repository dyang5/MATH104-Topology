\documentclass[11pt]{article}
\usepackage{graphicx}
\usepackage{amsthm}
\usepackage{amsmath}
\usepackage{amssymb}
\usepackage[shortlabels]{enumitem}
\usepackage[margin=1in]{geometry}

\newcommand{\R}{\mathbb{R}}
\newcommand{\Z}{\mathbb{Z}}

\newenvironment{solution}
  {\renewcommand\qedsymbol{$\blacksquare$}\begin{proof}[Solution]}
  {\end{proof}}

\setlength\parindent{0pt}

\begin{document}

	\hrule
	\begin{center}
        \textbf{MATH104: Topology}\hfill \textbf{Fall 2023}\newline

		{\Large Homework 9}

		David Yang
	\end{center}

\hrule

\vspace{1em}

\textit{Chapter 11 (The Seifert-van Kampen Theorem) Problems.} \\

\underline{Section 70 (The Seifert-van Kampen Theorem), 70.2} \\

\textbf{Suppose that $i_2$ is surjective.}
\begin{enumerate}[a)]
    \item \textbf{Show that $j_1$ induces an epimorphism}
    \[
        h\colon \pi_1(U, x_0) \rightarrow \pi_1(X, x_0),
    \]
    \textbf{where $M$ is the least normal subgroup of $\pi_1(U, x_0)$ containing $i_1(\mathrm{ker} i_2)$. [\textit{Hint:} Show $j_1$ is surjective.]}

    \item \textbf{Show that $h$ is an isomorphism. [\textit{Hint:} Use Theorem 70.1 to define a left inverse for $h$.]}
\end{enumerate}

\newpage

\underline{Hatcher Problem} \\

\textbf{Let $X$ be the union of $n$ lines through the origin in $\R^3$. Compute the fundamental group of $\R^3 \text{ -- } X$.}

\begin{solution}
We claim that there is a deformation retract of $\R^3 \text{ -- } X$ to $S^2$ with $2n$ points removed. To conceptualize this,
we describe the deformation retract. For any point of $\R^3 \text{ -- } X$ that is on $S^2$, then stay as is. 
Otherwise, for any point $y$ of $R^3 \text{ -- } X$ that is not on $S^2$, let $\ell$ be the line passing through the origin and $y$.
Deformation retract $y$ to its closest intersection of $\ell$ with $S^2$. Each of the $n$ lines removed passes through $S^2$ twice,
so this description characterizes a deformation retract from $\R^3 \text{ -- } X$ to $S^2$ with $2n$ points removed. \\

Through stereographic projection, we know that there is a homeomorphism from $S^2 \text{ -- pt}$ to $\R^2$. 
By defining one of our $2n$ points removed from $S^2$ to be the ``north pole'' and applying the stereographic projection map,
we get a homeomorphism from $S^2$ with $2n$ points removed to $\R^2$ with $2n-1$ points removed. \\

The fundamental group of $\R^2$ with $2n-1$ points removed is simply the fundamental group of the wedge of $2n-1$ circles, which is the free group with $2n-1$ generators.
\end{solution}


\end{document}
