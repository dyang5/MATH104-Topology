\documentclass[11pt]{article}
\usepackage{graphicx}
\usepackage{amsthm}
\usepackage{amsmath}
\usepackage{amssymb}
\usepackage[shortlabels]{enumitem}
\usepackage[margin=1in]{geometry}

\newcommand{\R}{\mathbb{R}}
\newcommand{\Z}{\mathbb{Z}}

\newenvironment{solution}
  {\renewcommand\qedsymbol{$\blacksquare$}\begin{proof}[Solution]}
  {\end{proof}}

\setlength\parindent{0pt}

\begin{document}

	\hrule
	\begin{center}
        \textbf{MATH104: Topology}\hfill \textbf{Fall 2023}\newline

		{\Large Homework 13}

		David Yang
	\end{center}

\hrule

\vspace{1em}

\textit{Assorted Topological Groups Problems.} \\

\underline{Section 26 (Compact Spaces), 26.13(a)(b)} \\

\textbf{Let $G$ be a topological group.}
\begin{enumerate}[a)]
    \item \textbf{Let $A$ and $B$ be subspaces of $G$. If $A$ is closed and $B$ is compact, show $A \cdot B$ is closed. [\textit{Hint:} If $c$ is not in $A \cdot B$, find a neighborhood $W$ of $c$ such that
    $W \cdot B^{-1}$ is disjoint from $A$.]}

    \begin{solution}
    To show that $A \cdot B$ is closed, we will show that $(A \cdot B)^C$ is open. Let $c \in (A \cdot B)^C$. As the hint says, we will show that there is a neighborhood $W$ of $c$ such that
    $W \cdot B^{-1}$ is disjoint from $A$. \\
    
    Since $G$ is a topological group, it follows from Exercise 145.1 that the map $f \colon G \times G \rightarrow G$ sending $(x, y)$ to $xy^{-1}$ is continuous. $A$ is closed, so $A^C$ is open. Since $f$ is continuous, the preimage of an open set is open, and so $f^{-1}(A^C)$ is open. The one-point set $\{ c \}$ and $B$ are each compact. We claim that $\{c\} \times B \subseteq f^{-1}(A^C)$. 
    Equivalently, we will show that $f(\{ c \} \times B) \subseteq A^C$. Since $c \in (A \cdot B)^C$, or equivalently, $c \notin A \cdot B$, it follows that $c \neq ab$ for any $a \in A$, $b \in B$. Equivalently, $c b^{-1} \neq a$ for any $a \in A$ and $b \in B$. Since $f(\{ c \} \times B)$ is defined as $\{cb^{-1} \mid b \in B\}$, 
    it follows that $f(\{c \} \times B)$ and $A$ are disjoint, or equivalently, $f(\{ c \} \times B) \subseteq A^C$. \\

    To summarize, we have that $f^{-1}(A^C)$ is an open set in $G \times G$, and $\{ c\}$ and $B$ are each compact in $G$, with $\{c \} \times B \subseteq f^{-1}(A^C)$. It follows by Exercise 26.9 that there exist open sets $W$ and $V$ in $G$ such that
    \[
        \{ c \} \times B \subseteq W \times V \subseteq f^{-1}(A^C).
    \]
    Note that $c \in W$ by construction, and $W \times B \subseteq W \times V$. Consequently, since $W \times B \subseteq W \times V \subseteq f^{-1}(A^C)$, it follows that $f(W \times B) \subseteq A^C$, or equivalently
    \[
        W \cdot B^{-1} \subseteq A^C.
    \]
    It follows that $(W \cdot B^{-1}) \cap A = \varnothing$. Consequently, $wb^{-1} \neq a$, or $w \neq ab$ for any $w \in W, a \in A$ and $b \in B$. Thus, $W \cap (A \cdot B) = \varnothing$. \\

    For any arbitrary $c \in (A \cdot B)^C$, we have found a neighborhood $W$ of $c$ such that $W \, \cap \, (A \cdot B) = \varnothing$, or equivalently, $W \subseteq (A \cdot B)^C$. It follows by definition that $(A \cdot B)^C$ is open in $G$,
    so $A \cdot B$ is closed, as desired.
    \end{solution}

    \item \textbf{Let $H$ be a subgroup of $G$; let $p \colon G \rightarrow G/H$ be the quotient map. If $H$ is compact, show that $p$ is a closed map.}
    
    \begin{solution}
    Let $A$ be a closed set in $G$. Note that $p(A)$ consists of all the cosets of $G/H$ containing elements of $A$. Then $p^{-1}(p(A))$ is
    \[
        p^{-1}(p(A)) = \{ ah \mid a \in A, h \in H \} = A \cdot H.
    \]
    Since $H$ is compact and $A$ is closed, we know from part (a) that $A \cdot H$ is closed. Since $p$ is a quotient map, we know by definition that
    $p^{-1}(p(A))$ is closed if and only if $p(A)$ is closed. Consequently, since $p^{-1}(p(A))$ is closed, $p(A)$ must also be closed, so $p$ is a closed map, as desired.
    \end{solution}
\end{enumerate}

\newpage

\underline{Section 52 (The Fundamental Group), 52.7(c)(d)} \\

\textbf{Let $G$ be a topological group with operation $\cdot$ and identity element $x_0$. Let $\Omega(G, x_0)$ denote the set of all loops in $G$ based at $x_0$. If $f, g \in \Omega(G, x_0)$, 
let us define a loop $f \otimes g$ by the rule}
\[
    (f \otimes g)(s) = f(s) \cdot g(s).
\]
\begin{enumerate}[c)]
    \item \textbf{Show that the two group operations $*$ and $\otimes$ on $\pi_1(G, x_0)$ are the same. [\textit{Hint:} Compute $(f * e_{x_0}) \otimes (e_{x_0} * g)$.]}
    
    \begin{solution}
    Let $[f]$ and $[g]$ be any two arbitrary elements in $\pi_1(G, x_0)$. We will show that $[f] \otimes [g] = [f] * [g]$.
    Consider $[f] \otimes [g]$. Since $[f] = [f] * [e_{x_0}]$ and $[g] = [e_{x_0}] * [g]$, it follows that
    \begin{align*}
        [f] \otimes [g] &= ([f] * [e_{x_0}]) \otimes ([e_{x_0}] * [g]) \\
        &= ([f] * [e_{x_0}]) \cdot ([e_{x_0}] * [g]) \\
        &= ([f] \cdot [e_{x_0}]) * ([e_{x_0}] \cdot [g])
    \end{align*}
    where the final step follows by the construction of products of paths:  
    \begin{align*}
        ([f] * [e_{x_0}]) \cdot ([e_{x_0}] * [g]) 
        &= \begin{cases}
            f(2s) &\text{ for } s \in \left[ 0, \frac{1}{2} \right]  \\
            e_{x_0} &\text{ for } s \in \left[\frac{1}{2}, 1 \right]
        \end{cases}
        \cdot
        \begin{cases}
            e_{x_0} &\text{ for } s \in \left[ 0, \frac{1}{2} \right]  \\
            g(2s-1) &\text{ for } s \in \left[\frac{1}{2}, 1 \right]
        \end{cases} \\
        &= \begin{cases}
            f(2s) \cdot e_{x_0} &\text{ for } s \in \left[ 0, \frac{1}{2} \right]  \\
            e_{x_0} \cdot g(2s-1) &\text{ for } s \in \left[\frac{1}{2}, 1 \right]
        \end{cases} \\
        &= ([f] \cdot [e_{x_0}]) * ([e_{x_0}] \cdot [g]).
    \end{align*}
    Simplifying further by noting that $[f] \cdot [e_{x_0}] = [f]$ and $[e_{x_0}] \cdot [g] = [g]$, we get that
    \begin{align*}
        [f] \otimes [g] &= ([f] \cdot [e_{x_0}]) * ([e_{x_0}] \cdot [g]) \\
        &= [f] * [g]
    \end{align*}
    and so the two group operations $*$ and $\otimes$ on $\pi_1(G, x_0)$ are the same.
    \end{solution}

    \item \textbf{Show that $\pi_1(G, x_0)$ is abelian.}
    \begin{solution}
    Let $[f]$ and $[g]$ be two arbitary elements in $\pi_1(G, x_0)$. 
    so we will show that $\pi_1(G, x_0)$ is abelian by showing $[f] * [g] = [g] * [f]$. \\
    
    Note that $[f] = [e_{x_0}] * [f] = [e_{x_0} * f]$ and $[g] = [g] * [e_{x_0}] = [g * e_{x_0}]$, so
    \begin{align*}
        [f] \otimes [g] &= [e_{x_0} * f] \otimes [g * e_{x_0}] \\
        &= [e_{x_0} * f] \cdot [g * e_{x_0}] \\
        &= [e_{x_0} \cdot g] * [f \cdot e_{x_0}] \footnotemark \\
        &= [g] * [f].
    \end{align*}\footnotetext{this follows by the ``distributive property'' proved in the previous part.}
    
    From part (c), we know that the group operations $*$ and $\otimes$ on $\pi_1(G, x_0)$ are the same, so $[f] \otimes [g] = [f] * [g]$. It follows that
    \[
        [f] * [g] = [g] * [f],
    \]
    so $\pi_1(G, x_0)$ is abelian, as desired.
    \end{solution}
\end{enumerate}
\end{document}
