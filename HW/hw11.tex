\documentclass[11pt]{article}
\usepackage{graphicx}
\usepackage{amsthm}
\usepackage{amsmath}
\usepackage{amssymb}
\usepackage[shortlabels]{enumitem}
\usepackage[margin=1in]{geometry}

\newcommand{\R}{\mathbb{R}}
\newcommand{\Z}{\mathbb{Z}}

\newenvironment{solution}
  {\renewcommand\qedsymbol{$\blacksquare$}\begin{proof}[Solution]}
  {\end{proof}}

\setlength\parindent{0pt}

\begin{document}

	\hrule
	\begin{center}
        \textbf{MATH104: Topology}\hfill \textbf{Fall 2023}\newline

		{\Large Homework 10}

		David Yang
	\end{center}

\hrule

\vspace{1em}

\textit{Chapter 13 (Classification of Covering Spaces) Problems.} \\

\underline{Section 79 (Equivalence of Covering Spaces), 79.5(b)} \\

\textbf{Let $T = S^1 \times S^1$ be the torus; let $x_0 = b_0 \times b_0$. Prove the following:} \\

\textbf{\textit{Theorem. If $E$ is a covering space of $T$, then $E$ is homeomorphic either to $\R^2$, or to $S^1 \times \R$, or to $T$.}} \\

\textbf{(Hint: You may use the following result from algebra: if $F$ is a free abelian group of rank $2$ and $N$ is a nontrivial subgroup, then there is a basis $a_1, a_2$ for $F$ 
such that either (1) $ma_1$ is a basis for $N$, for some positive integer $m$,
or (2) $ma_1, na_2$ is a basis for $N$, where $m$ and $n$ are positive integers.)}

\begin{solution}

\end{solution}

\newpage

\underline{Section 80 (The Universal Covering Space), 80.1(a)} \\

\textbf{Let $q\colon X \rightarrow Y$ and $r\colon Y \rightarrow Z$ be maps; let $p = r \circ q$.}
\begin{enumerate}[a)]
    \item \textbf{Let $q$ and $r$ be covering maps. Show that if $Z$ has a universal covering space, then $p$ is a covering map. (Compare Exercise $4$ of Section $53$.)}
    \begin{solution}
    
    Let $E$ be the universal covering space of $Z$; $E$ is simply connected. By definition, there is a covering map $s \colon E \rightarrow Z$. 
    By Theorem 80.3, since $s \colon E \rightarrow Z$ and $r \colon Y \rightarrow Z$ are covering maps, there exists a covering map $t \colon E \rightarrow Y$ such that $s = r \circ t$. 
    Furthermore, by Theorem 80.3, since $t \colon E \rightarrow Y$ and $q \colon X \rightarrow Y$ are covering maps, there exists a covering map $u\colon E \rightarrow X$ such that $t = q \circ u$. \\

    Note that $u \colon E \rightarrow X$ and $s \colon E \rightarrow Z$ are covering maps. Furthermore, we have that
    \[
        p \circ u = (r \circ q) \circ u = r \circ (q \circ u) = r \circ t = s
    \]
    by construction. By Lemma 80.2(b), since $s = p \circ u$, and $u$ and $s$ are covering maps, then so is $p$. 
    Thus, if $q$ and $r$ are covering maps, and if $Z$ has a universal covering space, then $p$ is a covering map.
    \end{solution}
\end{enumerate}




\newpage

\underline{Section 81 (Covering Transformations), 81.4} \\

\textbf{Let $G$ be a group of homeomorphisms of $X$. The action of $G$ on $X$ is said to be fixed-point free if no element of $G$ other than the identity $e$ has a fixed point. 
Show that if $X$ is Haussdorf, and if $G$ is a finite group of homeomorphisms of $X$ whose action is fixed-point free, then the action of $G$ is properly discontinuous.}

\begin{solution}
Let $x \in X$. Let $g$ be a homeomorphism in $G$ not equal to the identity. Consider $g(x)$. 
Since the action of $G$ on $X$ is fixed-point free, it follows that $g(x) \neq x$. Since $x$ and $g(x)$ are distinct points in $X$ and $X$ is Hausdorff,
it follows that there are disjoint open sets $V$ and $W$ about $x$ and $g(x)$ that are disjoint in $X$. \\

Since $g$ is a homeomorphism, it is continuous, and so $g^{-1}(W)$ is also open, and contains $x$. Consider $U = g^{-1}(W) \cap V$. 
Note that by construction, $U \subseteq V$ and $g(U) \subseteq W$. Since $V \cap W$ is empty, it follows that $g(U)$ and $U$ are disjoint. \\

Thus, since for every $x \in X$, there is a neighborhood $U$ of $x$ such that $g(U)$ is disjoint from $U$ whenever $g \neq e$, the action of $G$ is properly discontinuous, as desired.
\end{solution}


\end{document}
