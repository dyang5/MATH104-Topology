\documentclass[11pt]{article}
\usepackage{graphicx}
\usepackage{amsthm}
\usepackage{amsmath}
\usepackage{amssymb}
\usepackage[shortlabels]{enumitem}
\usepackage[margin=1in]{geometry}

\newcommand{\R}{\mathbb{R}}
\newcommand{\Z}{\mathbb{Z}}

\newenvironment{solution}
  {\renewcommand\qedsymbol{$\blacksquare$}\begin{proof}[Solution]}
  {\end{proof}}

\setlength\parindent{0pt}

\begin{document}

	\hrule
	\begin{center}
        \textbf{MATH104: Topology}\hfill \textbf{Fall 2023}\newline

		{\Large Homework 10}

		David Yang
	\end{center}

\hrule

\vspace{1em}

\textit{Chapter 13 (Classification of Covering Spaces) Problems.} \\

\underline{Section 79 (Equivalence of Covering Spaces), 79.5(b)} \\

\textbf{Let $T = S^1 \times S^1$ be the torus; let $x_0 = b_0 \times b_0$. Prove the following:} \\

\textbf{\textit{Theorem. If $E$ is a covering space of $T$, then $E$ is homeomorphic either to $\R^2$, or to $S^1 \times \R$, or to $T$.}} \\

\textbf{(Hint: You may use the following result from algebra: if $F$ is a free abelian group of rank $2$ and $N$ is a nontrivial subgroup, then there is a basis $a_1, a_2$ for $F$ 
such that either (1) $ma_1$ is a basis for $N$, for some positive integer $m$,
or (2) $ma_1, na_2$ is a basis for $N$, where $m$ and $n$ are positive integers.)}

\begin{solution}

\end{solution}

\newpage

\underline{Section 81 (Covering Transformations), 81.4} \\

\textbf{Let $G$ be a group of homeomorphisms of $X$. The action of $G$ on $X$ is said to be fixed-point free if no element of $G$ other than the identity $e$ has a fixed point. 
Show that if $X$ is Haussdorf, and if $G$ is a finite group of homeomorphisms of $X$ whose action is fixed-point free, then the action of $G$ is properly discontinuous.}

\begin{solution}

\end{solution}


\end{document}
