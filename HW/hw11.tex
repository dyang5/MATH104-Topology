\documentclass[11pt]{article}
\usepackage{graphicx}
\usepackage{amsthm}
\usepackage{amsmath}
\usepackage{amssymb}
\usepackage[shortlabels]{enumitem}
\usepackage[margin=1in]{geometry}

\newcommand{\R}{\mathbb{R}}
\newcommand{\Z}{\mathbb{Z}}

\newenvironment{solution}
  {\renewcommand\qedsymbol{$\blacksquare$}\begin{proof}[Solution]}
  {\end{proof}}

\setlength\parindent{0pt}

\begin{document}

	\hrule
	\begin{center}
        \textbf{MATH104: Topology}\hfill \textbf{Fall 2023}\newline

		{\Large Homework 11}

		David Yang
	\end{center}

\hrule

\vspace{1em}

\textit{Chapter 13 (Classification of Covering Spaces) Problems.} \\

\underline{Section 79 (Equivalence of Covering Spaces), 79.5(b)} \\

\textbf{Let $T = S^1 \times S^1$ be the torus; let $x_0 = b_0 \times b_0$. Prove the following:} \\

\textbf{\textit{Theorem. If $E$ is a covering space of $T$, then $E$ is homeomorphic either to $\R^2$, or to $S^1 \times \R$, or to $T$.}} \\

\textbf{(Hint: You may use the following result from algebra: if $F$ is a free abelian group of rank $2$ and $N$ is a nontrivial subgroup, then there is a basis $a_1, a_2$ for $F$ 
such that either (1) $ma_1$ is a basis for $N$, for some positive integer $m$,
or (2) $ma_1, na_2$ is a basis for $N$, where $m$ and $n$ are positive integers.)}

\begin{solution}

Let $E$ be a covering space of $T$, with $p$ representing the covering map from $E$ to $T$. Pick some $e_0 \in p^{-1}(x_0)$. 
Since $\pi_1(T, x_0) \cong \Z \times \Z$, and $p_{\ast} (\pi_1(E, e_0))$ must be a subgroup of $\Z \times \Z$, a free abelian group of rank $2$, we know that $p_{\ast} (\pi_1(E, e_0))$ is either the trivial subgroup $\{ (0, 0) \}$,
the subgroup generated by $ma_1$, or the subgroup generated by $ma_1$ and $na_2$, where $m$ and $n$ are positive integers and $a_1$ and $a_2$ represent some basis for $\Z \times \Z$.
\footnote{the form for the latter two nontrivial subgroups follows directly from the hint.} \\

Since there is an isomorphism of $\pi_1(T, x_0) \cong \Z \times \Z$ mapping the respective basis elements $a_1$ and $a_2$ to the respective canonical basis elements $(1, 0)$ and $(0, 1)$, we know by Exercise 79.5(a),
that this isomorphism is induced by a homeomorphism $f \colon T \rightarrow T$ mapping $x_0$ to $x_0$. Consider now the map $q \colon E \rightarrow T$ defined by $q = f \circ p$. Since $f$ is the composition of a covering map and 
a homeomorphism, it is itself a covering map.\footnote{alternatively, $f$ is a $1$-fold covering map, and by Exercise 54.4, since both $f$ and $p$ are covering maps and $f^{-1}(x)$ is finite for each $x \in T$, $q$ is also a covering map.} \\

Consider now $q_{\ast}(\pi_1(E, e_0)) = (f_{\ast} \circ p_{\ast}) (\pi_1(E, e_0))$. Note that $p_{\ast}$ is a map from $\pi_1(E, e_0)$ onto one of the three classified subgroups 
$\{ 0 , 0 \}$, $\langle ma_1 \rangle$, or $\langle ma_1, na_2 \rangle$ of $\pi_1(T, x_0)$. Then $f_{\ast}$ maps these subgroups to the respective subgroups in the canonical basis of $\pi_1(T, x_0)$, represented as the trivial subgroup $\{ 0, 0 \}$, $\langle (m, 0) \rangle$, or $\langle (m, 0), (0, n) \rangle$ of $\pi_1(T, x_0)$. 
These represent our three possible images of $q_{\ast}(\pi_1(E, e_0))$. \\

Recall from Example 1 that the covering spaces of $S^1$ are $\R$ and $S^1$; the respective covering maps include $p_m \colon S^1 \rightarrow  S^1$ with $p_m(z) = z^m$ for any positive integer $m$, 
and $q \colon \R \rightarrow S^1$ with $q(t) = (\cos (2\pi t), \sin (2\pi t))$. By Theorem 53.3, since the product of covering maps is a covering map, we can form three distinct covering maps $p^{\prime}_1$, $p^{\prime}_2$, and $p^{\prime}_3$ 
each of which map from some space $E^{\prime}$ to $T$. Let $e^{\prime}_0 = {p^{\prime}}^{-1}(x_0)$. \\

We see that 
\begin{enumerate}
    \item $p^{\prime}_1 \colon S^1 \times S^1 \rightarrow S^1 \times S^1$ defined by $p^{\prime}_1 = p_m \times p_n$, with $p^{\prime}_1(E^{\prime}, e^{\prime}_0) = \langle (m, 0), (0, n) \rangle$
    \item $p^{\prime}_2 \colon S^1 \times \R \rightarrow S^1 \times S^1$ defined by $p^{\prime}_2 = p_m \times q$, with $p^{\prime}_2(E^{\prime}, e^{\prime}_0) = \langle (m, 0) \rangle$
    \item $p^{\prime}_3 \colon \R \times \R \rightarrow S^1 \times S^1$ defined by $p^{\prime}_3 = q \times q$, with $p^{\prime}_3(E^{\prime}, e^{\prime}_0) = \{ (0, 0) \}$
\end{enumerate}
are three distinct covering maps of $T = S^1 \times S^1$. Note that these covering maps $p^{\prime}_1, p^{\prime}_2, p^{\prime}_3 \colon E^{\prime} \rightarrow T$ induce the same images in $\pi_1(T, x_0)$ 
as the three distinct possibilities for the image of the induced homomorphism $q_{\ast}(\pi_1(E, e_0))$ in $\pi_1(T, x_0)$. \\

Thus, by Theorem 79.2, there must be an equivalence between $E$ and the spaces of $E^{\prime}$, which were homeomorphic to $S^1 \times S^1$, $S^1 \times \R$, and $\R \times \R$. 
It follows that if $E$ is a covering space of $T$, then $E$ is homeomorphic either to $\R^2$, or to $S^1 \times \R$, or to $T$, as desired.
\end{solution}

\newpage

\underline{Section 80 (The Universal Covering Space), 80.1(a)} \\

\textbf{Let $q\colon X \rightarrow Y$ and $r\colon Y \rightarrow Z$ be maps; let $p = r \circ q$.}
\begin{enumerate}[a)]
    \item \textbf{Let $q$ and $r$ be covering maps. Show that if $Z$ has a universal covering space, then $p$ is a covering map. (Compare Exercise $4$ of Section $53$.)\footnote{
    We replace the requirement that $r^{-1}(z)$ is finite for each $z \in Z$ with the condition that $Z$ has a universal covering space to get that $p$ is a covering map.}}
    \begin{solution}
    
    Let $E$ be the universal covering space of $Z$; $E$ is simply connected. By definition, there is a covering map $s \colon E \rightarrow Z$. 
    By Theorem 80.3, since $s \colon E \rightarrow Z$ and $r \colon Y \rightarrow Z$ are covering maps, there exists a covering map $t \colon E \rightarrow Y$ such that $s = r \circ t$. 
    Furthermore, by Theorem 80.3, since $t \colon E \rightarrow Y$ and $q \colon X \rightarrow Y$ are covering maps, there exists a covering map $u\colon E \rightarrow X$ such that $t = q \circ u$. \\

    Note that $u \colon E \rightarrow X$ and $s \colon E \rightarrow Z$ are covering maps. Furthermore, we have that
    \[
        p \circ u = (r \circ q) \circ u = r \circ (q \circ u) = r \circ t = s
    \]
    by construction. By Lemma 80.2(b), since $s = p \circ u$, and $u$ and $s$ are covering maps, then so is $p$. 
    Thus, if $q$ and $r$ are covering maps, and if $Z$ has a universal covering space, then $p$ is a covering map.
    \end{solution}
\end{enumerate}

\newpage

\underline{Section 81 (Covering Transformations), 81.3} \\

\textbf{Let $p\colon X \rightarrow B$ be a covering map (not necessarily regular); let $G$ be its group of covering transformations.}
\begin{enumerate}[a)]
    \item \textbf{Show that the action of $G$ on $X$ is properly discontinuous.}
    
    \begin{solution}
    Let $x \in X$, so that $p(x) \in B$. Since $p$ is a covering map, there exists a neighborhood $V$ of $p(x)$ in $B$ that is evenly covered by $p$. 
    Equivalently, the inverse image $p^{-1}(V)$ is a disjoint union of neighborhoods in $U_\alpha$ in $X$ such that the restriction of $p$ to $U_\alpha$ is a homeomorphism of $U_\alpha$ onto $V$, for each $\alpha$. 
    Let $U$ be the neighborhood of $p^{-1}(V)$ in $X$ containing $x$. We will show that $g(U)$ and $U$ are disjoint, for any $g \neq e$. \\

    Suppose for the sake of contradiction that there exists some element $y \in g(U) \cap U$. 
    Then $y \in U$ and $y = g(z)$ for some $z \neq y$ in $U$ (as a non-identity covering transformation has no fixed points). 
    Since $g$ is a covering transformation, it follows that $p \circ g = p$, so $p(g(z)) = p(z)$. Simplifying, we get that
    $p(y) = p(z)$, where both $y$ and $z$ are in $U$ and $y \neq z$. The restriction of $p$ to $U$ cannot be a homeomorphism of $U$ onto $V$, as it is not injective; thus, we arrive at a contradiction,
    and conclude that $g(U)$ and $U$ are disjoint for every $g \neq e$ in $G$. \\

    By definition, it follows that the action of $G$ on $X$ is properly discontinuous.
    \end{solution}
%    \item \textbf{Let $\pi \colon X \rightarrow X / G$ be the projection map. Show that there is a covering map $k \colon X / G \rightarrow B$ such that $k \circ \pi = p$.}
    
\end{enumerate}

\newpage

\underline{Section 81 (Covering Transformations), 81.4} \\

\textbf{Let $G$ be a group of homeomorphisms of $X$. The action of $G$ on $X$ is said to be fixed-point free if no element of $G$ other than the identity $e$ has a fixed point. 
Show that if $X$ is Haussdorf, and if $G$ is a finite group of homeomorphisms of $X$ whose action is fixed-point free, then the action of $G$ is properly discontinuous.}

\begin{solution}
Let $x \in X$. Let $\{ g_1, \dots, g_n \} $ be the finite group of homeomorphisms in $G$ of $X$ that are not equal to the identity. 
Since the action of $G$ on $X$ is fixed-point free, it follows that $g_i(x) \neq x$ for all $i \in \{ 1, \dots, n \}$. 
Since $x$ and $g_i(x)$ are distinct points in $X$ and $X$ is Hausdorff, it follows that there are disjoint open sets $V_i$ and $W_i$ about $x$ and $g(x)$ that are disjoint in $X$ for each $i \in \{ 1, \dots, n \}$. \\

Let $\tilde{V} = \bigcap\limits_{i \in \{ 1, \dots, n \} } V_i$; by construction, $\tilde{V}$ is the intersection of finitely many open sets, so it is itself open. 
Consider
\[
    U = \bigcap_{i \in \{ 1, \dots, n \} } g_i^{-1}(W_i) \cap \tilde{V}
\]
Since each $g_i$ is a homeomorphism, they are each continuous, and so each $g_i^{-1}(W_i)$ is open, and by construction contains $x$. Consequently, each $g_i^{-1}(W_i) \cap \tilde{V}$ is open, and so $U$, the intersection of
finitely many such open sets, is also open. Furthermore, $U$ is nonempty, as it contains $x$ by construction. Consider $U = g^{-1}(W) \cap V$.\\

Note that for each $g_i$, $U \subseteq V_i$, and $U \subseteq g_i^{-1}(W_i)$, so $g_i(U) \subseteq W_i$. Since each $V_i$ and $W_i$ are disjoint open sets by construction,
$U$ and $g_i(U)$ must be disjoint open sets, for all $i \in \{ 1, \dots, n \}$. \\

Since for every $x \in X$, there is a neighborhood $U$ of $x$ such that for all $g_i$ in the finite group of homeomorphisms of $X$ that are not the identity, $g_i(U)$ is disjoint from $U$,
the action of $G$ is properly discontinuous, as desired.
\end{solution}


\end{document}
