\documentclass[11pt]{article}
\usepackage{graphicx}
\usepackage{amsthm}
\usepackage{amsmath}
\usepackage{amssymb}
\usepackage[shortlabels]{enumitem}
\usepackage[margin=1in]{geometry}

\newcommand{\R}{\mathbb{R}}

\newenvironment{solution}
  {\renewcommand\qedsymbol{$\blacksquare$}\begin{proof}[Solution]}
  {\end{proof}}

\setlength\parindent{0pt}

\begin{document}

	\hrule
	\begin{center}
        \textbf{MATH104: Topology}\hfill \textbf{Fall 2023}\newline

		{\Large Homework 3}

		David Yang
	\end{center}

\hrule

\vspace{1em}

\textit{Chapter 3 (Connectedness and Compactness) Problems.} \\

\underline{Section 22 (Connected Spaces), 22.4(a)} \\

\textbf{Define an equivalence relation on the plane $X = \R^2$ as follows:}
\[
	x_0 \times y_0 \sim x_0 \times y_1 \text{ if } x_0 + y_0^2 = x_1 + y_1^2.
\]
\textbf{Let $X^*$ be the corresponding quotient space. It is homeomorphic to a familiar space; what is it?
[\textit{Hint:} Set $g(x \times y) = x + y^2$.]} \\

\begin{solution}
We will show that $X^*$ is homeomomorphic to $\R$. We define, as per the hint, the continuous map $g\colon \R^2 \rightarrow \R$ defined by
$g(x \times y) = x + y^2$. Similarly, let us define the continuous map $f \colon \R \rightarrow \R^2$ by $f(x) = x \times 0$. Note that
\[
	(g \circ f)(x) = g(f(x)) = g(x \times 0) = x,
\]
for any $x \in \R$, so $g \circ f$ is the identity map of $\R$. By Exercise 22.2(a), it follows that $g$ is a quotient map from $\R^2$ to $\R$,
or equivalently, from $X$ to $R$. \\

Define $p \colon X \rightarrow X^*$ to be the surjective map mapping each point in the plane $X$ to its equivalence class. Equivalently,
since $g$ is a surjective continuous map from $X$ to $\R$, we can write the quotient space as
\[
	X^* = \{ g^{-1}(\{ r \}) \mid r \in \R\},
\]
matching the definition in Corollary 22.3. By Corollary 22.3(a), since $g$ is a quotient map, 
it induces a homeomorphism from the quotient space $X^*$ to $\R$. \\

Thus, we conclude that $X^*$ is homeomorphic to $\R$.
\end{solution}

\newpage

\underline{Section 26 (Compact Spaces), 26.5} \\

\textbf{Let $A$ and $B$ be disjoint compact subspaces of the Hausdorff space $X$. Show that there exist disjoint
open sets $U$ and $V$ containing $A$ and $B$, respectively.}

\begin{solution}
Let $a$ be an element of $A$. Since $A$ and $B$ are disjoint, it follows that $a$ is not in $B$. By Lemma 26.4, since 
$B$ is a compact subspace of the Hausdorff space $X$ and $a \notin B$, we know that there exist disjoint open sets $U_a$ and $V_a$ containing $a$ and $B$, respectively. 
\\

Consider the union of each of these respective open sets $U_a$ across all elements of $a$ in $A$:
\[
	\tilde{U} = \bigcup_{a \in A} U_a.
\]
By construction, $\tilde{U}$ is a union of open sets that contains every element in $A$, so it is an open covering of $A$. Since $A$ is compact, $\tilde{U}$ must have a finite
subcover consisting of open sets $U_{a_1}, U_{a_2}, \dots, U_{a_n}$ where each $a_1, \dots, a_n$ is a distinct element of $A$. Consider the 
corresponding finite collection of open sets $V_{a_1}, V_{a_2}, \dots, V_{a_n}$ in $B$ defined for those same elements $a_1, \dots, a_n$. \\


We define 
\[
	U = \bigcup_{i \in \{ 1, \dots, n \}} U_{a_i}
\]
and 
\[
	V = \bigcap_{i \in \{ 1, \dots, n \}} V_{a_i}.
\]

Note that $U$ is by construction a finite subcover of $A$, so it is open and contains $A$. 
Similarly, $V$ is an intersection of finitely many open sets in $B$, so it is also open. Furthermore, since each open set $V_{a_i}$ contains $B$ by construction, the intersection $V$ also contains $B$. \\

It remains to show that $U$ and $V$ are disjoint. Suppose for the sake of contradiction that there is some element $x \in U \cap V$. It follows that $x \in U$ so $x \in U_{a_j}$ for some $j$. Since $x \in V$, $x$ is in every $V_{a_i}$ for $i$ from $1$ to $n$. Consequently,
$x \in V_{a_j}$ for that same value of $j$. It follows that $x \in U_{a_j} \cap V_{a_j}$, contradicting the disjoint property for each pair of corresponding open sets given by Lemma 26.4. \\

Thus, $U = \bigcup\limits_{i \in \{ 1, \dots, n \}} U_{a_i}$ and $V = \bigcap\limits_{i \in \{ 1, \dots, n \}} V_{a_i}$ are disjoint open sets of $X$ containing $A$ and $B$, as desired.
\end{solution}

\newpage

\underline{Section 24 (Connected Subspaces of the Real Line), 24.2} \\

\textbf{Let $f \colon S^1 \rightarrow \R$ be a continuous map. Show there exists a point $x$ of $S^1$ such that $f(x) = f(-x)$.\footnote{In an effort to immortalize the legend of this problem, I will note that an equivalent problem was given to me in a graduate school interview,
after I had solved this problem for homework at 1 AM the previous day. The program will be named if I am accepted. Fingers crossed! (2/9/24)}}

\begin{solution}
Let us define $g \colon S^1 \rightarrow \R$ by $g(x) = f(x) - f(-x)$. Since $f(x)$ and $f(-x)$ are continuous, so is $g(x)$, as it is a difference of two continuous functions. 
Furthermore, note that
\[
	g(x) = f(x) - f(-x) \text{ and } g(-x) = f(-x) - f(x) = -g(x)
\]
for any $x \in S^1$. Consider some $y$ in $S^1$. We can assume that $f(y) \neq f(-y)$, otherwise we would be done. 
Equivalently, $g(y) \neq 0$. Suppose without loss of generality that $g(y) > 0$. It follows that $g(-y) = -g(y) < 0$.\\

Since $g$ is a continuous map from the connected space $S^1$ to the ordered set $\R$ in the order topology, and $0$ is a point of $\R$ between $g(-y)$ and $g(y)$, we know by the Intermediate Value Theorem that there must be some point
$x$ of $S^1$ such that $g(x) = 0$. It follows that for that point $x$, $f(x) = f(-x)$, as desired.
\end{solution}

\underline{Section 24 (Connected Subspaces of the Real Line), 24.3} \\

\textbf{Let $f \colon X \rightarrow X$ be continuous. Show that if $X = [0, 1]$, there is a point $x$ such that $f(x) = x$. The point $x$ is called a \textit{fixed point} of $f$.
What happens if $X$ equals $[0, 1)$ or $(0, 1)$?]}

\begin{solution}
Let $X = [0, 1]$. Let us define $g\colon X \rightarrow [-1, 1]$ by $g(x) = f(x) - x$. To show that $f$ has a fixed point, it suffices to show that $g$ has a zero in $X$. 
Note that $g$ is continuous, as it is a difference of continuous functions $f(x)$ and $x$. \\

Assume without loss of generality that $f(0) \neq 0$ and $f(1) \neq 1$; otherwise, $0$ or $1$ would be a fixed point for $f$. 
Since $f\colon [0, 1] \rightarrow [0, 1]$ It follows that
\[
	g(0) = f(0) - 0 > 0 \text{ and } g(1) = f(1) - 1 < 0.
\]
Since $g$ is a continuous map from a connected space $[0, 1]$ to the ordered set $[-1, 1]$ and $g(0) < 0 < g(1)$, 
it follows by the Intermediate Value Theorem that there must exist some point $x \in (0, 1)$ such that $g(x) = 0$. 
By construction, this point $x$ in $X$ satisfies $f(x) - x = 0$, and so it is a fixed point of $f$. \\

Note that $f$ does not necessarily have a fixed point if $X$ equals $[0, 1)$ or $(0, 1)$. In the former case, consider the function
\[
	f\colon [0, 1) \rightarrow [0, 1) \text{ defined by } f(x) = \frac{x}{2} + \frac{1}{2}.
\]

This has no fixed point in $[0, 1)$: $\frac{x}{2} + \frac{1}{2} = x$ if and only if $x = 1 \notin [0, 1)$. Similarly, 
in the latter case, consider
\[
	f\colon (0, 1) \rightarrow (0,1) \text{ defined by } f(x) = \frac{x}{2}.
\]
This has no fixed point in $(0, 1)$: $\frac{x}{2} = x$ if and only if $x = 0 \notin (0, 1)$. 
\end{solution}
\end{document}
