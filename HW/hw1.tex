\documentclass[11pt]{article}
\usepackage{graphicx}
\usepackage{amsthm}
\usepackage{amsmath}
\usepackage{amssymb}
\usepackage[shortlabels]{enumitem}
\usepackage[margin=1in]{geometry}

\newenvironment{solution}
  {\renewcommand\qedsymbol{$\blacksquare$}\begin{proof}[Solution]}
  {\end{proof}}

\setlength\parindent{0pt}

\begin{document}

	\hrule
	\begin{center}
        \textbf{MATH104: Topology}\hfill \textbf{Fall 2023}\newline

		{\Large Homework 1}

		David Yang
	\end{center}

\hrule

\vspace{1em}

\textit{Chapter 1 (Set Theory and Logic) Problems.} \\

\underline{Section 2 (Functions), 2.5} \\

\textbf{In general, let us denote the \textit{identity function} for a set $C$ by $i_C$. That is,
define $i_C \colon C \rightarrow C$ to be the function given by the rule $i_C(x) = x$ for all $x \in C$. Given $f \colon A \rightarrow B$, we say that a function
$g$ is a \textit{left inverse} for $f$ if $g \circ f = i_A$; and we asy that $h \colon B \rightarrow A$ is a \textit{right inverse} for $f$ if
$f \circ h = i_B$.}

\begin{enumerate}[a)]
    \item \textbf{Show that if $f$ has a left inverse, $f$ is injective; and if $f$ has a right inverse, $f$ is surjective.}
    \item \textbf{Give an example of a function that has a left inverse but no right inverse.}
    \item \textbf{Give an example of a function that has a right inverse but no left inverse.}
    \item \textbf{Can a function have more than one left inverse? More than one right inverse?}
    \item \textbf{Show that if a function $f$ has both a left inverse $g$ and right inverse $h$, then $f$ is bijective and $g = h = f^{-1}$.}
\end{enumerate}

\end{document}
