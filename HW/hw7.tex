\documentclass[11pt]{article}
\usepackage{graphicx}
\usepackage{amsthm}
\usepackage{amsmath}
\usepackage{amssymb}
\usepackage[shortlabels]{enumitem}
\usepackage[margin=1in]{geometry}

\newcommand{\R}{\mathbb{R}}
\newcommand{\Z}{\mathbb{Z}}

\newenvironment{solution}
  {\renewcommand\qedsymbol{$\blacksquare$}\begin{proof}[Solution]}
  {\end{proof}}

\setlength\parindent{0pt}

\begin{document}

	\hrule
	\begin{center}
        \textbf{MATH104: Topology}\hfill \textbf{Fall 2023}\newline

		{\Large Homework 7}

		David Yang
	\end{center}

\hrule

\vspace{1em}

\textit{Chapter 9 (The Fundamental Group) Problems.} \\

\underline{Section 60 (Fundamental Groups of Some Surfaces), 60.2} \\

\textbf{Let $X$ be the quotient space obtained from $B^2$ by identifying each point $x$ of $S^1$ with its antipode $-x$. 
Show that $X$ is homeomorphic to the projective plane $P^2$.} 

\begin{solution}

\end{solution}

\underline{Section 63 (The Jordan Curve Theorem), 63.1} \\

\textbf{Let $C_1$ and $C_2$ be disjoint simple closed curves in $S^2$.}

\begin{enumerate}[a)]
    \item \textbf{Show that $S^2 \text{ -- } C_1 \text{ -- } C_2$ has precisely three components. [\textit{Hint:} If $W_1$ is the component
    of $S^2 \text{ -- } C_1$ disjoint from $C_2$, and if $W_2$ is the component of $S^2 \text{ -- } C_2$ disjoint from $C_1$, show that
    $\overline{W}_1 \cup \overline{W}_2$ does not separate $S^2$.]}

    \item \textbf{Show that these three components have boundaries $C_1$ and $C_2$ and $C_1 \cup C_2$, respectively.}
    
\end{enumerate}

\end{document}
