\documentclass[11pt]{article}
\usepackage{graphicx}
\usepackage{amsthm}
\usepackage{amsmath}
\usepackage{amssymb}
\usepackage[shortlabels]{enumitem}
\usepackage[margin=1in]{geometry}

\newcommand{\R}{\mathbb{R}}

\newenvironment{solution}
  {\renewcommand\qedsymbol{$\blacksquare$}\begin{proof}[Solution]}
  {\end{proof}}

\setlength\parindent{0pt}

\begin{document}

	\hrule
	\begin{center}
        \textbf{MATH104: Topology}\hfill \textbf{Fall 2023}\newline

		{\Large Homework 2}

		David Yang
	\end{center}

\hrule

\vspace{1em}

\textit{Chapter 2 (Topological Spaces and Continuous Functions) Problems.} \\

\underline{Section 16 (The Subspace Topology), 16.5 (reformulated)} \\

\textbf{Let $X$ and $Y$ be two sets, each with two topologies. So we have $4$ topological spaces: $(X, T)$, $(X, T^{\prime})$, $(Y, U)$, $(Y, U^{\prime})$. 
Let $S$ be the product topology on $X \times Y$ induced by $T$ and $U$, and let $S^{\prime}$ be the product topology on $X \times Y$ induced by $T^{\prime}$ and $U^{\prime}$. 
(So a \textit{basis} for S consists of sets of the form $V \times W$ for $V$ in $T$ and $W$ in $U$,  but $S$ itself does not just equal $T \times U$).}

\begin{enumerate}[a)]
	\item \textbf{Suppose that $(X,T)$ is coarser than $(X, T^{\prime})$ and $(Y,U)$ is coarser than $(Y,U^{\prime})$. 
	Prove that $(X \times Y, S)$ is coarser than $(X \times Y, S^{\prime})$.}

	\begin{solution}
	Let $A \times B$ be an open set in $(X \times Y, S)$. Since $(X, T)$ is coarser than $(X, T^{\prime})$, it follows that $A \subset A^{\prime}$, where $A^{\prime}$ is an open
	set in $(X, S^{\prime})$. Similarly, since $(Y, U)$ is coarser than $(Y, U^{\prime})$, it follows that $B \subset B^{\prime}$ where $B^{\prime}$ is an open
	set in $(Y, U^{\prime})$. Thus, $A^{\prime} \times B^{\prime}$ is an open set in $(X \times Y, S^{\prime})$. Since every open set in $(X \times Y, S)$ is an open set in $(X \times Y, S^{\prime})$,
	it follows that $(X \times Y, S)$ is coarser than $(X \times Y, S^{\prime})$, as desired.
	\end{solution}
	
	\item \textbf{Suppose that $(X \times Y, S)$ is coarser than $(X \times Y, S^{\prime})$. 
	Does it necessarily follow that $(X, T)$ is coarser than $(X, T^{\prime})$ and that $(Y, U)$ is coarser than $(Y, U^{\prime})$?}

	\begin{solution}
	Let $(X \times Y, S)$ be coarser than $(X \times Y, S^{\prime})$. To show that $(X, T)$ is coarser than $(X, T^{\prime})$, we will show that any arbitrary open set in $(X, T)$ is also an open set in $(X, T^{\prime})$. 
	Let $A$ be a nonempty open set in $(X, T)$. We will show that $A$ is open in $(X, T^{\prime})$. \\

	By definition, since $A$ is open in $(X, T)$ and $Y$ is open in $U$, it follows that $A \times Y$ is open in the product topology $S$. Furthermore, since $(X \times Y, S)$ is coarser than $(X \times Y, S^{\prime})$, this means that
	$A \times Y$ is also open in the product topology $S^{\prime}$. Consequently, $A \times Y$ can be written as the union of nonempty open sets in $S^{\prime}$:
	\[
		A \times Y = \bigcup\limits_{\alpha \in J} V_{\alpha} \times W_{\alpha}
	\]
	where $V_{\alpha}$ is nonempty and open in $(X, T^{\prime})$ and $W_{\alpha}$ is nonempty open in $(Y, U^{\prime})$.\footnote{Note that we can assume that $V_\alpha$ and $W_\alpha$ are nonempty for each $\alpha$ as if either of them 
	were empty for a given $\alpha$, then the product $V_\alpha \times W_\alpha$ would be the empty set.} \\

	We claim that $A = \bigcup\limits_{\alpha \in J} V_\alpha$, and we will show containment in both directions. First, we will show that $A \in \bigcup\limits_{\alpha \in J} V_\alpha$. Let $x \in A$ and let $y$ be an arbitrary element in $Y$.
	Since $A \times Y = \bigcup\limits_{\alpha \in J} V_{\alpha} \times W_{\alpha}$ and $x \in A$, it follows that $x \times y \in \bigcup\limits_{\alpha \in J} V_{\alpha} \times W_{\alpha}$ and so $x \times y \in V_\alpha \times W_\alpha$ for at least one value of $\alpha$. Consequently, $x \in V_\alpha$ for that 
	value of $\alpha$, and so $x \in \bigcup\limits_{\alpha \in J} V_\alpha$. Thus, $A \subseteq \bigcup\limits_{\alpha \in J} V_\alpha$. \\

	For the other direction of containment, let $x \in \bigcup\limits_{\alpha \in J} V_\alpha$. Then $x \in V_\alpha$ for some value of $\alpha$. We can pick any arbitrary $y \in W_\alpha$ for that same value of $\alpha$, as $W_\alpha$ is nonempty. Then
	$x \times y \in \bigcup\limits_{\alpha \in J} V_{\alpha} \times W_{\alpha}$. Since $A \times Y = \bigcup\limits_{\alpha \in J} V_{\alpha} \times W_{\alpha}$, it follows that $x \times y \in A \times Y$ and so $x \in A$ as desired. \\

	Thus, by proving containment in both directions, we know $A = \bigcup\limits_{\alpha \in J} V_\alpha$. Since $A$ is a union of open sets $V_\alpha$ in $(X, T^{\prime})$, it follows that
	$A$ is also open in $(X, T^{\prime})$. \\

	Finally, since every open set in $(X, T)$ is also open in $(X, T^{\prime})$, we know that $(X, T)$ is coarser than $(X, T^{\prime})$. The same argument can be made
	to show that $(Y, U)$ is coarser than $(Y, U^{\prime})$. We conclude that if $(X \times Y, S)$ is coarser than $(X \times Y, S^{\prime})$, it follows that 
	$(X, T)$ is coarser than $(X, T^{\prime})$ and that $(Y, U)$ is coarser than $(Y, U^{\prime})$.
	\end{solution}
\end{enumerate}

\newpage

\underline{Section 17 (Closed Sets and Limit Points), 17.13} \\

\textbf{Show that $X$ is Hausdorff if and only if the diagonal $\triangle = \{ x \times x \mid x \in X \}$ is closed in $X \times X$.}

\begin{solution}
Suppose that $X$ is Hausdorff. Consider two distinct elements $x$ and $y$ in $X$. Since $X$ is Hausdorff, there exists 
disjoint open sets $A$ and $B$ in $X$ such that $x \in A$, $y \in B$. Since $A$ and $B$ are disjoint, they have no elements of $X$ in common. Consequently, it follows that $(A \times B) \cap \triangle = \varnothing$.
Since $A \times B$ is an open set $U$ in $X \times X$ containing $(x, y)$ that does not intersect $\triangle$, it follows that any point $(x, y)$ cannot be in the closure of $\triangle$, and so 
$\triangle = \overline{\triangle}$. Equivalently, $\triangle$ is closed. \\

The reverse implication follows similarly. Suppose that $\triangle$ is closed, so it is its own closure. Consequently, every point $(x, y) \in X \times X$ with $x \neq y$ is not in $\overline{\triangle}$.
It follows that there exists an open set $A \times B$ in $X \times X$ containing $(x, y)$ such that $(A \times B) \cap \triangle$ is empty. Equivalently, $A$ and $B$ must be disjoint open sets in $X$ where $x \in A$ and $y \in B$.
By definition, this tells us that $X$ is Hausdorff. \\

Thus, we conclude that $X$ is Hausdorff if and only if the diagonal $\triangle = \{ x \times x \mid x \in X \}$ is closed in $X \times X$. 
\end{solution}
\newpage

\underline{Section 17 (Closed Sets and Limit Points), 17.8} \\

\textbf{Let $A$, $B$, and $A_\alpha$ denote subsets of a space $X$. Determine whether the following equations hold; if an equality fails, determine whether
one of the inclusions $\subset$ or $\supset$ holds.}

\begin{enumerate}[a)]
	\item $\overline{A \cap B} = \overline{A} \cap \overline{B}$.
	
	\begin{solution}
	We claim that 
	\[
		\boxed{\overline{A \cap B} \subset \overline{A} \cap \overline{B}}.
	\] 
	Let $x \in \overline{A \cap B}$. By definition, every open set $U$ of $X$ containing $x$
	must intersect $A \cap B$. Consequently, $U$ must intersect both $A$ and $B$. Since every open set $U$ of $X$ containing $x$ intersects $A$, and every open set $U$ of $X$ containing $x$ intersects $B$, it follows
	that $x \in \overline{A}$ and $x \in \overline{B}$. Thus, $x \in \overline{A} \cap \overline{B}$. We conclude that $\overline{A \cap B} \subset \overline{A} \cap \overline{B}$. \\

	As a counterexample for the other direction of containment, let $A$ be the set of all positive real numbers, and let $B$ be the set of all negative real numbers; $A$ and $B$ are subsets of $\R$. Note that
	$\overline{A} \cap \overline{B} = \{ 0 \}$ whereas $\overline{A \cap B} = \overline{\varnothing} = \varnothing$. Thus, $\overline{A} \cap \overline{B} \not\subset \overline{A \cap B}$.
	\end{solution}

	\item $\overline{\bigcap A_\alpha} = \bigcap \overline{A}_\alpha$.
	
	\begin{solution}
	We claim that
	\[
		\boxed{\overline{\bigcap A_\alpha} \subset \bigcap \overline{A}_\alpha}.
	\]

	Let $x \in \overline{\bigcap A_\alpha}$. By definition, every open set $U$ of $X$ containing $x$ must intersect $\bigcap A_\alpha$. Since $U$ intersects the intersection of the $A_\alpha$,
	it must also intersect each $A_\alpha$. Thus, for each $\alpha$, $x \in \overline{A}_\alpha$; equivalently, $x \in \bigcap \overline{A}_\alpha$. We conclude that $\overline{\bigcap A_\alpha} \subset \bigcap \overline{A}_\alpha$. \\

	As a counterexample for the other direction of containment, we can refer to the example in part (a); let $A_1$ be the set of all positive real numbers, and let $A_2$ be the set of all negative real numbers; $A_1$ and $A_2$ are subsets of $\R$. 
	Since $\overline{A_1} \cap \overline{A_2} = \{ 0 \}$ whereas $\overline{A_1 \cap A_2} = \overline{\varnothing} = \varnothing$, we conclude that $\bigcap \overline{A}_\alpha \not\subset \overline{\bigcap A_\alpha}$.
	\end{solution}

	\item $\overline{A \text{ -- } B} = \overline{A} \text{ -- } \overline{B}$. 
	
	\begin{solution}
	We claim that
	\[
		\boxed{\overline{A} \text{ -- } \overline{B} \subset \overline{A \text{ -- } B}}.
	\]

	Let $x \in \overline{A} \text{ -- } \overline{B}$. By definition, every open set $U$ of $X$ containing $x$ must intersect $A$ and some open set $U^{\prime}$ containing $x$ will not intersect $B$. 
	However, every open set of $X$ containing $x$ will intersect $A \text{ -- } B$. Thus, $x \in \overline{A \text{ -- } B}$ and we conclude that $\overline{A} \text{ -- } \overline{B} \subset \overline{A \text{ -- } B}$. \\

	As a counterexample for the other direction of containment, let $A$ be the set of real numbers and let $B$ be the set of rationals; $A$ and $B$ are both subsets of $\R$. 
	Note that $\overline{A} = \R$, $\overline{B}  = \R$ (as the rationals are dense in the reals), and $\overline{A \text{ -- } B} = \R$ (as the irrationals are also dense in the reals). 
	Consequently, $\overline{A \text{ -- } B} = \R$ is not contained in $\overline{A} \text{ -- } \overline{B} = \varnothing$, and we conclude that $\overline{A \text{ -- } B} \not\subset \overline{A} \text{ -- } \overline{B}$. 	\end{solution}
\end{enumerate}
\end{document}
