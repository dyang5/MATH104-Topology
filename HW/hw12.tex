\documentclass[11pt]{article}
\usepackage{graphicx}
\usepackage{amsthm}
\usepackage{amsmath}
\usepackage{amssymb}
\usepackage[shortlabels]{enumitem}
\usepackage[margin=1in]{geometry}

\newcommand{\R}{\mathbb{R}}
\newcommand{\Z}{\mathbb{Z}}

\newenvironment{solution}
  {\renewcommand\qedsymbol{$\blacksquare$}\begin{proof}[Solution]}
  {\end{proof}}

\setlength\parindent{0pt}

\begin{document}

	\hrule
	\begin{center}
        \textbf{MATH104: Topology}\hfill \textbf{Fall 2023}\newline

		{\Large Homework 12}

		David Yang
	\end{center}

\hrule

\vspace{1em}

\textbf{Prove that if $G$ and $G^{\prime}$ are homeomorphic finite linear graphs, then they have the same Euler characteristic.}

\begin{solution}
Let $G$ and $G^{\prime}$ be homeomorphic finite linear graphs. We will first consider the case where $G$ and $G^{\prime}$ are both connected. 
Since $G$ and $G^{\prime}$ are homeomorphic, they have isomorphic fundamental groups. By Theorem 85.2, 
since $G$ and $G^{\prime}$ are both finite, connected linear graphs, the cardinality of a system of free generators for the fundamental groups for $G$ and $G^{\prime}$ are 
$1 - \chi(G)$ and $1 - \chi(G^{\prime})$, respectively. \\

We know that that the fundamental groups of $G$ and $G^{\prime}$, two finite linear graphs, must be free groups (if $G$ and $G^{\prime}$ are trees, their
fundamental groups are trivial, and we can realize these as a free group with $0$ generators).
Let the fundamental groups of $G$ and $G^{\prime}$ be $F_m$ and $F_n$, the free groups with $m$ and $n$ generators, respectively. If $F_m \cong F_n$, then
their abelianizations $\Z^m$ and $\Z^n$ must be isomorphic. By Theorem 67.8, it follows that $m = n$. Thus, the cardinalities
of the system of free generators for the fundamental groups of $G$ and $G^{\prime}$ must be equal, so
\[
    1 - \chi(G) = 1 - \chi(G^{\prime}),
\]
meaning $\chi(G) = \chi(G^{\prime})$ as desired. \\

We now consider the case where $G$ and $G^{\prime}$ are not connected. Since $G$ and $G^{\prime}$ are homeomorphic, there
is a homeomorphism between their connected components; consequently, $G$ and $G^{\prime}$ must also have the same number of connected components.
We know by the above logic that the Euler characteristics are the same for each pair of homeomorphic connected components $G_\alpha$ and $G^{\prime}_\alpha$.
Thus, summing over all connected components $G_\alpha$ and $G^{\prime}_\alpha$ using the property $\chi(G_\alpha) = \chi(G^{\prime}_\alpha)$, we find that 
\[
    \chi(G) = \sum_{\alpha \in J} \chi(G_\alpha) =  \sum_{\alpha \in J} \chi(G^{\prime}_\alpha) = \chi(G^{\prime}).
\]
Thus, if $G$ and $G^{\prime}$ are homeomorphic finite linear graphs, then they have the same Euler characteristic, as desired.
\end{solution}

\newpage

\textbf{Let $F = \langle a, b\rangle$ be the free group on two generators, and let $F^{\prime} = [F, F]$. We now know that $F^{\prime}$, as a subgroup of a free group, is free.
Find a set of free generators for $F^{\prime}$ by using covering space theory.}

\begin{solution}
Let $B$ represent the wedge of two circles; the fundamental group of $B$ at its intersection point is precisely $F$. 
Consider the integer lattice grid covering space $E$ of the wedge of $B$, where the loops $a$ and $b$ of $B$
lift to one unit horizontal and vertical shifts in $E$, and let $p$ represent the covering map from $E$ to $B$. Let $e_0 \in p^{-1}(b_0)$. 
We claim that $p_{\ast}(\pi_1(E, e_0)) = F^{\prime}$. \\

We will show containment in both directions. First, let $x$ be generated by the commutators of $F$, so that $x \in F^{\prime}$.
Since the powers of $a$ and $b$ on every commutator in $F$ sum to $0$, so do the powers of $a$ and $b$ in $x$, which is generated by the commutators. Consequently, when we
lift $x$ to $E$, the net horizontal and vertical shift from $e_0$ is $0$, and so we get a loop in $E$ based at $e_0$. Thus, $x \in p_{\ast}(\pi_1(E, e_0))$,
and we conclude that $F^{\prime} \subseteq p_{\ast}(\pi_1(E, e_0))$. \\

It remains to show that $p_{\ast}(\pi_1(E, e_0)) \subseteq F^{\prime}$. 
We will use results from Exercise 85.3, which followed from Theorem 84.7. As we did in Exercise 85.3, consider some maximal tree $T$
in $E$; one such maximal tree consists of all vertical grid lines and one horizontal grid line passing through $e_0$. From the result
of Exercise 85.3 and Theorem 84.7, we know that the system of free generators for the subgroup $p_{\ast}(\pi_1(E, e_0))$ 
is in bijective correspondence with each edge of $E$ not in $T$, and can be expressed as the set of all $a^m b^n a b^{-n} a^{-(m+1)}$, for integer $m$ and $n$. 
Note that each of these generators is itself a commutator:
\[
    a^m b^n a b^{-n} a^{-(m+1)} = (a^m b^n) a (b^{-n} a^{-m}) a^{-1} = (a^m b^n) a (a^m b^n)^{-1} a^{-1}.
\]
Consider some loop $\ell$ based at $e_0$ in $E$, and decompose it into $p_1 q_1 p_2 \dots q_r p_r$ where the $p$ terms represent
paths along the maximal tree $T$ and the $q$ terms represent edges of $E$ not in $T$. Note that the free generator of $p_{\ast}(\pi_1(E, e_0)$) corresponding to
the edge $q_1$ is a commutator, and traverses $p_1$. Similarly, for any $i \in \{ 1, \dots, r \}$, the free generator of $p_{\ast}(\pi_1(E, e_0))$ corresponding to
the edge $q_i$ is a commutator, and traverses $p_i$. Thus, the image of the induced homomorphism of the loop $\ell$ based at $e_0$ can be represented as a product
of commutators, and so if $x \in p_{\ast}(\pi_1(E, e_0))$, then $x \in F^{\prime}$. \\

We have shown that $p_{\ast}(\pi_1(E, e_0)) = F^{\prime}$. From Exercise 85.3, we know the system of free generators for $p_{\ast}(\pi_1(E, e_0))$ --
the set of elements of the form $a^m b^n a b^{-n} a^{-(m+1)}$ for integer $m, n$. Thus, since $F^{\prime} = p_{\ast}(\pi_1(E, e_0))$, we know
that a set of free generators for $F^{\prime}$ is simply the same set of free generators for $p_{\ast}(\pi_1(E, e_0))$, namely
\[
    \boxed{\{ a^m b^n a b^{-n} a^{-(m+1)} \mid m, n \in \Z\}}.
\]
\end{solution}

\newpage

\underline{Section 85 (Subgroups of Free Groups), 85.2} \\

\textbf{Let $F$ be a free group on two generators $\alpha$ and $\beta$. Let $H$ be the subgroup generated by $\alpha$. Show that $H$ has infinite index in $F$.}

\begin{solution}
Suppose for the sake of contradiction that $H$ has finite index $k > 0$ in $F$. By Theorem 85.3, since $F$ has $1 + 1 = 2$ free generators, 
$H$ must have $k + 1$ free generators. However, $H$ has one free generator, meaning $k = 0$, which contradicts the assumption of finite index. 
Thus, $H$ has infinite index in $F$.
\end{solution}

\begin{solution}
Note that any element $\beta^i$ for $i \in \Z^+$ gives a distinct coset $\beta^i H$ of $F$. Thus, since there are infinitely
many such elements (which are not necessarily even representative of all cosets of $H$ in $F$), $H$ must have infinite index in $F$.
\end{solution}
\end{document}
