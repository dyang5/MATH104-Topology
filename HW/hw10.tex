\documentclass[11pt]{article}
\usepackage{graphicx}
\usepackage{amsthm}
\usepackage{amsmath}
\usepackage{amssymb}
\usepackage[shortlabels]{enumitem}
\usepackage[margin=1in]{geometry}

\newcommand{\R}{\mathbb{R}}
\newcommand{\Z}{\mathbb{Z}}

\newenvironment{solution}
  {\renewcommand\qedsymbol{$\blacksquare$}\begin{proof}[Solution]}
  {\end{proof}}

\setlength\parindent{0pt}

\begin{document}

	\hrule
	\begin{center}
        \textbf{MATH104: Topology}\hfill \textbf{Fall 2023}\newline

		{\Large Homework 10}

		David Yang
	\end{center}

\hrule

\vspace{1em}

\textit{Chapter 12 (Classification of Surfaces) Problems.} \\

\underline{Section 74 (Classification of Surfaces), 74.7} \\

\textbf{If $m > 1$, show the fundamental group of the $m$-fold projective plane is not abelian. [\textit{Hint}: There is a homomorphism mapping this group onto the group $\Z / 2 * \Z / 2$.]}


\newpage

\underline{Section 78 (Constructing Compact Surfaces), 78.2(a)(b)} \\

\textbf{Let $H^2$ be the subspace of $\R^2$ consisting of all points $(x_1, x_2)$ with $x_2 \geq 0$. A 2-\textit{manifold with boundary (or surface with boundary)}
is a Hausdorff space $X$ with a countable basis such that each point $x$ of $X$ has a neighborhood homeomorphic with an open set of $\R^2$ or $H^2$.
The \textit{boundary} of $X$ (denoted $\partial X$) consists of those points $x$ such that $x$ has no neighborhood homeomorphic with an open set of $\R^2$.}

\begin{enumerate}[a)]
    \item \textbf{Show that no point of $H^2$ of the form $(x_1, 0)$ has a neighborhood (in $H^2$) that is homeomorphic to an open set of $\R^2$.}
    \item \textbf{Show that $x \in \partial X$ if and only if there is a homeomorphism $h$ mapping a neighborhood of $x$ onto an open set of $H^2$ such that $h(x) \in \R \times 0$.}
\end{enumerate}

\end{document}
