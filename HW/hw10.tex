\documentclass[11pt]{article}
\usepackage{graphicx}
\usepackage{amsthm}
\usepackage{amsmath}
\usepackage{amssymb}
\usepackage[shortlabels]{enumitem}
\usepackage[margin=1in]{geometry}

\newcommand{\R}{\mathbb{R}}
\newcommand{\Z}{\mathbb{Z}}

\newenvironment{solution}
  {\renewcommand\qedsymbol{$\blacksquare$}\begin{proof}[Solution]}
  {\end{proof}}

\setlength\parindent{0pt}

\begin{document}

	\hrule
	\begin{center}
        \textbf{MATH104: Topology}\hfill \textbf{Fall 2023}\newline

		{\Large Homework 10}

		David Yang
	\end{center}

\hrule

\vspace{1em}

\textit{Chapter 12 (Classification of Surfaces) Problems.} \\

\underline{Section 74 (Classification of Surfaces), 74.7} \\

\textbf{If $m > 1$, show the fundamental group of the $m$-fold projective plane is not abelian. [\textit{Hint}: There is a homomorphism mapping this group onto the group $\Z / 2 * \Z / 2$.]}

\begin{solution}
Let $m > 1$. $G$ be the free group on the set $\{ \alpha_1, \dots, \alpha_m \}$ and let $H$ be the group $\Z / 2 * \Z / 2$, which we define
to have generators $\gamma$ and $\delta$ satisfying $\gamma^2 = \delta^2 = 1$.
Consider the homomorphism $\varphi$ of $G$ onto $H$ defined by
\[
    \varphi(\alpha_i) = \begin{cases}
        \gamma &\text{ if $i = 1$}  \\
        \delta &\text{ otherwise}
    \end{cases}.
\]
\textit{(Defining where the generators of $G$ are mapped defines the entire homomorphism).}\\

 Note that under the homomorphism $\varphi$, we have that
\[
    \varphi(\alpha_1^2 \dots \alpha_m^2) = \varphi(\alpha_1)^2 \cdots \varphi(\alpha_m)^2 = \gamma^2 \cdots \delta^2 = 1,
\]
as both $\gamma$ and $\delta$ are defined to have order $2$. \\

Consider the least normal subgroup $N$ of $G$ generated by $\alpha_1^2 \dots \alpha_m^2$. 
By the Very Useful Lemma, since $\varphi$ is a homomorphism from $G$ to $H$ vanishing on the normal subgroup $N$, we have a well-defined homomorphism $\psi\colon G / N \rightarrow H$ mapping $gN$ to $\varphi(g)$. \\

Note that the domain $G/N$ of $\psi$ is isomorphic to the fundamental group of the $m$-fold projective plane. Thus, $\psi$ is a homomorphism
from the fundamental group of the $m$-fold projective plane onto the group $H \cong \Z / 2 * \Z / 2$, which is not abelian. \\

Take two elements $h_1$ and $h_2$ in $H$ where $h_1h_2 \neq h_2h_1$, and let $g_1$ and $g_2$ be elements in $G/N$ such that $\psi(g_1) = h_1$ and $\psi(g_2) = h_2$ ($g_1$ and $g_2$ exist as $\psi$ is surjective). 
If the fundamental group of the $m$-fold projective plane were abelian, then $g_1g_2 = g_2g_1$, so we would have
\[
    h_1 h_2 = \psi(g_1)\psi(g_2) = \psi(g_1g_2) = \psi(g_2g_1) = \psi(g_2)\psi(g_1) = h_2 h_1,
\]
contradicting the fact that $h_1$ does not commute with $h_2$. Thus, for $m > 1$, the fundamental group of the $m$-fold projective plane is not abelian.
\end{solution}

\newpage

\underline{Section 74 (Classification of Surfaces), 74.7} \\

\textbf{Let $X$ be the quotient space obtained from an $8$-sided polygonal region $P$ by pasting its edges together according to the labelling scheme $acadbcb^{-1}d$.}

\begin{enumerate}[a)]
    \item \textbf{Check that all vertices of $P$ are mapped to the same point of the quotient space $X$ by the pasting map.}
    
    \begin{solution}
    Consider the vertex corresponding that is the base of the edge $a$. This vertex is directly identified to the end of edges $c$ and $d$.
    The end of edge $c$ is identified to the end of edge $b$, and the end of edge $d$ is identified to the base of $b$. Finally, the 
    base of edge $b$ is identified to the base of edge $d$, which is in turn identified to the end of edge $a$, which is identified to the base of edge $c$. \\
    
    Thus, all vertices of $P$ are mapped to the same point of the quotient space $X$ by the pasting map.
    \end{solution}

    \item \textbf{Calculate $H_1(X)$.}
    
    \begin{solution}
    By Theorem 74.2, since the pasting map identifies all vertices of $P$ to the same point of the quotient space, $\pi_1(X, x_0)$ is isomorphic to the quotient 
    of the free group on the four generators $a, b, c, d$ (which we will denote $F$) by the least normal subgroup containing the element $acadbcb^{-1}d$. We will
    use $x$ to denote the element $acadbcb^{-1}d$ and $N$ to denote the least normal subgroup containing $x$. \\

    Since $\pi_1(X, x_0) = F/N$, by Corollary 75.2, we have that
    \[
        H_1(X) \cong \frac{\pi_1(X, x_0)}{[\pi_1(X, x_0), \pi_1(X, x_0)]} \cong \frac{(F / [F, F])}{ \langle p(N) \rangle}
    \]
    where $p$ is the projection map from $F$ to $F/[F, F]$. Note that $F / [F, F]$ is simply the free abelian group on four generators $a, b, c, d$, i.e. $\Z^4$. On the other hand, $p(N) = a^2c^2d^2$,
    the ``abelianized'' version of the element $x$. \\

    Thus, $H_1(X) \cong \Z^4 / \langle a^2c^2d^2\rangle.$ For convenience, let us think of $a = (1, 0, 0, 0), b = (0, 1, 0, 0), c = (0, 0, 1, 0)$, and $d = (0, 0, 0, 1)$,
    so $a^2c^2d^2 = (2, 0, 2, 2)$. By considering $\Z^4$ as a four-tuple of integers, one such generating set for $\Z^4$ is $(1, 0, 0, 0)$, $(0, 1, 0, 0), (0, 0, 1, 0)$, and $(1, 0, 1, 1)$. \\

    It follows that
    \begin{align*}
        H_1(X) &\cong \frac{[\langle (1, 0, 0, 0) \rangle \times \langle (0, 1, 0, 0) \rangle \times \langle (0, 0, 1, 0) \rangle \times \langle (1, 0, 1, 1) \rangle]}{\langle 2, 0, 2, 2\rangle} \\
        &\cong \langle (1, 0, 0, 0) \rangle \times \langle (0, 1, 0, 0) \rangle \times \langle (0, 0, 1, 0) \rangle \times \Z / 2\Z \\
        &\cong \boxed{\Z^3 \times \Z/2\Z}. \, \qedhere
    \end{align*} 
    \end{solution}
    
    \item \textbf{Assuming $X$ is homeomorphic to one of the surfaces given in Theorem 75.5 (which it is), which surface is it?}
    
    \begin{solution}
    The first homology group of the $4$-fold projective plane is precisely $\Z^3 \times \Z/2\Z$, so $X$ is homeomorphic to $\boxed{P_4}$, the $4$-fold projective plane. 
    \end{solution}
\end{enumerate}
\newpage


\underline{Section 78 (Constructing Compact Surfaces), 78.2(a)(b)} \\

\textbf{Let $H^2$ be the subspace of $\R^2$ consisting of all points $(x_1, x_2)$ with $x_2 \geq 0$. A 2-\textit{manifold with boundary (or surface with boundary)}
is a Hausdorff space $X$ with a countable basis such that each point $x$ of $X$ has a neighborhood homeomorphic with an open set of $\R^2$ or $H^2$.
The \textit{boundary} of $X$ (denoted $\partial X$) consists of those points $x$ such that $x$ has no neighborhood homeomorphic with an open set of $\R^2$.}

\begin{enumerate}[a)]
    \item \textbf{Show that no point of $H^2$ of the form $(x_1, 0)$ has a neighborhood (in $H^2$) that is homeomorphic to an open set of $\R^2$.}
    \begin{solution}
    Suppose for the sake of contradiction that there is a point $y$ of $H^2$ of the form $(x_1, 0)$ that has a neighborhood $U$ in $H^2$ that is homeomorphic to an open set $U^{\prime}$ of $\R^2$.  
    Let $h$ be the homeomorphism between $U$ and $U^{\prime}$. By construction, $h(y) \in U^{\prime}$. \\

    Since $U$ and $U^{\prime}$ are homeomorphic by assumption, removing a point from both $U$ and $U^{\prime}$ should preserve their homeomorphic nature. 
    However, this is not the case. Removing the point $y$ from $U$ preserves the simply-connectedness nature of $U$, and so the fundamental group of $U \setminus \{ y \}$ is trivial. 
    On the other hand, removing a point from an open set $U^{\prime}$ of $\R^2$ gives us a space which has fundamental group $\Z$. Thus, $U \setminus \{ y\}$ and $U^{\prime} \setminus \{\text{pt}\}$ are not homeomorphic, and
    so $U$ and $U^{\prime}$ cannot be homeomorphic. \\

    We conclude that there is no point of $H^2$ of the form $(x_1, 0)$ that has a neighborhood $U$ in $H^2$ that is homeomorphic to an open set $U^{\prime}$ of $\R^2$.  
    \end{solution}
    
    \item \textbf{Show that $x \in \partial X$ if and only if there is a homeomorphism $h$ mapping a neighborhood of $x$ onto an open set of $H^2$ such that $h(x) \in \R \times 0$.}
    \begin{solution}
    
    Suppose that $x \in \partial X$. Let $h$ be a homeomorphism from $U$ a neighborhood of $X$ to $U^{\prime}$, an open set of $H^2$. Suppose
    for the sake of contradiction that $h(x) \in \R \times c$ for some $c > 0$. Then $U^{\prime} \cap B_{\frac{c}{2}}(h(x))$ is an open set of $\R^2$, and 
    $h^{-1}(U \cap B_{\frac{c}{2}}(h(x)))$ is a neighborhood of $X$ homeomorphic to an open set of $\R^2$. By definition, since $x$ has a neighborhood homeomorphic
    with an open set of $\R^2$, then $x \notin \partial X$, contradicting our initial assumption. It follows that $h(x) \in \R \times 0$.
    Thus, if $x \in \partial X$, then there is a homeomorphism $h$ mapping a neighborhood of $x$ onto an open set of $H^2$ such that $h(x) \in \R \times 0$. \\
    
    On the other hand, suppose that there is a homeomorphism $h$ mapping a neighborhood of $x$ onto an open set of $H^2$ such that $h(x) \in \R \times 0$. From part (a),
    since $h(x) = (x_1, 0)$, $h(x)$ cannot have a neighborhood in $H^2$ that is homeomorphic to an open set of $\R^2$. By definition, then, $x \in \partial X$.
    \end{solution}
\end{enumerate}



\end{document}
