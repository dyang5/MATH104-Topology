\documentclass[11pt]{article}
\usepackage{graphicx}
\usepackage{amsthm}
\usepackage{amsmath}
\usepackage{amssymb}
\usepackage[shortlabels]{enumitem}
\usepackage[margin=1in]{geometry}

\newcommand{\R}{\mathbb{R}}
\newcommand{\Z}{\mathbb{Z}}

\newenvironment{solution}
  {\renewcommand\qedsymbol{$\blacksquare$}\begin{proof}[Solution]}
  {\end{proof}}

\setlength\parindent{0pt}

\begin{document}

	\hrule
	\begin{center}
        \textbf{MATH104: Topology}\hfill \textbf{Fall 2023}\newline

		{\Large Homework 8}

		David Yang
	\end{center}

\hrule

\vspace{1em}

\textit{Chapter 11 (The Seifert-van Kampen Theorem) Problems.} \\

\underline{Section 67 (Direct Sums of Abelian Groups), 67.4(b), (c)} \\

\textbf{The \textit{order} of an element $a$ of an abelian group $G$ is the smallest positive integer $m$ such that $ma = 0$, if such exists; otherwise, the order of $a$ is said to be infinite.
The order of $a$ thus equals the order of the subgroup generated by $a$.}

\begin{enumerate}[b)]
    \item \textbf{Show that if $G$ is free abelian, it has no nontrivial\footnote{added to problem for clarification.} elements of finite order.}
    
    \begin{solution}
    Since $G$ is free abelian, it has the elements $\{ a_\alpha \}$ as a basis, where each $a_\alpha$ generates an infinite cyclic subgroup $G_\alpha$. Let $a$ be a general element of $G$, and let its order be denoted by $m$, so that $ma = 0$. Since $G$
    is free abelian, it is by definition also the direct sum of the groups $\{ G_\alpha \}$, so 
    \[
        a = \sum_{\alpha_i} m_{\alpha_i} a_{\alpha_i} 
    \]
    where each $a_{\alpha_i}$ is a basis element of $G$ and so $m_{\alpha_i} a_{\alpha_i}$ is an element of the group $G_{\alpha_i}$. Multiplying both sides of the above equation by $m$ and using the fact that  of $a$ is $m$, we know tthe orderhat
    \[
        0 = ma = m \sum m_{\alpha_i} a_{\alpha_i} = \sum (mm_{\alpha_i}) a_{\alpha_i}.
    \]
    By the uniqueness caused by $G$ being a direct sum of the groups $G_\alpha$, we know that each $(mm_{\alpha_i}) a_{\alpha_i} = 0$; however, each $a_{\alpha_i}$ is of infinite order as they are generators of their respective infinite cyclic groups $G_{\alpha_i}$. 
    Consequently, we must have that $mm_{\alpha_i} = 0$ for each $\alpha_i$. \\

    Since $m > 0$ by construction, it follows that $m_{\alpha_i} = 0$ for each $\alpha_i$. Thus, $a = 0$, and so the only element of finite order in $G$ is $0$. We conclude that if $G$ is free abelian, it has no nontrivial elements of finite order.
    \end{solution}
    
    \item \textbf{Show the additive group of rationals has no elements of finite order, but is not free abelian. [\textit{Hint}: If $\{ a_\alpha \}$ is a basis, express $\frac{1}{2}a_\alpha$ in terms of this basis.]} 

    \begin{solution}
        The additive group of rationals has no element of finite order: if $q$ is an element in the additive group of rationals, then $mq = 0$ for a finite $m > 0$ if and only if $q = 0$. \\

        To show that the additive group of rationals is not free abelian, we will proceed with a proof by contradiction. Suppose for the sake of contradiction that the additive group of rationals is free abelian, in which case it has some basis $\{ a_\alpha \}$. 
        Consider an element $a_{\tilde{\alpha}}$ where $a_{\tilde{a}}$ is a basis element of the additive group of rationals. Since $\frac{1}{2}a_{\tilde{a}}$ is still rational, and the additive group of rationals is free abelian by assumption, we know
        \[
            \frac{1}{2}a_{\tilde{\alpha}} = \sum m_{\alpha_i} a_{\alpha_i}
        \]
        where each $a_{\alpha_i}$ is a basis element and each $m_{\alpha_i}$ is an integer. Multiplying both sides by $2$, we get that
        \[
            2\left(\frac{1}{2}a_{\tilde{\alpha}}\right)= 2\sum m_{\alpha_i} a_{\alpha_i}
        \]
        or equivalently, 
        \[
            a_{\tilde{\alpha}} = \sum 2m_{\alpha_i} a_{\alpha_i}.
        \]
        Since $a_{\tilde{\alpha}}$ is a basis element and it must be uniquely represented by a linear combination of basis elements as the additive group of rationals is by assumption free abelian,
        it follows that $2m_{\tilde{\alpha}} = 1$, in which case $m_{\tilde{\alpha}} = \frac{1}{2}$. This contradicts the fact that each $m_{\alpha_i}$ must be an integer. \\
        
        Thus, the additive group of rationals is not free abelian.
    \end{solution}
\end{enumerate}

\newpage

\underline{Section 68 (Free Product of Groups), 68.2} \\

\textbf{Let $G = G_1 * G_2$, where $G_1$ and $G_2$ are nontrivial groups.}

\begin{enumerate}[a)]
    \item \textbf{Show $G$ is not abelian.}
    
    \begin{solution}
    
    Let $x \in G_1$ and let $y \in G_2$; it follows that $x$ and $y$ are both elements of $G$. Note that $x * y$ and $y * x$ are both reduced words in $G$. Let $g = x*y$.
    Since $G$ is a free product between $G_1$ and $G_2$, by definition, there can only one unique reduced word in $G$ representing $g$. Thus, $g \neq y * x$, so
    \[
        x * y \neq y * x
    \]
    or equivalently, $G$ is not abelian.
    \end{solution}

    \item \textbf{If $x \in G$, define the \textit{length} of $x$ to be the length of the unique reduced word in the elements of $G_1$ and $G_2$ that represents $x$.
    Show that if $x$ has even length (at least $2$), then $x$ does not have finite order. 
    Show that if $x$ has odd length (at least $3$), then $x$ is conjugate to an element of shorter length.}

    \begin{solution}
    Let $x$ be a reduced word of even length. Without loss of generality, suppose that $x = g_1h_1\dots g_nh_n$ where $n \in \Z^+$ and each $g_i \in G_1$ and $h_i \in G_2$ for $i \in \{ 1, \dots, n \}$.
    If $x$ were to have finite order $m$, then $x^m$, which is $m$ copies of $x$ concatenated to each other, would be another reduced word (as each of its neighboring terms belongs to distinct groups $G_1$ and $G_2$) representing the identity. 
    Since $G$ is a free product, the identity must have a unique representation, so $x$ cannot have finite order. \\

    Let $x$ be a reduced word of odd length. Without loss of generality, suppose that $x = g_1h_1\dots g_n$ where $n$ is a positive integer greater than $1$ and each $g_i \in G_1$ and $h_i \in G_2$ for $i \in \{ 1, \dots, n \}$.
    Note that conjugating $x$ by $g_1^{-1}$ gives
    \[
        g_1^{-1}xg_1 = g_1^{-1} (g_1h_1\dots g_n) g_1 = (g_1^{-1} g_1) h_1\dots (g_n g_1) = h_1 \dots (g^{\prime}_n),
    \]
    where $g^{\prime}_n = g_n g_1$ is an element in $G_1$ due to the closure property of groups. Since $x$ was a reduced word of length $2n+1$ 
    and is conjugate to a reduced word $g_1^{-1}xg_1 = h_1 \dots (g^{\prime}_n)$ of length $2n$,
    we know that any reduced word of odd length (at least $3$) is conjugate to an element of shorter length, as desired.
    \end{solution}
    
    \item \textbf{Show that the only elements of $G$ that have finite order are the elements of $G_1$ and $G_2$ that have finite order, and their conjugates.}

    \begin{solution}
    Let $x$ be an element of $G$. If $x$ is of length $1$, it must be an element of $G_1$ or $G_2$, 
    in which case it has finite order if and only if it is an element $G_1$ or $G_2$ of finite order. 
    
    Suppose that $x$ has length greater than or equal to $2$. By part (b), if the unique reduced word of $x$ has even length, then it cannot have finite order. 
    On the other hand, if the unique reduced word of $x$ has odd order, we can continually conjugate it to an element of shorter length, following the procedure outlined in the solution to part (b). Since elements that are conjugate to each other have the same order,
    $x$ will have finite order in this case if and only if we can continually conjugate it to shorter and shorter elements of odd order; equivalently, $x$ must be conjugate to an element of length $1$, which are precisely the elements of $G_1$ and $G_2$. \\

    Thus, the only elements of $G$ that have finite order are the elements of $G_1$ and $G_2$ that have finite order, and their conjugates, as desired. \end{solution}
\end{enumerate}
\newpage

\underline{Section 68 (Free Product of Groups), 68.4} \\

\textbf{Prove Theorem 68.4 (Uniqueness of Free Products): Let $\{G_\alpha\}$ be a family of groups. Suppose $G$ and $G^{\prime}$ are groups
and $i_\alpha\colon G_\alpha \rightarrow G$ and $i^{\prime}_\alpha \colon G_\alpha \rightarrow G^{\prime}$ are families of monomorphisms, such that the families $\{ i_\alpha(G_\alpha) \}$ and $\{ i^{\prime}_\alpha(G_\alpha) \}$ generate $G$ and $G^{\prime}$, respectively.
If both $G$ and $G^{\prime}$ have the extension property stated in the preceding lemma (68.3), then there is a unique isomorphism $\varphi\colon G \rightarrow G^{\prime}$ such that $\varphi \circ i_\alpha = i^{\prime}_\alpha$ for all $\alpha$.}

\begin{solution}
Consider the group $H = G^{\prime}$ and the family of homomorphisms $h_\alpha = i^{\prime}_\alpha \colon G_\alpha \rightarrow G^{\prime}$. Since $G$ has the extension property stated in Lemma 68.3, it follows that 
there exists a homomorphism $\varphi \colon G \rightarrow G^{\prime}$ such that $\varphi \circ i_\alpha = i^{\prime}_\alpha$. \\

Similarly, consider the group $H = G$ and the family of homomorphisms $h_\alpha = i_\alpha \colon G_\alpha \rightarrow G$. Since $G^{\prime}$ has the extension property stated in Lemma 68.3,
it follows that there exist a homomorphism $\psi \colon G^{\prime} \rightarrow G$ with $\psi \circ i^{\prime}_\alpha = i_\alpha$. \\

Note that $\psi \circ \varphi \colon G \rightarrow G$ has the property that 
\[
    \psi \circ \varphi \circ i_\alpha = \psi \circ (\varphi \circ i_\alpha) = \psi \circ i^{\prime}_\alpha = i_\alpha.
\]
for each $\alpha$; since the identity map of $G$ has the same property, the uniqueness part of Lemma 68.3 tells us that $\psi \circ \varphi$ equal to the identity map of $G$. \\

Similarly, note that $\varphi \circ \psi \colon G^{\prime} \rightarrow G^{\prime}$ has the property that 
\[
    \varphi \circ \psi \circ i^{\prime}_\alpha = \varphi \circ (\psi \circ i^{\prime}_\alpha) = \varphi \circ i_\alpha = i^{\prime}_\alpha.
\]
for each $\alpha$; since the identity map of $G^{\prime}$ has the same property, the uniqueness part of Lemma 68.3 tells us that $\varphi \circ \psi$ equal to the identity map of $G^{\prime}$. \\

Thus, $\varphi$ and $\psi$ are inverses of each other, and so $\varphi: G \rightarrow G^{\prime}$ is the unique isomorphism such that $\varphi \circ i_\alpha = i^{\prime}_\alpha$ for all $\alpha$, as desired.
\end{solution}


\end{document}
