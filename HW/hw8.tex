\documentclass[11pt]{article}
\usepackage{graphicx}
\usepackage{amsthm}
\usepackage{amsmath}
\usepackage{amssymb}
\usepackage[shortlabels]{enumitem}
\usepackage[margin=1in]{geometry}

\newcommand{\R}{\mathbb{R}}
\newcommand{\Z}{\mathbb{Z}}

\newenvironment{solution}
  {\renewcommand\qedsymbol{$\blacksquare$}\begin{proof}[Solution]}
  {\end{proof}}

\setlength\parindent{0pt}

\begin{document}

	\hrule
	\begin{center}
        \textbf{MATH104: Topology}\hfill \textbf{Fall 2023}\newline

		{\Large Homework 7}

		David Yang
	\end{center}

\hrule

\vspace{1em}

\textit{Chapter 11 (The Seifert-van Kampen Theorem) Problems.} \\

\underline{Section 67 (Direct Sums of Abelian Groups), 67.4(b), (c)} \\

\textbf{The \textit{order} of an element $a$ of an abelian group $G$ is the smallest positive integer $m$ such that $ma = 0$, if such exists; otherwise, the order of $a$ is said to be infinite.
The order of $a$ thus equals the order of the subgroup generated by $a$.}

\begin{enumerate}[b)]
    \item \textbf{Show that if $G$ is free abelian, it has no elements of finite order.}
    
    \begin{solution}
    Since $G$ is free abelian, it has the elements $\{ a_\alpha \}$ as a basis, where each $a_\alpha$ generates an infinite cyclic subgroup $G_\alpha$. Let $a$ be a general element of $G$, and let its order be denoted by $m$, so that $ma = 0$. Since $G$
    is free abelian, it is by definition also the direct sum of the groups $\{ G_\alpha \}$, so 
    \[
        a = \sum_{\alpha_i} m_{\alpha_i} a_{\alpha_i} 
    \]
    where each $a_{\alpha_i}$ is a basis element of $G$ and so $m_{\alpha_i} a_{\alpha_i}$ is an element of the group $G_{\alpha_i}$. Multiplying both sides of the above equation by $m$ and using the fact that the order of $a$ is defined to be $m$, we know that
    \[
        0 = ma = m \sum m_{\alpha_i} a_{\alpha_i} = \sum (mm_{\alpha_i}) a_{\alpha_i}.
    \]
    By the uniqueness caused by $G$ being a direct sum of the groups $G_\alpha$, we know that each $(mm_{\alpha_i}) a_{\alpha_i} = 0$; however, each $a_{\alpha_i}$ is of infinite order as they are the generator of an infinite cyclic group $G_{\alpha_i}$. 
    Consequently, we must have that $mm_{\alpha_i} = 0$ for each $\alpha_i$. \\

    By definition, since $m > 0$, it follows that $m_{\alpha_i} = 0$ for each $\alpha_i$. Thus, $a = 0$, and so the only element of finite order in $G$ is $0$. We conclude that if $G$ is free abelian, it has no elements of finite order.
    \end{solution}
    \item \textbf{Show the additive group of rationals has no elements of finite order, but is not free abelian. [\textit{Hint}: If $\{ a_\alpha \}$ is a basis, express $\frac{1}{2}a_\alpha$ in terms of this basis.]} 
\end{enumerate}

\underline{Section 68 (Free Product of Groups), 68.4} \\

\textbf{Prove Theorem 68.4 (Uniqueness of Free Products): Let $\{G_\alpha\}$ be a family of groups. Suppose $G$ and $G^{\prime}$ are groups
and $i_\alpha\colon G_\alpha \rightarrow G$ and $i^{\prime}_\alpha \colon G_\alpha \rightarrow G^{\prime}$ are families of monomorphisms, such that the families $\{ i_\alpha(G_\alpha) \}$ and $\{ i^{\prime}_\alpha(G_\alpha) \}$ generate $G$ and $G^{\prime}$, respectively.
If both $G$ and $G^{\prime}$ have the extension property stated in the preceding lemma, then there is a unique isomorphism $\varphi\colon G \rightarrow G^{\prime}$ such that $\varphi \cdot i_\alpha = i^{\prime}_\alpha$ for all $\alpha$.}

\begin{solution}

\end{solution}

\end{document}
