\documentclass[11pt]{article}
\usepackage{graphicx}
\usepackage{amsthm}
\usepackage{amsmath}
\usepackage{amssymb}
\usepackage[shortlabels]{enumitem}
\usepackage[margin=1in]{geometry}

\newcommand{\R}{\mathbb{R}}
\newcommand{\Z}{\mathbb{Z}}

\newenvironment{solution}
  {\renewcommand\qedsymbol{$\blacksquare$}\begin{proof}[Solution]}
  {\end{proof}}

\setlength\parindent{0pt}

\begin{document}

	\hrule
	\begin{center}
        \textbf{MATH104: Topology}\hfill \textbf{Fall 2023}\newline

		{\Large Homework 14}

		David Yang
	\end{center}

\hrule

\vspace{1em}

\textit{Swarthmore Honors Exams Problems.} \\

\underline{2022, Problem 11} \\

\textbf{Let $X$ be the one point union of a torus and a $2$-sphere.}
\begin{enumerate}[a)]
    \item \textbf{Compute $\pi_1(X)$.}
    
    \begin{solution} 
    Let $x_0$ be the shared point between the torus $T$ and $2$-sphere $S$ in $X$. Consider 
    \[
        U = T \vee W_1, \, V = S \vee W_2
    \]
    where $W_1$ and $W_2$ are neighborhoods of $x_0$ in $S$ and $T$ that deformation retract to $x_0$, respectively. 
    Note that $U$ and $V$ are each open in $X$. Furthermore, the intersection $U \cap V = W_1 \vee W_2$ is simply connected (it deformation retracts to $x_0$
    which has trivial fundamental group and is path-connected -- for any two points in the intersection, there is either a path fully in $W_1$ or $W_2$ or there is a path in $W_1$ to $x_0$ to $W_2$ or vice versa). 
    By Seifert-Van Kampen, since $U \cap V$ is simply connected,
    the fundamental group of $X$ is isomorphic to the free product of the fundamental groups of $U$ and $V$. Thus, since $\pi_1(U, x_0) \cong \Z \times \Z$, and $\pi_1(V, x_0)$ is trivial, we 
    get that $\boxed{\pi_1(X) \cong \Z \times \Z}$. 
    \end{solution}

    \item \textbf{Describe the universal cover of $X$.}
    \begin{solution}
    The universal cover of $X$ is $\R \times \R$ with a $2$-sphere attached at each point in $\Z \times \Z$. \\

    It remains to show that this universal cover is simply connected. Note that it is path-connected; $\R^2$ is path-connected, and the union of
    a $2$-sphere at each point in $\Z \times \Z$ is a union of path-connected spaces at a common point, which is path-connected. On the other hand,
    the fundamental group of the universal cover is also trivial. Consider any loop in the universal covering space. It is composed of a number of components, which are either paths in $\R^2$ or
    loops around a $2$-sphere at an integer grid point. Since the $2$-sphere is simply connected, each of the loops around a $2$-sphere can be continuously deformed to a constant map at the point of intersection. Consequently,
    the loop in the universal covering space can deformation retract to a loop in $\R^2$ (by applying the above deformation to each part of the loop around a $2$-sphere). This loop in $\R^2$ similarly deformation retracts to the trivial loop at a given base point. 
    Thus, the fundamental group of our universal cover is trivial, and we have our universal cover.
    \end{solution}
\end{enumerate}

\newpage

\underline{2019, Problem 2} \\

\textbf{Let $X$ be the set of all functions $f\colon \R \rightarrow \R$. For each $t \in \R$, let $X_t$ be the subset of all $f \in X$ such that}
\[
    \sup\limits_{x \in \R} f(x) < t.
\]
\textbf{Let $\tau$ be the coarsest topology on $X$ that contains $X_t$ for each $t \in \R$.}
\begin{enumerate}[a)]
    \item \textbf{Show that $(X, \tau)$ is not Hausdorff.}
    \begin{solution}
    First, note that open sets in $(X, \tau)$ are simply the sets $X_t$; any union or intersection of such sets remain of the same form. \\

    Let $f(x) = x$ and let $g(x) = 1$. Note that since $\mathrm{sup}(f(x))$ is infinite, the only open set in $(X, \tau)$ containing $f(x)$ is $X$ itself. Thus, any open set containing $g(x)$ must also intersect $X$,
    the only open set containing $f(x)$, so $(X, \tau)$ is not Hausdorff.
    \end{solution}

    \item \textbf{Show that every subset $K \subseteq (X, \tau)$ that contains the identity function is compact.}
    
    \begin{solution}
    Let $K$ be a subset of $(X, \tau)$ containing the identity function. Following the above reasoning, since the supremum of the identity function defined over the reals is infinite, the only open set in $(X, \tau)$ containing
    the identity function is $X$ itself. Consequently, any open cover of $K$ must include $X$. Clearly, any open cover of $K$ will then have a finite subcover consisting of the open set $X$ itself.
    Thus, every subset $K \subseteq (X, \tau)$ containing the identity function is compact.
    \end{solution}

    \item \textbf{Let $Y \subseteq (X, \tau)$ be the subset of all constant functions. Is $Y$ compact? Is it connected? Is it path-connected?}
    
    \begin{solution}
    $Y$ is not compact. Consider the open cover $A = \{ X_t \mid t \in \R \}$ of $Y$. Let $X_{t_1}, \dots, X_{t_n}$ correspond to an arbitrary finite collection of open sets in $A$. Let us use $f_a$ to denote the constant function $f(x) = a$ for all $x \in \R$.
    Note that the constant function $f_{\max(t_1, \dots, t_n) + 1}$ in $Y$ is not covered by this finite collection of open sets, so $A$ has no finite subcovers. Thus, by definition, $Y$ is not compact. \\

    $Y$ is connected. Note that any two open sets in $X$ must intersect in $Y$ -- the open sets $X_a$ and $X_b$ both include the constant functions $f_{c}$ where $c < \mathrm{min}(a, b)$. Thus, there cannot be two open sets $U$ and $V$ of $Y$ such that
    $U$ and $V$ are disjoint satisfying $U \cup V = Y$, or equivalently, there is no separation of $Y$. Thus, $Y$ is connected. \\

    $Y$ is path-connected. Let $f_a$ and $f_b$ be an arbitrary pair of constant functions in $Y$. We will construct a path between them. Consider $\gamma \colon I \rightarrow Y$ defined by
    \[
        \gamma(t) = f_{(1-t)a + tb}.
    \]
    This path $\gamma$ is constructed from the straight-line homotopy between $f_a$ and $f_b$, which is the composition (sum and product) of continuous functions, so it is itself continuous. 
    Furthermore, $\gamma(0) = f_a$, $\gamma(1) = f_b$, and $\gamma(t) \in Y$ for all $t \in [0, 1]$.  Thus, $\gamma$ is a path from $f_a$ to $f_b$ in $Y$, so $Y$ is path-connected.\footnote{note that since every path-connected space is connected, it also follows that $Y$ is connected.}
    \end{solution}
\end{enumerate}

\end{document}
