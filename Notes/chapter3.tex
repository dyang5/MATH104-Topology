\newpage

\section{Chapter 3: Connectedness and Compactness}
\setcounter{subsection}{22}

\subsection{Connected Spaces}

\begin{definition}[Separation and Connected Spaces]
Let $X$ be a topological space. A \textbf{separation} of $X$ is a pair $U$, $V$ of disjoint nonempty open subsets of $X$
whose union is $X$. \\

The space is said to be \textbf{connected} if there does not exist a separation of $X$.
\end{definition}

\begin{remark}
An equivalent formulation of a connected space is that $X$ is a connected space if and only if the subsets of $X$
that are both open and closed in $X$ are the empty set and $X$ itself.
\end{remark}

\begin{lemma}
If $Y$ is a subspace of $X$, a separation of $Y$ is a pair of disjoint nonempty sets $A$ and $B$ whose union is $Y$, neither of which
contains a limit point of each other. The space $Y$ is connected if there exists no separation of $Y$.
\end{lemma}

\begin{eg}
The rationals are not connected: if $Y$ is a subspace of $\Q$ containing two points $p$ and $q$, one can choose
an irrational number $a \in (p, q)$ and write $Y$ as the union of the open sets 
\[
    Y \cap (-\infty, a) \text{ and } Y \cap (a, \infty),
\]
thus showing that there exists a separation of $Y$ and proving that $Y$ is not connected. \\

Consequently, the only connected subspaces of $\Q$ are the one-point sets.
\end{eg}

\begin{eg}
Consider the following subset of the plane $\R^2$:
\[
    X = \{ x \times y \mid y = 0 \} \cup \{ x \times y \mid x > 0 \text{ and } y = \frac{1}{x} \}. 
\]
Then $X$ is not connected, as neither subset contains a limit point of the other.
\end{eg}

\begin{lemma}
If the sets $C$ and $D$ form a separation of $X$ and if $Y$ is a connected subspace of $X$, then $Y$ lies entirely
within either $C$ or $D$.
\end{lemma}

\begin{theorem}
The union of a collection of connected subspaces of $X$ that have a point in common is connected.
\end{theorem}

\begin{theorem}
Let $A$ be a connected subspace of $X$. If $A \subset B \subset \overline{A}$, then $B$ is also connected.
\end{theorem}

\begin{theorem}
The image of a connected space under a continuous map is connected.
\end{theorem}

\begin{theorem}
A finite cartesian product of connected spaces is connected.    
\end{theorem}

\subsection{Connected Subspaces of the Real Line}

\begin{definition}[Linear Continuum]
A simply ordered set $L$ having more than one element is called a
\textbf{linear continuum} if the following hold:
\begin{enumerate}
    \item $L$ has the least upper bound property.
    \item If $x < y$, there exists $z$ such that $x < z < y$.
\end{enumerate}
\end{definition}

\begin{theorem}
If $L$ is a linear continuum in the order topology, then $L$
is connected and so are intervals and rays in $L$.
\end{theorem}

\begin{theorem}[Intermediate Value Theorem]
Let $f\colon X \rightarrow Y$ be a continuous map, where $X$ is a connected space and $Y$ is an ordered set in the order topology.
If $a$ and $b$ are two points of $X$ and if $r$ is a point of $Y$ lying between
$f(a)$ and $f(b)$, then there exists a point $c$ of $X$ such that $f(c) = r$.
\end{theorem}

\begin{definition}
Given points $x$ and $y$ of the space $X$, a \textbf{path} in $X$ from $x$ to $y$
is a continuous map $f\colon [a, b] \rightarrow X$ of some closed interval in the real line into $X$, such
that $f(a) = x$ and $f(b) = y$. \\

A space $X$ is said to be \textbf{path connected} if every pair of points of $X$ can be joined by a path in $X$.
\end{definition}

\begin{remark}
A path-connected space $X$ is connected; suppose that $X = A \cup B$ is a separation of $X$. Let $f\colon [a, b] \rightarrow X$ be any path in $X$. 
The set $f([a, b])$ is the image of a connected set, so it is connected, and must lie entirely in $A$ or $B$. Thus, there is no path $X$
joining a point of $A$ to a point of $B$, contradicting the assumption of path connectedness of $X$. \\

On the other hand, the converse does not hold.
\end{remark}