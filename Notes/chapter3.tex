\newpage

\section{Chapter 3: Connectedness and Compactness}
\setcounter{subsection}{22}

\subsection{Connected Spaces}

\begin{definition}[Separation and Connected Spaces]
Let $X$ be a topological space. A \textbf{separation} of $X$ is a pair $U$, $V$ of disjoint nonempty open subsets of $X$
whose union is $X$. \\

The space is said to be \textbf{connected} if there does not exist a separation of $X$.
\end{definition}

\begin{remark}
An equivalent formulation of a connected space is that $X$ is a connected space if and only if the subsets of $X$
that are both open and closed in $X$ are the empty set and $X$ itself.
\end{remark}

\begin{lemma}
If $Y$ is a subspace of $X$, a separation of $Y$ is a pair of disjoint nonempty sets $A$ and $B$ whose union is $Y$, neither of which
contains a limit point of each other. The space $Y$ is connected if there exists no separation of $Y$.
\end{lemma}

\begin{eg}
The rationals are not connected: if $Y$ is a subspace of $\Q$ containing two points $p$ and $q$, one can choose
an irrational number $a \in (p, q)$ and write $Y$ as the union of the open sets 
\[
    Y \cap (-\infty, a) \text{ and } Y \cap (a, \infty),
\]
thus showing that there exists a separation of $Y$ and proving that $Y$ is not connected. \\

Consequently, the only connected subspaces of $\Q$ are the one-point sets.
\end{eg}

\begin{eg}
Consider the following subset of the plane $\R^2$:
\[
    X = \{ x \times y \mid y = 0 \} \cup \{ x \times y \mid x > 0 \text{ and } y = \frac{1}{x} \}. 
\]
Then $X$ is not connected, as neither subset contains a limit point of the other.
\end{eg}

\begin{lemma}
If the sets $C$ and $D$ form a separation of $X$ and if $Y$ is a connected subspace of $X$, then $Y$ lies entirely
within either $C$ or $D$.
\end{lemma}

\begin{theorem}
The union of a collection of connected subspaces of $X$ that have a point in common is connected.
\end{theorem}

\begin{theorem}
Let $A$ be a connected subspace of $X$. If $A \subset B \subset \overline{A}$, then $B$ is also connected.
\end{theorem}

\begin{theorem}
The image of a connected space under a continuous map is connected.
\end{theorem}

\begin{theorem}
A finite cartesian product of connected spaces is connected.    
\end{theorem}

\subsection{Connected Subspaces of the Real Line}

\begin{definition}[Linear Continuum]
A simply ordered set $L$ having more than one element is called a
\textbf{linear continuum} if the following hold:
\begin{enumerate}
    \item $L$ has the least upper bound property.
    \item If $x < y$, there exists $z$ such that $x < z < y$.
\end{enumerate}
\end{definition}

\begin{theorem}
If $L$ is a linear continuum in the order topology, then $L$
is connected and so are intervals and rays in $L$.
\end{theorem}

\begin{theorem}[Intermediate Value Theorem]
Let $f\colon X \rightarrow Y$ be a continuous map, where $X$ is a connected space and $Y$ is an ordered set in the order topology.
If $a$ and $b$ are two points of $X$ and if $r$ is a point of $Y$ lying between
$f(a)$ and $f(b)$, then there exists a point $c$ of $X$ such that $f(c) = r$.
\end{theorem}

\begin{definition}
Given points $x$ and $y$ of the space $X$, a \textbf{path} in $X$ from $x$ to $y$
is a continuous map $f\colon [a, b] \rightarrow X$ of some closed interval in the real line into $X$, such
that $f(a) = x$ and $f(b) = y$. \\

A space $X$ is said to be \textbf{path connected} if every pair of points of $X$ can be joined by a path in $X$.
\end{definition}

\begin{remark}
A path-connected space $X$ is connected; suppose that $X = A \cup B$ is a separation of $X$. Let $f\colon [a, b] \rightarrow X$ be any path in $X$. 
The set $f([a, b])$ is the image of a connected set, so it is connected, and must lie entirely in $A$ or $B$. Thus, there is no path $X$
joining a point of $A$ to a point of $B$, contradicting the assumption of path connectedness of $X$. \\

On the other hand, the converse does not hold.
\end{remark}

\begin{eg}
The ordered square $I_o^2$ is connected but not path connected.
\end{eg} 
\vspace{1cm}
\begin{eg}
Let $S$ denote the following subset of the plane:
\[
    S = \{ x \times \sin \left( \frac{1}{x} \right) \mid 0 < x < 1\}. 
\]
$\overline{S}$ is known as the \textbf{topologist's sine curve}, and is not path-connected.
\end{eg}

\subsection{Components and Local Connectedness}

\begin{definition}[Components]
Given $X$, define an equivalence relation on $X$ by setting $x \sim y$ if there is a connected subspace of $X$ containing both $x$ and $y$. \\

The equivalence classes are the \textbf{components} (or connected components) of $X$.
\end{definition}

\begin{theorem}
The components of $X$ are connected disjoint subspaces of $X$ whose union is $X$, such that each nonempty connected subspace of $X$ intersects only one of them.
\end{theorem}

\begin{definition}[Path Components]
Given $X$, define an equivalence relation on $X$ by setting $x \sim y$ if there is a path in $X$ from $x$ to $y$. \\

The equivalence classes form the \textbf{path components}.
\end{definition}

\begin{theorem}
The path components of $X$ are path-connected disjoint subspaces of $X$ whose union is $X$, such that each nonempty path-connected subspace of $X$ intersects only one of them.
\end{theorem}

\begin{remark}
Each component of $X$ is closed in $X$, since its closure is also connected. If $X$ has a finite number of components, each component is also open. \\

On the other hand, the path components of $X$ do not need not be open nor closed in $X$.
\end{remark}

\begin{eg}
Each component of the subspace $\Q$ of $\R$ is a single point -- none of these components are open in $\Q$. \\

$\overline{S}$ is a space with a single component and two path components ($S$ and $V = 0 \times [-1, 1]$). Note that $S$ in open in $\overline{S}$ but not closed, while $V$ is closed but not open.
\end{eg}
    
\begin{definition}[Local Connectedness]
A space $X$ is said to be \textbf{locally connected at $x$} if for every neighborhood $U$ of $x$, there is a connected neighborhood $V$ of $x$ contained in $U$.
If $X$ is locally connected at each of its points, it is said simply to be \textbf{locally connected}. \\

Similarly, a space $X$ is said to be \textbf{locally path connected at $x$} if for every neighborhood $U$ of $x$, there is a path connected neighborhood $V$ of $x$ contained in $U$.
If $X$ is locally path connected at each of its points, it is said simply to be \textbf{locally path connected}. \\
\end{definition}

\begin{eg}
The subspace $[-1, 0) \subset (0, 1]$ of $\R$ is not connected, but is locally connected. \\

The topologist's sine curve is connected but not locally connected. The rationals are neither connected nor locally connected.
\end{eg}

\begin{theorem}
A space $X$ is locally (path) connected if and only if for every open set $U$ of $X$, each (path) component of $U$ is open in $X$.
\end{theorem}

\begin{theorem}
If $X$ is a topological space, each path component of $X$ lies in a component of $X$. If $X$ is locally path connected, then the components and the path components of $X$ are the same.
\end{theorem}


\subsection{Compact Spaces}

\begin{definition}[Cover and Open Cover]
A collection $\mathscr{A}$ of subsets of a space $X$ \textbf{covers} $X$ if the union of the elements of $\mathscr{A}$ is equal to $X$. \\

It is an \textbf{open cover} if the subsets in $\mathscr{A}$ are open sets.
\end{definition}

\begin{definition}[Compact]
A space $X$ is \textbf{compact} if every open covering $\mathscr{A}$ of $X$ contains a finite subcollection that covers $X$.
\end{definition}

\begin{eg}
$\R$ is not compact but the subspace 
\[
    X = \{ 0 \} \cup \left\{ \frac{1}{n} \mid n \in \Z_+ \right\}  
\]
of $\R$ is compact.
\end{eg}

\begin{lemma}
Let $Y$ be a subspace of $X$. Then $Y$ is compact if and only if every covering of $Y$ by sets open in $X$ contains a finite subcollection covering $Y$.
\end{lemma}

\begin{theorem}
Every closed subspace of a compact space is compact.
\end{theorem}

\begin{theorem}
Every compact subspace of a Hausdorff space is closed.
\end{theorem}

\begin{lemma}
If $Y$ is a compact subspace of the Hausdorff space $X$ and $x_0$ in $Y$, then there exist disjoint open sets $U$ and $V$ of $X$ containing $x_0$ and $Y$, respectively. 
\end{lemma}

\begin{theorem}
The image of a compact space under a continuous map is compact.
\end{theorem}

\begin{theorem}
Let $f\colon X \rightarrow Y$ be a bijective continuous function. If $X$ is compact and $Y$ is Hausdorff, then $f$ is a homeomorphism.
\end{theorem}

\begin{theorem}
The product of finitely many compact spaces is compact.
\end{theorem}

\begin{proof}
Apply tube lemma (below), and take the union of finitely many tubes to be a finite subcollection covering the entire space.
\end{proof}

\begin{lemma}[Tube Lemma]
Consider the product space $X \times Y$, where $Y$ is compact. If $N$ is an open set of $X \times Y$ containing the slice $x_0 \times Y$ of $X \times Y$, then $N$
contains some tube $W \times Y$ about $x_0 \times Y$, where $W$ is a neighborhood of $x_0$ in $X$.    
\end{lemma}

\begin{theorem}[Tychnoff Theorem]
The product of infinitely many compact spaces is compact.
\end{theorem}

\begin{definition}[Finite Intersection Property]
A collection $\mathscr{C}$ of subsets of $X$ has the \textbf{finite intersection property} if for every finite subcollection 
\[
    \{ C_1, \dots, C_n \} 
\]
of $\mathscr{C}$, the intersection $C_1 \cap \dots \cap C_n$ is nonempty.
\end{definition}

\begin{theorem}
Let $X$ be a topological space. Then $X$ is compact if and only if for every collection $\mathscr{C}$ of closed sets of $X$ having the finite intersection property,
the intersection $\bigcap_{C \in \mathscr{C}} C$ of all the elements of $\mathscr{C}$ is nonempty.
\end{theorem}