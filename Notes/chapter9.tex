\newpage
\setcounter{section}{8}

\section{The Fundamental Group}

\setcounter{subsection}{50}

\subsection{Homotopy of Paths}

\begin{definition}[Homotopy]
If $f$ and $f^{\prime}$ are continuous maps of the space $X$ into the space $Y$, we say that $f$ is \textbf{homotopic} to $f^{\prime}$
if there is a continuous map $F\colon X \times I \rightarrow Y$ such that
\[
    F(x, 0) = f(x) \text{ and } F(x, 1) = f^{\prime}(x)
\]
for each $x$. (Here $I = [0, 1]$.) The map $F$ is called a \textbf{homotopy} between $f$ and $f^{\prime}$. \\

If $f$ is homotopic to $f^{\prime}$, we write $f \simeq f^{\prime}$. If $f \simeq f^{\prime}$ and $f^{\prime}$ is a constant map, we say that
$f$ is \textbf{nulhomotopic}.
\end{definition}

\begin{remark}
Think of a homotopy as a continuous one-parameter family of maps from $X$ to $Y$; the homotopy $F$ represents a continuous ``deforming'' of the map $f$ to $f^{\prime}$, 
as $t$ goes from $0$ to $1$.
\end{remark}

\begin{definition}[Path Homotopy]
Two paths $f$ and $f^{\prime}$, mapping the interval $I = [0, 1]$ into $X$, are said to be \textbf{path homotopic} if they have the same initial point
$x_0$ and the same final point $x_1$, and if there is a continuous map $F\colon I \times I \rightarrow X$ such that
\begin{align*}
    F(s, 0) = f(s) &\text{ and } F(s, 1) = f^{\prime}(s), \\
    F(0, t) = x_0 &\text{ and } F(1, t) = x_1,
\end{align*}
for each $s \in I$ and each $t \in I$. We call $F$ a \textbf{path homotopy} between $f$ and $f^{\prime}$. If $f$ is path homotopic to $f^{\prime}$, we write $f \simeq_p f^{\prime}$.
\end{definition}

\begin{remark}
The first condition says that $F$ is a homotopy between $f$ and $f^{\prime}$, or equivalently, that $F$ representsa continuous way of deforming the path $f$ to the path $f^{\prime}$. \\

The second condition says that for each $t$, the path $f_t$ definde by $f_t(s) = F(s, t)$ is a path from $x_0$ to $x_1$, or equivalently, the endpoints of path remain fixed during the deformation.
\end{remark}

\begin{lemma}
The relations $\simeq$ and $\simeq_p$ are equivalence relations.
\end{lemma}

\begin{eg}
Let $f$ and $g$ be any two maps of a space $X$ into $\R^2$. The map 
\[
    F(x, t) = (1-t) f(x) + tg(x)
\]
is known as the \textbf{straight-line homotopy} between them.
\end{eg}

\begin{remark}
If $f$ and $g$ are paths from $x_0$ to $x_1$, then $F$ is a path homotopy. \\

More generally, let $A$ be a convex subspace of $\R^n$. Then any two paths $f, g$ in $A$ from $x_0$ to $x_1$ are path homotopic in $A$, as
the straight-line homotopy $F$ between them has image set in $A$.
\end{remark}

\begin{definition}[Product]
If $f$ is a path in $X$ from $x_0$ to $x_1$, and if $g$ is a path in $X$ from $x_1$ to $x_2$, the \textbf{product} $f * g$ of $f$ and $g$ is the path $h$ given by
\[
    h(s) = \begin{cases}
        f(2s) &\text{ for } s \in \left[ 0, \frac{1}{2} \right] \\
        g(2s - 1) &\text{ for } s \in \left[ \frac{1}{2}, 1 \right]
    \end{cases}
\]
\end{definition}

\begin{remark}
The function $h$ is well-defined and continuous, by the pasting lemma. It is a path in $X$ from $x_0$ to $x_2$. \\

The product operation on paths induce a well-defined operation on path-homotopy classes: $[f] * [g] = [f*g]$.
\end{remark}

\begin{theorem}
The operation $*$ has the following properties:
\begin{enumerate}
    \item (Associativity) If $[f] * ([g] * [h])$ is defined, so is $([f] * [g]) * [h]$, and they are equal.
    \item (Right and left identities) Given $x \in X$, let $e_x$ denote the constant path $e_x\colon I \rightarrow X$ carrying all of $I$ to the point $x$.
    If $f$ is a path from $x_0$ to $x_1$, then 
    \[
        [f] * [e_{x_1}] = [f] \text{ and } [e_{x_0}] * [f] = [f].
    \]
    \item (Inverse) Given the path $f$ in $X$ from $x_0$ to $x_1$, let $\overline{f}$ be the path defined by $\overline{f}(s) = f(1-s)$. It is the \textbf{reverse} of $f$. Then
    \[
        [f] * [\overline{f}] = [e_{x_0}] \text{ and } [\overline{f}] * [f] = [e_{x_1}].
    \]
\end{enumerate}
\end{theorem}

\begin{theorem}
Let $f$ be a path in $X$ and let $a_0, \dots, a_n$ be numbers such that $0 = a_0 < a_1 < \dots < a_n = 1$. Let $f_i\colon I \rightarrow X$ be the path that equals the positive linear map of $I$ onto $[a_{i-1}, a_i]$ followed by $f$. Then
\[
    [f] = [f_1] * \dots * [f_n].
\]
\end{theorem}


\subsection{The Fundamental Group}

\begin{definition}[Algebra Review]
A \textbf{homomorphism} $f\colon G \rightarrow G^{\prime}$ is a map such that $f(x \cdot y) = f(x) \cdot f(y)$. (It follows that $f(e) = e^{\prime}$ and $f(x^{-1}) = f(x)^{-1}$.) \\

The \textbf{kernel} of $f$ is the set $f^{-1}(e^{\prime})$ and is a subgroup of $G$. The image is also a subgroup of $G$. \\

The homomorphism $f$ is called a \textbf{monomorphism} if it is injective (or equivalently, the kernel consists of $e$ alone), an \textbf{epimorphism} if it is surjective,
and an \textbf{isomorphism} if it is bijective.
\end{definition}

\begin{definition}[Quotient Group]
If $H$ is a normal subgroup of $G$, the operation $(xH) \cdot (yH) = (x \cdot y)H$ is a well-defined operation on $G/H$ that makes it a group.
This is known as the \textbf{quotient} of $G$ by $H$.
\end{definition}

\begin{remark}
The map $f\colon G \rightarrow G/H$ mapping $x$ to $xH$ is an epimorphism with kernel $H$. \\

Conversely, if $f\colon G \rightarrow G^{\prime}$ is an epimorphism, then its kernel $N$ is a normal subgroup of $G$ and $f$ induces an isomorphism
$G/N \rightarrow G^{\prime}$ that carries $xN$ to $f(x)$ for each $x \in G$.
\end{remark}

\begin{definition}[The Fundamental Group]
Let $X$ be a space; let $x_0$ be a point of $X$. A path in $X$ that begins and ends at $x_0$ is called a \textbf{loop} based at $x_0$. \\

The set of path homotopy classes of loops based at $x_0$, with the operation $*$, is the \textbf{fundamental group} of $X$ relative to the \textbf{base point} $x_0$,
and is denoted by $\pi_1(X, x_0)$. This group is also sometimes called the \textbf{first homotopy group} of $X$.
\end{definition}

\begin{eg}
$\pi_1(\R^n, x_0)$ is the trivial group -- if $f$ is a loop in $\R^n$ based at $x_0$, the straight-line homotopy is a path homotopy between $f$
and the constant path at $x_0$. \\

If $X$ is any convex subset of $\R^n$, then $\pi_1(X, x_0)$ is the trivial group.
\end{eg}

\begin{definition}
Let $\alpha$ be a path in $X$ from $x_0$ to $x_1$. Define the map
\[
    \hat{\alpha} \colon \pi_1(X, x_0) \rightarrow \pi_1(X, x_1)
\]
by the equation
\[
    \hat{\alpha}([f]) = [\overline{\alpha}] * [f] * [\alpha].
\]
\end{definition}

\begin{theorem}
$\hat{\alpha}$ is a group isomorphism.
\end{theorem}

\begin{corollary}
If $X$ is path connected and $x_0$ and $x_1$ are two points of $X$, then $\pi_1(X, x_0)$ is isomorphic to $\pi_1(X, x_1)$.
\end{corollary}

\begin{definition}[Simply Connected]
A space $X$ is said to be \textbf{simply connected} if it is a path-connected space and if $\pi_1(X, x_0)$ is the trivial one-element group
for some $x_0 \in X$, and hence for every $x_0 \in X$. (The trivial fundamental group is often expressed as $\pi_1(X, x_0) = 0$.)
\end{definition}


\begin{lemma}
In a simply connected space $X$, any two paths having the same initial and final points are path homotopic.
\end{lemma}

\begin{definition}
Let $h\colon (X, x_0) \rightarrow (Y, y_0)$ be a continuous map. Define 
\[
    h_{\ast} \colon \pi_1(X, x_0) \rightarrow \pi _1(Y, y_0)
\]
by the equation $h_{\ast}([f]) = [h \circ f]$. The map $h_{\ast} $ is called the \textbf{homomorphism induced by $h$}, relative to the base point $x_0$.
\end{definition}

\begin{theorem}
If $h\colon (X, x_0) \rightarrow (Y, y_0)$ and $k\colon (Y, y_0) \rightarrow (Z, z_0)$ is continuous, then $(k \circ h)_{\ast} = k_{\ast} \circ h_{\ast}$. If $i\colon (X, x_0) \rightarrow (X, x_0)$ is the identity map,
then $i_{\ast}$ is the identity homomorphism.
\end{theorem}

\begin{corollary}
If $h\colon (X, x_0) \rightarrow (Y, y_0)$ is a homeomorphism of $X$ with $Y$, then $h_{\ast} $ is an isomorphism of $\pi_1(X, x_0)$ with $\pi_1(Y, y_0)$.
\end{corollary}


\subsection{Covering Spaces}

\begin{definition}
Let $p\colon E \rightarrow B$ be a continuous surjective map. The open set $U$ of $B$ is said to be \textbf{evenly covered} by $p$ if the inverse image $p^{-1}(U)$ can be written as the
union of disjoint open sets $V_\alpha$ in $E$ such that for each $\alpha$, the restriction of $p$ to $V_\alpha$ is a homeomorphism of $V_\alpha$ onto $U$. \\

The collection $\{ V_\alpha \}$ is a partition of $p^{-1}(U)$ into \textbf{slices}.
\end{definition}

\begin{definition}[Covering Spaces]
Let $p\colon E \rightarrow B$ be a continuous surjective map. If every point $b$ of $B$ has a neighborhood $U$ that is evenly covered by $p$, then $p$ is a \textbf{covering map},
and $E$ is said to be a \textbf{covering space} of $B$.
\end{definition}

\begin{eg}
Let $X$ be any space; let $i\colon X \rightarrow X$ be the identity map. Then $i$ is a covering map. \\

More generally, let $E$ be the space $X \times \{ 1, 2, \dots, n \}$ consisting of $n$ disjoint copies of $X$. The map $p\colon E \rightarrow X$ given by $p(x, i) = x$ 
for all $i$ is a covering map.
\end{eg}

\begin{theorem}
The map $p\colon \R \rightarrow S^1$ given by 
\[
    p(x) = (\cos 2\pi x, \sin 2\pi x)
\]
is a covering map.
\end{theorem}

\begin{definition}[Local homeomorphism]
If $p\colon E \rightarrow B$ is a covering map, then $p$ is a \textbf{local homeomorphism} of $E$ with $B$. \\

That is, each point $e$ of $E$ has a neighborhood that is mapped homeomorphically by $p$ onto an open subset of $B$.
\end{definition}

\begin{theorem}
Let $p\colon E \rightarrow B$ be a covering map. If $B_0$ is a subspace of $B$, and if $E_0 = p^{-1}(B_0)$, then the map $p_0\colon E_0 \rightarrow B_0$ obtained by
restricting $p$ is a coveirng map.
\end{theorem}

\begin{theorem}
If $p\colon E \rightarrow B$ and $p^{\prime}\colon E^{\prime} \rightarrow B^{\prime}$ are covering maps, then
\[
    p \times p^{\prime} \colon E \times E^{\prime} \rightarrow B \times B^{\prime}
\]
is a covering map.
\end{theorem}

\subsection{The Fundamental Group of the Circle}

\begin{definition}[Lifting]
Let $p\colon E \rightarrow B$ be a map. If $f$ is a continuous mapping of some space $X$ into $B$, a \textbf{lifting} of $f$ is a map $\tilde{f}\colon X \rightarrow E$ such that $p \circ \tilde{f} = f$. 
\end{definition}


\begin{eg}
Consider the covering $p\colon \R \rightarrow S^1$ given by $p(x) = (\cos (2\pi x), \sin (2\pi x))$. \\

The path $f\colon [0, 1] \rightarrow S^1$ beginning at $b_0 = (1, 0)$ given by $f(s) = (\cos (\pi s), \sin (\pi s))$ lifts to the path $\tilde{f}(s) = \frac{s}{2}$ beginning at $0$ and ending at $\frac{1}{2}$.
\end{eg}

\begin{lemma}
Let $p\colon E \rightarrow B$ be a covering map, let $p(e_0) = b_0$. Any path $f\colon [0, 1] \rightarrow B$ beginning at $b_0$ has a unique lifting to a path $\tilde{f}$ in $E$ beginning at $e_0$.
\end{lemma}

\begin{lemma}
Let $p\colon E \rightarrow B$ be a covering map, let $p(e_0) = b_0$. Let the map $F\colon I \times I \rightarrow B$ be continuous, with $F(0, 0) = b_0$. There is a unique lifting of $F$ to a continuous map
\[
    \tilde{F}\colon I \times I \rightarrow E
\]
such that $\tilde{F}(0, 0) = e_0$. If $F$ is a path homotopy, then $\tilde{F}$ is a path homotopy.
\end{lemma}

\begin{theorem}
Let $p\colon E \rightarrow B$ be a covering map, let $p(e_0) = b_0$. Let $f$ and $g$ be two paths in $B$ from $b_0$ to $b_1$ and let $\tilde{f}$ and $\tilde{g}$ be their respective liftings to paths in $E$ beginning at $e_0$.
If $f$ and $g$ are path homotopic, then $\tilde{f}$ and $\tilde{g}$ end at the same point of $E$ and are path homotopic.
\end{theorem}

\begin{definition}[Lifting Correspondence]
Let $p\colon E \rightarrow B$ be a covering map and let $b_0 \in B$. Choose $e_0$ so that $p(e_0) = b_0$. Given an element $[f] \in \pi _1(X, x_0)$, let $\tilde{f}$ be the lifting of $f$ to a path in $E$ that begins at $e_0$. 
Let $\phi([f])$ denote the endpoint of $\tilde{f}(1)$ of $\tilde{f}$. Then $\phi$ is a well-defined set map
\[
    \phi\colon \pi _1(B, b_0) \rightarrow p^{-1}(b_0).
\]
$\phi$ is the \textbf{lifting correspondence} derived from the covering map $p$, and depends on the choice of $e_0$.
\end{definition}

\begin{theorem}
Let $p\colon E \rightarrow B$ be a covering map, let $p(e_0) = b_0$. If $E$ is path connected then the lifting correspondence 
\[
    \phi\colon \pi _1(B, b_0) \rightarrow p^{-1}(b_0)
\]
is surjective. If $E$ is simply connected, it is bijective.
\end{theorem}

\begin{theorem}
The fundamental group of $S^1$ is isomorphic to the additive group of integers.
\end{theorem}

\begin{theorem}
Let $p\colon E \rightarrow B$ be a covering map, let $p(e_0) = b_0$.
\begin{enumerate}[a)]
    \item The homomorphism $p_{\ast} \colon \pi _1(E, e_0) \rightarrow \pi _1(B, b_0)$ is a monomorphism (injective).
    \item Let $H = p_{\ast} (\pi _1(E, e_0))$. The lifting correspondence $\phi$ induces an injective map
    \[
        \Phi\colon \pi _1(B, b_0) / H \rightarrow p^{-1}(b_0)
    \]
    of the collection of right cosets of $H$ into $p^{-1}(b_0)$, which is bijective if $E$ is path connected.
    \item If $f$ is a loop in $B$ based at $b_0$, then $[f] \in H$ if and only if $f$ lifts to a loop in $E$ based at $e_0$.
\end{enumerate}
\end{theorem}


\subsection{Retractions and Fixed Points}

\begin{definition}
If $A \subset X$, a \textbf{retraction} of $X$ onto $A$ is a continuous map $r\colon X \rightarrow  A$ such that $r \mid A$ is the identity map of $A$. \\

If such a map exists, then $A$ is known as a \textbf{retract} of $X$.
\end{definition}

\begin{lemma}
If $A$ is a retract of $X$, then the homomorphism of fundamental groups induced by inclusion $j\colon A \rightarrow X$ is injective.
\end{lemma}

\begin{theorem}
There is no retraction of $B^2$ onto $S^1$.
\end{theorem}

\begin{proof}
The homomorphism induced by inclusion cannot be injective, as the fundamental group of $S^1$ is nontrivial and the fundamental group of $B^2$ is trivial.
\end{proof}

\begin{lemma}
Let $h\colon S^1 \rightarrow X$ be a continuous map. Then the following conditions are equivalent:
\begin{enumerate}[1)]
    \item $h$ is nulhomotopic.
    \item $h$ extends to a continuous map $k\colon B^2 \rightarrow X$.
    \item $h_{\ast}$ is the trivial homomorphism of fundamental groups.
\end{enumerate}
\end{lemma}

\begin{theorem}[Brouwer Fixed-Point Theorem for the Disc]
If $f\colon B^2 \rightarrow B^2$ is continuous, then there exists a point $x \in B^2$ such that $f(x) = x$.
\end{theorem}


\subsection{The Fundamental Theorem of Algebra}
\begin{theorem}[Fundamental Theorem of Algebra]
A polynomial equation
\[
    x^n + a_{n-1}x^{n-1} + \dots + a_1 x + a_0 = 0
\]
of degree $n > 0$ with real or complex coefficients has at least one (real or complex) root.
\end{theorem}

\subsection{The Borsuk-Ulam Theorem}

\begin{theorem}[Borsok-Ulam Theorem for $S^2$]
Given a continuous map $f\colon S^2 \rightarrow \R^2$, there is a point $x$ of $S^2$ such that $f(x) = f(-x)$. 
\end{theorem}

\begin{theorem}[The Bisection Theorem]
Given two bounded polygonal regions in $\R^2$, there exists a line in $\R^2$ that bisects both of them.
\end{theorem}

\subsection{Deformation Retracts and Homotopy Type}
\begin{definition}[Deformation Retracts]
Let $A$ be a subspace of $X$. $A$ is a \textbf{deformation retract} of $X$ if the identity map of $X$ is homotopic to a map that carries all of $X$ into $A$, such that each point of $A$ remains fixed during the homotopy. \\

This means that there is a continuous map $H \colon X \times I \rightarrow X$ such that $H(x, 0) = x$ and $H(x, 1) \in A$ for all $x \in X$, and $H(a, t) = a$ for all $a \in A$. \\

The homotopy $H$ is a \textbf{deformation retraction} of $X$ onto $A$. The map $r \colon X \rightarrow A$ defined by $r(x) = H(x, 1)$ is a retraction of $X$ onto $A$, and $H$ is a homotopy between the identity map of $X$ and the map $j \circ r$ where $j \colon A \rightarrow X$ is inclusion.
\end{definition}

\begin{theorem}[Isomorphism of Fundamental Groups under Deformation Retracts]
Let $A$ be a deformation retract of $X$; let $x_0 \in A$. Then the inclusion map
\[
    j \colon (A, x_0) \rightarrow (X, x_0)
\]
induces an isomorphism of fundamental groups.
\end{theorem}

\begin{eg}
Let $B$ denote the $z$-axis in $\R^3$ and consider $\R^3 - B$. It has the punctured $xy$-plane $(\R^2 - \mathbf{0}) \times 0$ as a deformation retract.
\end{eg}

\begin{eg}
$\R^2 - p - q$, the doubly punctured plane, has the figure eight space as a deformation retract.
\end{eg}

\begin{eg}
Another deformation retraction of the doubly punctured plane $\R^2 - p - q$ is the theta space
\[
    \theta = S^1 \cup (0 \times [-1, 1]).
\]
\end{eg}

\begin{definition}[Homotopy Equivalence]
Let $f \colon X \rightarrow Y$ and $g \colon Y \rightarrow X$ be continuous maps. Suppose that $g \circ f \colon X \rightarrow X$ is homotopic to the identity map of $X$, and the map $f \circ g \colon Y \rightarrow Y$ is homotopic to the identity map of $Y$. \\

Then the maps $f$ and $g$ are \textbf{homotopy equivalences} and each is said to be a \textbf{homotopy inverse} of the other. \\

Two spaces that are homotopy equivalent have the same $\textbf{homotopy type}$.
\end{definition}

\begin{theorem}
Let $f\colon X \rightarrow Y$ be continuous; let $f(x_0) = y_0$. If $f$ is a homotopy equivalence, then
\[
    f_{\ast} \colon \pi _1(X, x_0) \rightarrow \pi _1(Y, y_0)
\]
is an isomorphism.
\end{theorem}

\begin{remark}
The relation of homotopy equivalence is more general than the notion of deformation retraction. \\

Both the theta space and the figure eight spaces are deformation retracts of the doubly punctured plane, so they are homotopy equivalent to the doubly punctured plane and thus to each other. But neither is homeomorphic to a deformation retract of the other; in fact, neither of them can even be imbedded in each other. 
\end{remark}