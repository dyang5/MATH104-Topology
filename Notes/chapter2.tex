\newpage

\section{Chapter 2: Topological Spaces and Continuous Functions}
\setcounter{subsection}{11}

\subsection{Topological Spaces}
\begin{definition}[Topology]
A \textbf{topology} on a set $X$ is a collection $\mathscr{T}$ of subsets of $X$ having the following properties:
\begin{enumerate}
    \item $\varnothing$ and $X$ are in $\mathscr{T}$.
    \item The union of the elements of any subcollection of $\mathscr{T}$ is in $\mathscr{T}$.
    \item The intersection of the elements of any finite subcollection of $\mathscr{T}$ is in $\mathscr{T}$.
\end{enumerate}
A set $X$ for which a topology $\mathscr{T}$ has been specified is a \textbf{topological space}.
\end{definition}

\begin{definition}[Open Set]
If $X$ is a topological space with topology $\mathscr{T}$, a subset $U$ of $X$ is an \textbf{open set} of $X$
if $U$ belongs to the collection $\mathscr{T}$.  
\end{definition}

\begin{remark}
Using the open set terminology, a topological space can be thought of as a set $X$ together with a collection
of subsets of $X$, called \textit{open sets}, such that $\varnothing$ and $X$ are both open, and such that arbitrary unions
and finite intersections of open sets are open.
\end{remark}

\begin{eg}
The topologies on $X = \{ a, b, c \}$ can be specified by permuting $a$, $b$, and $c$ and specifying the open sets defined by subsets of $X$. \\

However, note that not every collection of subsets of $X$ is a topology on $X$: consider 
\[
    \{\{ a, b \}, \{ b, c \}, \{  a, b, c \}\}.
\]
This is not closed under intersection.
\end{eg}

\begin{eg}
Let $X$ be any set. The collection of all subsets of $X$ is a topology on $X$ known as the \textbf{discrete topology}. \\

The collection consisting of only $\varnothing$ and $X$ is also a topology on $X$ known as the \textbf{indiscrete topology} (or \textbf{trivial topology}).
\end{eg}

\begin{eg}
Let $X$ be a set and let $\mathscr{T}_f$ be the collection of all subsets $U$ of $X$ such that $X\text{ -- }U$ either is finite or all of $X$. Then $\mathscr{T}_f$ is a topology on $X$
known as the \textbf{finite complement topology}. \\

Similarly, let $X$ be a set and let $\mathscr{T}_c$ be the collection of all subsets $U$ of $X$ such that $X\text{ -- }U$ either is countable or is all of $X$. 
Then $\mathscr{T}_c$ is a topology on $X$.
\end{eg}

\begin{definition}[Finer and Coarser Topologies]
Suppose $\mathscr{T}$ and $\mathscr{T}^{\prime}$ are two topologies on a given set $X$. If $\mathscr{T}^{\prime} \subset \mathscr{T}$, $\mathscr{T}^{\prime}$ is \textbf{finer} than $\mathscr{T}$;
if $\mathscr{T}^{\prime}$ properly contains $\mathscr{T}$, then $\mathscr{T}^{\prime}$ is \textbf{strictly finer} than $\mathscr{T}$. \\

In the resepctive situations, we say that $\mathscr{T}$ is \textbf{coarser} than $\mathscr{T^{\prime}}$;
and $\mathscr{T}$ is \textbf{strictly coarser} than $\mathscr{T}^{\prime}$. \\

$\mathscr{T}$ is \textbf{comparable} with $\mathscr{T}^{\prime}$ if either $\mathscr{T}^{\prime} \subset \mathscr{T}$ or $\mathscr{T} \subset \mathscr{T}^{\prime}$.
\end{definition}

\begin{intuition}
Think of a topological space as something like a truckload full of gravel. Then the pebbles and all unions of collections of pebbles are the open sets.\\

Furthermore, if the pebbles are smashed into smaller pieces, the collection of open sets has been enlarged, adn the topology, like the gravel, is made finer by the operation.
\end{intuition}

\begin{remark}
The finer and coaser terminology can also be replaced by \textbf{larger} and \textbf{smaller} as well as \textbf{stronger} and \textbf{weaker}, respectively.
\end{remark}

\subsection{Basis for a Topology}

Instead of specifying a topology by describing the entire collection $\mathscr{T}$ of open sets, it is often easier to specify a smaller collection of subsets of $X$ and 
to define the topology from that collection.
\begin{definition}[Basis for Topology]
If $X$ is a set, a \textbf{basis} for a topology on $X$ is a collection $\mathscr{B}$ of subsets of $X$ (\textbf{basis elements}) such that
\begin{enumerate}
    \item For each $x \in X$, there is at least one basis element $B$ containing $x$.
    \item If $x$ belongs to the intersection of two basis elements $B_1$ and $B_2$, then there is a basis element $B_3$ containing $x$ such that $B_3 = B_1 \cap B_2$.
\end{enumerate}

If $\mathscr{B}$ satisfies these conditions, then the \textbf{topology $\mathscr{T}$ generated by $\mathscr{B}$} is defined as follows:
a subset $U$ of $X$ is said to be open in $X$ (or an element of $\mathscr{T}$) if for each $x \in U$, there is a basis element $B \in \mathscr{B}$ such that $x \in B$ and $B \subset U$
(each basis element is itself an element of $\mathscr{T}$).
\end{definition}

\begin{remark}
The collection $\mathscr{T}$ generated by the basis $\mathscr{B}$ satisfies the requirements for a topology on $X$: 
\begin{enumerate}
    \item The empty set and $X$ are both in $\mathscr{T}$
    \item $U = \bigcup_{\alpha \in J} U_\alpha$ for an indexed family $\{ U_\alpha \}_{\alpha \in J}$ of elements of $\mathscr{T}$ belongs to $\mathscr{T}$
    \item The intersection of two elements $U_1$ and $U_2$ in $\mathscr{T}$ is also in $\mathscr{T}$; the finite intersection case follows from induction.
\end{enumerate}
\end{remark}

\begin{eg}
Let $\mathscr{B}$ be the collection of all circular regions (interiors of circles) in the plane; $\mathscr{B}$ is a basis.
In the topology generated by $\mathscr{B}$, a subset $U$ of the plane is open if every $x \in U$ lies in some circular region contained in $U$.
\end{eg}

\begin{eg}
If $X$ is any set, the collection of all one-point subsets in $X$ is a basis for the discrete topology on $X$.
\end{eg}

\begin{lemma}
Let $X$ be a set; let $\mathscr{B}$ be a basis for a topology $\mathscr{T}$ on $X$. Then $\mathscr{T}$ equals the collection of
all unions of elements of $\mathscr{B}$.
\end{lemma}

\begin{remark}
The above lemma tells us that every open set $U$ in $X$ can be expressed as a union of basis elements. However, unlike the notion of a basis
in linear algebra, this expression may not be unique.
\end{remark}

\begin{lemma}[Obtaining a Basis from a Topology]
Let $X$ be a topological space. Suppose that $\mathscr{C}$ is a collection of open sets of $X$ such that for each open set $U$ of $X$ and each $x$ in $U$,
there is an element $C$ of $\mathscr{C}$ such that $x \in C \subset U$. Then $\mathscr{C}$ is a basis for the topology of $X$.
\end{lemma}

\begin{lemma}[Comparing Fineness of Topologies]
Let $\mathscr{B}$ and $\mathscr{B}^{\prime}$ be bases for the topologies $\mathscr{T}$ and $\mathscr{T}^{\prime}$, respectively, on $X$. Then the following are equivalent:
\begin{enumerate}
    \item $\mathscr{T}^{\prime}$ is finer than $\mathscr{T}$.
    \item For each $x \in X$ and each basis element $B \in \mathscr{B}$ containing $x$, there is a basis element $B^{\prime} \in \mathscr{B}^{\prime}$ such that $x \in B^{\prime} \subset B$.
\end{enumerate}
\end{lemma}


\begin{intuition}
Recall the analogy between a topological space and a truckload of gravel. Here, the pebbles are the basis elements of the topology; 
after the pebbles are smashed into dust, the dust particles are the basis elements for the new topology. 
The new topology is finer than the old one, and the dust particle was previously contained in a pebble.
\end{intuition}

\begin{remark}
The collection of all circular regions in the plane generates the same topology as the collection of all rectangular regions in the plane:
each rectangle in the rectangular basis has a circle enclosed in it, and each circle in the circular basis has a rectangle enclosed in it. 
This tells us that each topology is finer than the other, and so they generate the same topology.
\end{remark}

\begin{definition}[Topologies on $\R$]
Let $\mathscr{B}$ be the collection of open intervals on the real line; the \textbf{standard topology} on the real line is the topology generated by $\mathscr{B}$. \\

Let $\mathscr{B}^{\prime}$ be the collection of all half-open intervals $[a, b)$ on the real line; the \textbf{lower limit topology} on the real line 
is the topology generated by $\mathscr{B}^{\prime}$ and is denoted by $\R_\ell$. \\

Finally, let $K$ be the set of all numbers of the form $\frac{1}{n}$ for $n \in \Z_+$. Let $\mathscr{B}^{\prime\prime}$ be the collection
of all open intervals along with sets of the form $(a, b) \text{ -- } K$. The topology generated by $\mathscr{B}^{\prime\prime}$ is the \textbf{K-topology} on $\R$, denoted by $\R_k$.
\end{definition}

\begin{definition}[Subbasis for a Topology]
A \textbf{subbasis} $\mathscr{S}$ for a topology on $X$ is a collection of subsets of $X$ whose union equals $X$. \\

The \textbf{topology generated by the subbasis $\mathscr{S}$} is the collection $\mathscr{T}$ of all unions of finite intersections of elements of $\mathscr{S}$.
\end{definition}

\subsection{The Order Topology}

\begin{definition}[Order Topology]
Let $X$ be a set with a simple order relation and assume that $X$ has more than one element. Let $\mathscr{B}$ be the collection
of all sets of the following types:
\begin{enumerate}
    \item All open intervals $(a, b)$ in $X$.
    \item All intervals of the form $[a_0, b)$ where $a_0$ is the smallest element (if any) of $X$.
    \item All intervals of the form $(a, b_0]$ where $b_0$ is the largest element (if any) of $X$.
\end{enumerate}

The collection $\mathscr{B}$ is a basis for a topology on $X$, known as the \textbf{order topology}. \\

\textit{If $X$ has no smallest element, there are no sets of type $2$, and if $X$ has no largest element, there are no sets of type $3$.}
\end{definition}

\begin{eg}
The standard topology on $\R$ is the order topology derived from the usual order on $\R$. \\

The order topology on $\R \times \R$ in the dictionary order has as basis the collection of all intervals of the form $(a \times b, c \times d)$
for $a < c$ and for $a = c$ and $b < d$. The subcollection consisting of only intervals of the second type is also an order topology on $\R \times \R$.
\end{eg}

\begin{eg}
Consider the set $X = \{ 1, 2 \} \times \Z_+$ in the dictionary order, and denote $1 \times n$ as $a_n$ and $2 \times n$ by $b_n$. Then $X = \{ a_1, a_2, \dots ; b_1, b_2, \dots \}$. \\

Note that the order topology is not the discrete topology (of which the collection of one-point subsets is a basis), as the one-point set
$\{ b_1 \}$ is not open. Any open set containing $b_1$ must contain a basis element about $b_1$, and all basis elements of $b_1$ contain points in $a_i$.
\end{eg}

\begin{definition}[Rays of $X$]
If $X$ is an ordered set and $a$ is an element of $X$, there are four subsets of $X$ known as the \textbf{rays} determined by $a$: two open rays and two closed rays.
\end{definition}
\begin{remark}
The open rays in $X$ are open sets in the order topology. Furthermore, the open rays form a subbasis for the order topology on $X$.
\end{remark}

\subsection{The Product Topology on $X \times Y$}
\begin{definition}
Let $X$ and $Y$ be topological spaces. The \textbf{product topology} on $X \times Y$ is the topology having as basis the collection
of all sets of the form $U \times V$, where $U$ is an open subset of $X$ and $V$ is an open subset of $Y$.
\end{definition}

\begin{theorem}
If $\mathscr{B}$ is a basis for the topology of $X$ and $\mathscr{C}$ is a basis for the topology of $Y$ then the collection
\[
    \mathscr{D} = \{ B \times C \mid B \in \mathscr{B} \text{ and } C \in \mathscr{C} \} 
\]
is a basis for the topology of $X \times Y$.
\end{theorem}

\begin{eg}
The standard topology on $\R$ is the order topology. The product of this topology with itself is the \textbf{standard topology} on $\R \times \R$. \\

One such basis, by the above theorem, for the standard topology is the collection of all products $(a, b) \times (c, d)$ of open intervals in $\R$.
\end{eg}

\subsection{The Subspace Topology}

\begin{definition}
Let $X$ be a topological space with topology $\mathscr{T}$. If $Y$ is a subset of $X$, the collection
\[
    \mathscr{T}_Y = \{ Y \cap U \mid U \in \mathscr{T} \} 
\]
is a topology on $Y$, known as the \textbf{subspace topology}. \\

With this topology, $Y$ is a \textbf{subspace} of $X$; its open sets consist of all intersections of open sets of $X$ with $Y$.
\end{definition}

\begin{theorem}
If $A$ is a subspace of $X$ and $B$ is a subspace of $Y$, then the product topology on $A \times B$ is the same as the topology $A \times B$ inherits as a subspace of $X \times Y$.
\end{theorem}

\begin{remark}
Let $X$ be an ordered set in the order topology and let $Y$ be a subset of $X$. The order relation on $X$ when restricted to $Y$ makes $Y$ into an ordered set.
However, the \textit{resulting order topology on $Y$ need not be the same as the topology that $Y$ inherits as a subspace of $X$.}
\end{remark}

\begin{eg}
Consider the subset $Y = [0, 1]$ on $\R$ in the subspace topology. The subspace topology has as basis all sets of the form $(a, b) \cap Y$
for $(a, b) \in \R$, which is
\[
    (a, b) \cap Y = \begin{dcases}
        (a, b), &\text{ if $a$ and $b$ are in $Y$}  ;\\
        [0, b), &\text{ if only $b$ is in $Y$}  ;\\
        (a, 1], &\text{ if only $a$ is in $Y$} ;\\
        Y \text{ or } \varnothing, &\text{ if neither $a$ nor $b$ is in $Y$};
    \end{dcases}
\]

Each of these sets is open in $Y$, but sets of the second and third types are not open in $\R$. 
Furthermore, note that these sets form a basis for the order topology on $Y$, so
the order topology and subspace topology of $Y$ as a subspace of $\R$ are the same.
\end{eg}

\begin{definition}[Convex Subsets]
Given an ordered set $X$, the subset $Y$ of $X$ is \textbf{convex} in $X$ if for each pair of points $a < b$ of $Y$, the entire
interval $(a, b)$ of points of $X$ lies in $Y$.\\

\textit{Note: intervals and rays in $X$ are convex in $X$.}
\end{definition}

\begin{theorem}
Let $X$ be an ordered set in the order topology; let $Y$ be a subset of $X$ that is convex in $X$. Then the order topology on $Y$ is the
same as the topology $Y$ inherits as a subspace of $X$.
\end{theorem}

\subsection{Closed Sets and Limit Points}

\begin{definition}[Closed Set]
A subset $A$ of a topological space $X$ is said to be \textbf{closed} if the set $X \text{ -- } A$ is open. 
\end{definition}

\begin{eg}
The subsets $[a, b]$ and $[a, \infty)$ of $\R$ are closed, while $[a, b)$ of $\R$ is neither open nor closed.
\end{eg}

\begin{eg}
In the discrete topology on the set $X$, every set is both open and closed.
\end{eg}

\begin{definition}[Interior and Closure]
Given a subset $A$ of a topological space $X$, the \textbf{interior} of $A$, denoted $\mathrm{Int} \, A$, is the union of all open sets contained in $A$. \\

The \textbf{closure} of $A$, denoted $\mathrm{Cl} \, A$ or $\overline{A}$, is the intersection of all closed sets containing $A$. \\

It follows by definition that $\mathrm{Int} \, A \subset A \subset \overline{A}$.  
\end{definition}

\begin{theorem}
Let $Y$ be a subspace of $X$, let $A$ be a subset of $Y$, and let $\overline{A}$ denote the closure of $A$ in $X$. Then the closure of $A$ in $Y$ is $\overline{A} \cap Y$.  
\end{theorem}

\begin{theorem}
Let $A$ be a subset of the topological space $X$. 
\begin{enumerate}[a)]
    \item Then $x \in \overline{A}$ if and only if every open set $U$ containing $x$ (i.e. neighborhood of $x$) intersects $A$.
    \item Supposing that the topology of $X$ is given by a basis, then $x \in \overline{A}$ if and only if every
    basis element $B$ containing $x$ intersects $A$.
\end{enumerate}    
\end{theorem}

\begin{eg}
If $A = (0, 1]$, then $\overline{A} = [0, 1]$. If $B = \{ \frac{1}{n} \mid n \in \Z_+ \} $, then $\overline{B} = \{0\} \cup B$.
\end{eg}

\begin{eg}
Consider $Y = (0, 1]$, a subspace of $\R$, and the set $A = (0, \frac{1}{2})$, a subset of $Y$. \\

$\overline{A} = [0, \frac{1}{2}]$ in $\R$ while the closure of $A$ in $Y$ is $\overline{A} \cap Y  = (0, \frac{1}{2}]$. 
\end{eg}

\begin{definition}[Limit Points]
Let $A$ be a subset of the topological space $X$ and let $x$ be a point of $X$. \\

$x$ is a \textbf{limit point} (or ``cluster point'', or ``point of accumulation'') of $A$ if every neighborhood of $x$ intersects $A$
in some point \textit{other than $x$ itself}. \\

Equivalently, $x$ is a limit point of $A$ if it belongs to the closure of $A \text{ -- } \{ x \}$.
\end{definition}

\begin{theorem}
Let $A$ be a subset of the topological space $X$ and let $A^{\prime}$ be the set of all limit points of $A$. Then
\[
    \overline{A} = A \cup A^{\prime}. 
\]
\end{theorem}

\begin{definition}[Hausdorff Spaces]
A topological space $X$ is a \textbf{Hausdorff space} if for each pair $x_1$, $x_2$ of distinct points of $X$, there exist
neighborhoods $U_1$ and $U_2$ of $x_1$ and $x_2$, respectively, that are disjoint.
\end{definition}

\begin{theorem}
If $X$ is a Hausdorff space, then a sequence of points of $X$ converges to at most one point of $X$, known as the \textbf{limit} of that sequence of points.
\end{theorem}

\subsection{Continuous Functions}

\begin{definition}[Continuous Functions]
Let $X$ and $Y$ be topological spaces. A function $f \colon X \rightarrow Y$ is said to be \textbf{continuous} if for each open subset $V$ of $Y$,
the set $f^{-1}(V)$ is an open subset of $X$.
\end{definition}

\begin{remark}
To prove continuity of $f$, it suffices to show that the inverse image of every basis element is open; 
it can even suffice to show that the inverse of each subbasis is open.
\end{remark}

\begin{theorem}[Continuity Equivalences]
Let $X$ and $Y$ be topological spaces; let $f \colon X \rightarrow Y$. Then the following are equivalent:
\begin{enumerate}
    \item $f$ is continuous.
    \item For every subset $A$ of $X$, one has $f(\overline{A}) \subset \overline{f(A)}$.
    \item For every closed set $B$ of $Y$, the set $f^{-1}(B)$ is closed in $X$.
    \item For each $x \in X$ and each neighborhood $V$ of $f(x)$, there is a neighborhood $U$ of $x$ such that $f(U) \subset V$. 
\end{enumerate}

If condition 4 holds for the point $x$ of $X$, we say $f$ is \textbf{continuous at the point $x$}.
\end{theorem}

\begin{definition}[Homeomorphism]
Let $X$ and $Y$ be topological spaces; let $f \colon X \rightarrow Y$ be a bijection. If both the function $f$ and the inverse function
\[
    f^{-1} \colon Y \rightarrow X
\]

are continuous, then $f$ is a \textbf{homeomorphism}. \\

Equivalently, $f$ is a homeomorphism if there is a bijective correspondence $f \colon X \rightarrow Y$ such that $f(U)$ is open if and only if $U$ is open.
\end{definition}

\begin{remark}
Just as isomorphisms preserve algebraic properties of groups/rings, homeomorphisms preserve topological structures.
\end{remark}

\begin{definition}[Imbedding]
Let $f \colon X \rightarrow Y$ be an injective continuous map between topological spaces $X$ and $Y$. Let $Z$ be the image set $f(X)$ considered as a subspace of $Y$;
then the function $f^{\prime} \colon X \rightarrow Z$ obtained by restricting the range is bijective. \\

If $f^{\prime}$ is a homemorphism of $X$ with $Z$, then the map $f$ is a \textbf{topological imbedding} (or \textbf{imbedding}) of $X$ in $Y$.
\end{definition}

\begin{eg}
The function $F: (-1, 1) \rightarrow \R$ defined by $F(x) = \frac{x}{1-x^2}$ is a homemomorphism.
\end{eg}

\begin{remark}
A bijective function $f \colon X \rightarrow Y$ can be continuous without being a homomorphism: consider $F: [0, 1) \rightarrow S^1$ by $f(t) = (\cos 2\pi t, \sin 2\pi  t)$;
$f^{-1}$ is not continuous, as $f^{-1}([0, \frac{1}{4}))$ does not lie in an open set $V$ of $\R^2$ such that $V \cap S^1 \subset f([0, \frac{1}{4}))$.
\end{remark}

\begin{theorem}[Rules for Constructing Continuous Functions]
Let $X$, $Y$, and $Z$ be topological spaces. 
\begin{enumerate}[a)]
    \item (Constant function) If $f \colon X \rightarrow Y$ maps all of $X$ into the single point $y_0$ of $Y$, then $f$ is continuous.
    \item (Inclusion) If $A$ is a subspace of $X$, the inclusion function $j \colon A \rightarrow X$ is continuous. 
    \item (Composites) If $f \colon X \rightarrow Y$ and $g \colon Y \rightarrow Z$ are continuous, then so is $g \circ f : X \rightarrow Z$.
    \item (Restricting the domain) If $f \colon X \rightarrow Y$ is continuous, and if $A$ is a subspace of $X$, then the restricted function $f \mid A \colon A \rightarrow Y$ is continuous.
    \item (Restricting or expanding the range) Let $f \colon X \rightarrow Y$ be continuous. If $Z$ is a subspace of $Y$ containing the image set $f(X)$, then the function
    $g \colon X \rightarrow Z$ obtained by restricting the range of $f$ is continuous. If $Z$ is a space having $Y$ as a subspace, then the function $h \colon X \rightarrow Z$ obtained by
    by expanding the range of $f$ is continuous.
    \item (Local formulation of continuinity) The map $f \colon X \rightarrow Y$ is continuous if $X$ can be written as the union of open sets $U_\alpha$ such that $f \mid U_\alpha$ is continuous for each $\alpha$.
\end{enumerate}
\end{theorem}

\begin{theorem}[The Pasting Lemma]
Let $X = A \cup B$ where $A$ and $B$ are closed in $X$, and let $f \colon A \rightarrow Y$ and $g \colon B \rightarrow Y$ be continuous. 
If $f(x) = g(x)$ for every $x \in A \cap B$, then $f$ and $g$ combine to give a continuous function $h\colon X \rightarrow Y$, defined by
$h(x) = f(x)$ if $x \in A$ and $h(x) = g(x)$ if $x \in B$.
\end{theorem}

\begin{theorem}[Maps into Products]
Let $f \colon A \rightarrow X \times Y$ be given by the equation $f(a) = (f_1(a), f_2(a))$. Then $f$ is continuous if and only if the functions
$f_1 \colon A \rightarrow X$ and $f_2\colon A \rightarrow Y$ are continuous.
\end{theorem}

\subsection{The Product Topology}

\begin{definition}[Box Topology]
Let $\{ X_\alpha \}_{\alpha \in J}$ be an indexed family of topological spaces. Take as a basis for a topology on the product space $\prod _{\alpha \in J} X_\alpha$ the collection of all sets of the form
\[
    \prod_{\alpha \in J} U_\alpha,
\]
where $U_\alpha$ is open in $X_\alpha$, for each $\alpha \in J$. The topology generated by this basis is the \textbf{box topology}.
\end{definition}

\begin{definition}[Projection Mapping]
Let $\pi_{\beta} \colon  \prod _{\alpha \in J} X_\alpha \rightarrow X_\beta$ be the function assigning to each element of the product space its $\beta$th coordinate, 
\[
    \pi _\beta((x_\alpha)_{\alpha \in J}) = x_\beta.
\]
$\pi_\beta$ is the \textbf{projection mapping} associated with index $\beta$.
\end{definition}

\begin{definition}[Product Topology]
Let $\mathscr{S}_\beta$ denote the collection
\[
    \mathscr{S}_\beta = \{ \pi _\beta^{-1} (U_\beta) \mid U_\beta \text{ open in $X_\beta$}\} 
\]

and let $S$ denote the union of these collections, $S = \bigcup_{\beta \in J} \mathscr{S}_\beta$. \\

The topology generated by the subbasis $\mathscr{S}$ is the \textbf{product topology} and the topology $\prod_{\alpha \in J} X_\alpha$ is a \textbf{product space}. 
\end{definition}

\begin{theorem}[Comparison of Box and Product Topologies]
The box topology on $\prod X_\alpha$ has as basis all sets of the form $\prod  U_\alpha$ where $U_\alpha$ is open in $X_\alpha$ for each $\alpha$. \\

The product topology on $\prod X_\alpha$ has as basis all sets of the form $\prod  U_\alpha$ where $U_\alpha$ is open in $X_\alpha$ for each $\alpha$ and $U_\alpha$ equals $X_\alpha$ except for finitely many values of $\alpha$.
\end{theorem}

\begin{remark}
For finite products, these topologies are identical. Furthermore, the box topology is in general finer than the product topology.
\end{remark}

\subsection{The Metric Topology}

\begin{definition}[Metric]
A \textbf{metric} on a set $X$ is a function $d \colon X \times X \rightarrow R$ having the following properties:
\begin{enumerate}
    \item $d(x, y) \geq 0$ for all $x, y \in X$; equality holds iff $x =y$.
    \item $d(x, y) = d(y, x)$ for all $x,y \in X$.
    \item (Triangle Inequality) $d(x, y) + d(y, z) \geq d(x, z)$ for all $x, y, z \in X$.
\end{enumerate}

\end{definition}

There is also the idea of a \textbf{$\varepsilon$-ball centered at $x$}: $B_d(x, \varepsilon) = \{ y \mid d(x, y) < \varepsilon \}$
which we have grown to love from our Analysis classes. \\

\begin{definition}[Metric Topology]
If $d$ is a metric on the set $X$, then the collection of all $\varepsilon$-balls $B_d(x, \varepsilon)$, for $x \in X$ and $\varepsilon > 0$, is a basis for a topology on $X$,
called the \textbf{metric topology} induced by $d$.
\end{definition}

\begin{remark}
Equivalently, we can think of the metric topology in terms of its open sets:
$U$ is open in the metric topology induced by $d$ if and only if for each $y \in U$, there is a $\delta > 0$ such that $B_d(y, \delta) \subset U$.
\end{remark}

\begin{definition}
If $X$ is a topological space, $X$ is \textbf{metrizable} if there exists a metric $d$ on a set $X$ that induces the topology of $X$. \\

A \textbf{metric space} is a metrizable space $X$ together with a specific metric $d$ that gives the topology on $X$.
\end{definition}

\begin{theorem}[Standard Bounded Metric]
Let $X$ be a metric space with metric $d$. Define $\overline{d} \colon X \times X \rightarrow \R$ by the equation
\[
    \overline{d} (x, y) = \min \{ d(x, y), 1 \}. 
\]

Then $\overline{d}$ is known as the \textbf{standard bounded metric} corresponding to $d$, and it is a metric that
induces the same topology as $d$.
\end{theorem}

\begin{lemma}
Let $d$ and $d^{\prime}$ be two metrics on the set $X$; let $\mathscr{T}$ and $\mathscr{T}^{\prime}$ be the topologies they induce.
Then $\mathscr{T}^{\prime}$ is finer than $\mathscr{T}$ if and only if for each $x$ in $X$ and each $\varepsilon > 0$, there exists a $\delta > 0$ such that
\[
    B_d^{\prime}(x, \delta) \subset B_d(x, \varepsilon).
\]
\end{lemma}

\begin{definition}[Uniform Metric and Topology]
Given an index set $J$ and points $\mathbf{x} = (x_\alpha)_{\alpha \in J}$ and $\mathbf{y} = (y_\alpha)_{\alpha \in J}$ of $\R^J$, the metric $\overline{\rho}$ on $\R^J$ defined by
\[
    \overline{\rho}(\mathbf{x} , \mathbf{y}) = \sup \{ \overline{d}(x_\alpha, y_\alpha) \mid \alpha \in J \}  
\]
where $\overline{d}$ is the standard bounded metric on $\R$ is known as the \textbf{uniform metric} on $\R^J$, and
the topology it induces is called the \textbf{uniform topology}.
\end{definition}


