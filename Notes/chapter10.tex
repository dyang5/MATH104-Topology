\newpage

\section{Separation Theorems in the Plane}

\setcounter{subsection}{60}

\subsection{The Jordan Separation Theorem}

\begin{lemma}[Nulhomotopy Lemma]
Let $a$ and $b$ be points of $S^2$. Let $A$ be a compact space, and let
\[
    f\colon A \rightarrow S^2 \text{ -- } a \text{ -- } b
\]
be a continuous map. If $a$ and $b$ lie in the same component of $S^2 \text{ -- } f(A)$, then $f$ is nulhomotopic.
\end{lemma}

\begin{definition}
If $X$ is a connected space and $A \subset X$, then $A$ \textbf{separates} $X$ if $X \text{ -- } A$ is not connected;
if  $X \text{ -- } A$ has $n$ components, then $A$ \textbf{separates $X$ into $n$ components}.
\end{definition}

\begin{definition}
An \textbf{arc} $A$ is a space homeomorphic to the unit interval $[0, 1]$. The \textbf{endpoints} of $A$ are the two points $p$ and $q$ of $A$
such that $A \text{ -- } p$ and $A \text{ -- } q$ are connected; the other points of $A$ are the \textbf{interior points} of $A$. \\

A \textbf{simple closed curve} is a space homeomorphic to the unit circle $S^1$.
\end{definition}

\begin{theorem}[The Jordan Separation Theorem]
Let $C$ be a simple closed curve in $S^2$. Then $C$ separates $S^2$.
\end{theorem}

\begin{theorem}[A General Separation Theorem]
Let $A_1$ and $A_2$ be closed connected subsets of $S^2$ whose intersection consists of precisely two points $a$ and $b$. Then the set
$C = A_1 \cup A_2$ separates $S^2$.
\end{theorem}


\subsection{Invariance of Domain}

\begin{lemma}[Homotopy Extension Lemma]
Let $X$ be a space such that $X \times I$ is normal. Let $A$ be a closed subspace of $X$, and let $f\colon A \rightarrow Y$ be a continuous map,
where $Y$ is an open subspace of $\R^n$. If $f$ is nulhomotopic, then $f$ may be extended to a continuous map
$g \colon X\rightarrow Y$ that is also nulhomotopic.
\end{lemma}

\begin{lemma}[Borsuk Lemma]
Let $a$ and $b$ be points of $S^2$. Let $A$ be a compact space, and let $f\colon A \rightarrow S^2 \text{ -- } a \text{ -- } b$ be a continuous injective map.
If $f$ is nulhomotopic, then $a$ and $b$ lie in the same component of $S^2 \text{ -- } f(A)$. 
\end{lemma}

\begin{theorem}[Invariance of Domain]
If $U$ is an open subset of $\R^2$ and $f\colon U \rightarrow \R^2$ is continuous and injective, then $f(U)$ is open in $\R^2$ and the inverse function
$f^{-1} \colon f(U) \rightarrow U$ is continuous.
\end{theorem}

\subsection{The Jordan Curve Theorem}

\begin{theorem}
Let $X$ be the union of two open sets $U$ and $V$, such that $U \cap V$ can be written as the union of two disjoint open sets $A$ and $B$. 
Assume that there is a path $\alpha$ in $U$ from a point $a$ of $A$ to a point $b$ of $B$, and that there is a path $\beta$ in $V$
from $b$ to $a$. Let $f$ be the loop $f = \alpha * \beta$.
\begin{enumerate}[a)]
    \item The path-homotopy class $[f]$ generates an infinite cyclic subgroup of $\pi_1(X, a)$.
    \item If $\pi_1(X, a)$ is itself infinite cyclic, it is generated by $[f]$.
    \item Assume that there is a path $\gamma$ in $U$ from $a$ to the point $a^{\prime}$ of $A$, and that there is a path $\delta$ in $V$ from $a^{\prime}$ to $a$. 
    Let $g$ be the loop $g = \gamma * \delta$. Then the subgroups of $\pi_1(X, a)$ generated by $[f]$ and $[g]$ intersect in the identity element alone.
\end{enumerate}
\end{theorem}

\begin{theorem}[Non-Separation Theorem]
Let $D$ be an arc in $S^2$. Then $D$ does not separate $S^2$.
\end{theorem}


\begin{theorem}[General Non-Separation Theorem]
Let $D_1$ and $D_2$ be closed subsets of $S^2$ such that $S^2 \text{ -- } D_1 \cap D_2$ is simply connected. 
If neither $D_1$ nor $D_2$ separates $S^2$, then $D_1 \cup D_2$ does not separate $S^2$.
\end{theorem}


\begin{theorem}[The Jordan Curve Theorem]
Let $C$ be a simple closed curve in $S^2$. Then $C$ separates $S^2$ into precisely two components $W_1$ and $W_2$. Each of the sets
$W_1$ and $W_2$ has $C$ as its boundary; that is, $C = \overline{W}_i - W_i$ for $i = 1, 2$.  
\end{theorem}

\begin{theorem}
Let $C_1$ and $C_2$ be closed connected subsets of $S^2$ whose intersection consists of two points. If neither $C_1$ nor $C_2$ separates $S^2$,
then $C_1 \cup C_2$ separates $S^2$ into precisely two components.
\end{theorem}

\begin{theorem}[Schoenflies Theorem]
If $C$ is a simple closed curve in$ S^2$ and $U$ and $V$ are the components of $S^2 \text{ -- } C$, then $\overline{U}$ and $\overline{V}$
are each homeomorphic to the closed unit ball $B^2$. 
\end{theorem}

\begin{remark}
The separation theorems can be generalized to higher dimensions as follows:
\begin{enumerate}
    \item Any subspace of $C$ of $S^n$ homeomorphic to $S^{n-1}$ separates $S^n$.
    \item No subspace $A$ of $S^n$ homeomorphic to $[0, 1]$ or to some ball $B^m$ separates $S^n$.
    \item Any subspace $C$ of $S^n$ homeomorphic to $S^{n-1}$ separates $S^n$ into two components, of which $C$ is the common boundary.
\end{enumerate}
\end{remark}



