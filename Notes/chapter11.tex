\newpage

\section{The Seifert-van Kampen Theorem}

\setcounter{subsection}{66}
\subsection{Direct Sums of Abelian Groups}

\begin{definition}
Suppose $G$ is an abelian group, and $\{ G_\alpha \}_{\alpha \in J}$ is an indexed family of subgroups of $G$. \\

The groups $G_\alpha$ \textbf{generate} $G$ if every element $x$ of $G$ can be written as a finite sum of elements of the groups $G_\alpha$. \\

$G$ is a \textbf{direct sum} of the groups $G_\alpha$ if for each $x \in G$, there is only one $J$-tuple $(x_\alpha)_{\alpha \in J}$ with $x_\alpha = 0$ for all 
but finitely many $\alpha$ such that $x = \sum x_\alpha$. $G$ is written as
\[
    G = \bigoplus_{\alpha \in J} G_\alpha.
\]
\end{definition}

\begin{eg}
The cartesian product $\R^\omega$ is an abelian group under the operation of coordinate-wise addition. The set $G_n$ consisting 
of those tuples $(x_i)$ such that $x_i = 0$ for $i \neq n$ is a subgroup isomorphic to $\R$.
\end{eg}

The following lemma characterize the extension condition for direct sums:
\begin{lemma}
Let $G$ be an abelian group; let $\{ G_\alpha \}$ be a family of subgroups of $G$. If $G$ is the direct sum of the groups $G_\alpha$, then
$G$ satisfies the following condition: \\

\textit{Given any abelian group $H$ and any family of homomorphisms $h_\alpha \colon G_\alpha \rightarrow H$, there exists a homomorphism $h\colon G \rightarrow H$ whose restriction
to $G_\alpha$ equals $h_\alpha$, for each $\alpha$.}
\end{lemma}

\begin{definition}[External Direct Sum]
Let $\{ G_\alpha \}_{\alpha \in J}$ be an indexed family of abelian groups. Suppose that $G$ is an abelian group and that $i_\alpha \colon G_\alpha \rightarrow G$ is a family of monomorphisms,
such that $G$ is the direct sum of the groups $i_\alpha(G_\alpha)$. Then we say that $G$ is the \textbf{external direct sum} of the groups $G_\alpha$, relative to the monomorphisms $i_\alpha$. 
\end{definition}

\begin{lemma*}[67.5: Extension Condition for External Direct Sums]
    Let $\{G_\alpha\}_{\alpha \in J}$ be an indexed family of abelian groups; let $G$ be an abelian group; let $i_\alpha \colon G_\alpha \rightarrow G$ be a family of homomorphism. If each $i_\alpha$ is a monomorphism and $G$ is a direct sum of the groups $i_\alpha(G_\alpha)$, then $G$ satisfies the following extension condition: \\
    
    \begin{addmargin}{1em}
    Given any abelian group $H$ and any family of homomorphism $h_\alpha \colon G_\alpha \rightarrow H$, there exists a homomorphism $h \colon G \rightarrow H$ such that $h \circ i_\alpha = h_\alpha$ for each $\alpha$. \\
    \end{addmargin}
\end{lemma*}

Theorem 67.6 gives the uniqueness of the direct sums.

\begin{definition}[Free Abelian Groups]
Let $G$ be an abelian group and let $\{ a_\alpha \}$ be an indexed family of elements of $G$; let $G_\alpha$ be the subgroup of $G$ generated by $a_\alpha$. If the groups $G_\alpha$ generate $G$, we 
also say that the elements $a_\alpha$ generate $G$. \\

If each group $G_\alpha$ is infinite cyclic, and if $G$ is the direct sum of the groups $G_\alpha$, then $G$ is a \textbf{free abelian group} having the elements $\{ a_@a
\}$ as a \textbf{basis}.
\end{definition}


\subsection{Free Products of Groups}

\begin{definition}[Free Product]
Let $G$ be a group, let $\{ G_\alpha \}_{\alpha \in J}$ be a family of subgroups of $G$ that generates $G$. Suppose that
$G_\alpha \cap G_\beta$ consists of the identity element alone whenever $\alpha \neq \beta$. We say that $G$ is the \textbf{free product} of the groups $G_\alpha$ if for each $x \in G$,
there is only one reduced word in the groups $G_\alpha$ that represents $x$. In this case,
\[
    G = \prod_{\alpha \in J}^* G_\alpha,
\]
or in the finite case, $G = G_1 * \dots * G_n$. 
\end{definition}

\begin{remark}
If $G$ is the free product of groups $G_\alpha$ that generate $G$, then the representation of $1$ by the empty word must be unique.
\end{remark}

\begin{lemma}[Extension Condition of Free Products]
Let $G$ be a group; let $\{ G_\alpha \}$ be a family of subgroups of $G$. If $G$ is the free product of the groups $G_\alpha$, then $G$ satisfies the following condition: \\ 
    
\begin{addmargin}{1em}
    Given any group $H$ and any family of homomorphism $h_\alpha \colon G_\alpha \rightarrow H$, there exists a homomorphism $h \colon G \rightarrow H$ whose restriction to $G_\alpha$ equals $h_\alpha$, for each $\alpha$. \\
\end{addmargin}

Furthermore, $h$ is unique.
\end{lemma}

\begin{definition}[External Free Product]
    Let $\{ G_\alpha \}_{\alpha \in J}$ be an indexed family of abelian groups. Suppose that $G$ is an abelian group and that $i_\alpha \colon G_\alpha \rightarrow G$ is a family of monomorphisms,
    such that $G$ is the free product of the groups $i_\alpha(G_\alpha)$. Then we say that $G$ is the \textbf{external free product} of the groups $G_\alpha$, relative to the monomorphisms $i_\alpha$. 
\end{definition}

There are, similar to the previous conditions, equivalent extension conditions for external free products (68.5), and a theorem about the Uniqueness of the Free Product (68.4).

\begin{definition}
if $x$ and $y$ are elements of a group $G$, $y$ is \textbf{conjugate} to $x$ if $y = cxc^{-1}$ for some $c \in G$. \\

A normal subgroup of $G$ is one that contains all conjugates of its elements. \\

If $S$ is a subset of $G$, the intersection of all normal subgroups of $G$ that contain $S$ is itself a subgroup, known as the \textbf{least normal subgroup} of $G$ containing $S$.
\end{definition}

\subsection{Free Groups}

\begin{definition}[Generate]
Let $G$ be a group; let $(a_\alpha)$ be a family of elements of $G$, for $\alpha \in J$. We say the elements 
$(a_\alpha)$ \textbf{generate} $G$ if every element of $G$ can be written as a product of powers of the elements $a_\alpha$. 
If the family $(a_\alpha)$ is finite, we say $G$ is \textbf{finitely generated.}
\end{definition}

\begin{definition}[Free Group]
Let $(a_\alpha)$ be a family of elements of a group $G$. Suppose each $a_\alpha$ generates an infinite cyclic subgroup $G_\alpha$ of $G$. 
If $G$ is the free product of the groups $\{ G_\alpha \}$ then $G$ is said to be a \textbf{free group}, 
and the family $\{ a_\alpha \}$ is said to be a \textbf{family of free generators for $G$}.
\end{definition}

\begin{lemma}
    Let $G$ be a group; let $\{ a_\alpha \}_{\alpha \in J}$ be a family of elements of $G$. If $G$ is a free group with a system of free generators $\{ a_\alpha \}$, then $G$ satisfies the following condition:
        
    \begin{addmargin}{1em}
        Given any group $H$ and any family $\{ y_\alpha \}$ of elements of $H$ , 
        there is a homomorphism $h \colon G \rightarrow H$ such that $h(a_\alpha) = y_\alpha$ for each $\alpha$. 
    \end{addmargin}
    
    Furthermore, $h$ is unique. Conversely, if the extension condition holds, then $G$ is a free group with system of free generators $a_\alpha$.
\end{lemma}

\begin{definition}[Free Group on Elements]
Let $\{ a_\alpha \}_{\alpha \in J}$ be an arbitrary indexed family. Let $G_\alpha$ denote the set of all symbols of the form $a_\alpha^n$ for $n \in \Z$. We make $G_\alpha$ into a group by defining
\[
    a_\alpha^n \cdot a_\alpha^m = a_\alpha^{n+m}.
\]
The external free product of the groups $\{ G_\alpha \}$ is the \textbf{free group on the elements $a_\alpha$}. 
\end{definition}

\begin{definition}[Commutator and Commutator Subgroup]
Let $G$ be a group. If $x, y \in G$, we denote $[x, y]$ by
\[
    [x, y] = xyx^{-1}y^{-1}
\]
of $G$, the \textbf{commutator of $x$ and $y$}. The subgroup of $G$ generated by the set of all commutators in $G$ is the \textbf{commutator subgroup} of $G$ and is denoted $[G, G]$. 
\end{definition}

\subsection{The Seifert-van Kampen Theorem}

\begin{theorem}[Seifert-van Kampen Theorem]
Let $X = U \cup V$, where $U$ and $V$ are open in $X$; assume $U$, $V$, and $U \cap V$ are path connected; let $x_0 \in U \cap V$. Let $H$ be a group, and let
\[
    \varphi_1\colon \pi_1(U, x_0) \rightarrow H \text{ and } \varphi_2 \colon \pi_1(V, x_0) \rightarrow  H
\]
be homomorphisms. Let 
\[
    i_1 \colon \pi_1(U \cap V, x_0) \rightarrow \pi_1(U, x_0), \, i_2 \colon \pi_1(U \cap V, x_0) \rightarrow \pi_1(V, x_0), 
\]
\[
    \, j_1 \colon \pi_1(U, x_0) \rightarrow \pi_1(X, x_0), \text{ and } j_2\colon \pi_1(V, x_0) \rightarrow \pi_1(X, x_0)
\]
be homomorphisms, each induced by inclusion.
If $\varphi_1 \circ i_1 = \varphi_2 \circ i_2$, then there is a unique homomorphism $\Phi\colon \pi_1(X, x_0) \rightarrow H$ such that
$\Phi \colon \pi_1(X, x_0) \rightarrow H$ such that $\Phi \circ j_1 = \varphi_1$ and $\Phi \circ j_2 = \varphi_2$. 
\end{theorem}

\begin{theorem}[Seifert-van Kampen, classical version]
Assume the hypotheses of the preceding theorem. Let
$j\colon \pi_1(U, x_0) * \pi_1(V, x_0) \rightarrow \pi_1(X, x_0)$ be the homomorphism of the free product that extends the homomorphisms $j_1$ and $j_2$ induced by inclusion.
Then $j$ is surjective, and its kernal is the least normal subgroup $N$ of the free product that contains all elements represented by words of the form
\[
    (i_1(g)^{-1}, i_2(g)),
\]
for $g \in \pi_1(U \cap V, x_0)$. \\

Said differently, the kernel of $j$ is generated by all elements o the free product of the form $i_1(g)^{-1}i_2(g)$ and their conjugates.
\end{theorem}

\subsection{The Fundamental Group of a Wedge of Circles}

\begin{definition}[Wedge of Circles]
Let $X$ be a Hausdorff space that is a union of the subspaces $S_1, \dots, S_n$, each of which is homeomorphic to the unit circle $S^1$.
Assume that there is a point $p$ of $X$ such that $S_i \cap S_j = \{ p \}$ whenever $i \neq j$. \\

Then $X$ is called the \textbf{wedge of the circles} $S_1, \dots, S_n$.
\end{definition}

\begin{theorem}
Let $X$ be the wedge of the circles $S_1, \dots, S_n$; let $p$ be the common point of these circles. Then $\pi_1(X, p)$ is a free group. If $f_i$ is a loop in $S_i$ 
that represents a generator of $\pi_1(S_i, p)$, then the loops $f_1, \dots, f_n$ represent a system of free generators for $\pi_1(X, p)$.
\end{theorem}

\begin{definition}[Coherence]
Let $x$ be a space that is the union of the subspaces $X_\alpha$ for $\alpha \in J$. The topology of $X$ is said to be \textbf{coherent}
with the subspaces $X_\alpha$ provided a subset $C$ of $X$ is closed in $X$ if $C \cap X_\alpha$ is closed in $X_\alpha$ for each $\alpha$.
\end{definition}

\begin{remark}
An equivalent condition is that a set is open in $X$ if its intersection with each $X_\alpha$ is open in $X_\alpha$. \\

In the finite case, if $X$ is the union of finitely many closed subspaces $X_1, \dots, X_n$, then $X$ is automatically coherent with these subspaces;
if $C \cap X_i$ is closed in $X_i$, it is closed in $X$, and $C$ is the finite union of the sets $C \cap X_i$, which is itself closed.
\end{remark}

\begin{definition}[Wedge of Circles: Generalized]
    Let $X$ be a space that is the union of subspaces $S_\alpha$, for $\alpha \in J$, each of which is homeomorphic to the unit circle.
    Assume there is a point $p$ of $X$ such that $S_\alpha \cap S_\beta = \{ p \}$ whenever $\alpha \neq \beta$. \\
    
    If the topology of $X$ is coherent with the subspaces $S_\alpha$, then $X$ is called the \textbf{wedge of the circles} $S_\alpha$.
\end{definition}

\begin{theorem*}[71.3]
    Let $X$ be the wedge of the circles $S_\alpha$ for $\alpha \in J$; let $p$ be the common point of these circles. Then $\pi_1(X, p)$ is a free group. If $f_\alpha$ is a loop in $S_\alpha$ 
    representing a generator of $\pi_1(S_\alpha, p)$, then the loops $\{ f_\alpha \} $ represent a system of free generators for $\pi_1(X, p)$.
\end{theorem*}


\subsection{Adjoining a Two-Cell}

\begin{theorem}
Let $X$ be a Hausdorff space; let $A$ be a closed path-connected subspace of $X$. Suppose that there is a continuous map $h\colon B^2 \rightarrow X$ that maps $\mathrm{Int} B^2$
bijectively onto $X \text{ -- } A$ and maps $S^1 = \mathrm{Bd} B^2$ into $A$. Let $p \in S^1$ and let $a = h(p)$; let $k\colon (S^1, p) \rightarrow (A, a)$ be the map obtained by restricting $h$.
Then the homomorphism
\[
    i_{\ast} \colon \pi_1(A, a) \rightarrow \pi_1(X, a)
\]  
induced by inclusion is surjective, and its kernel is the least normal subgroup of $\pi_1(A, a)$ containing the image of $k_{\ast} \colon \pi_1(S^1, p) \rightarrow \pi_1(A, a)$.
\end{theorem}

\begin{remark}
$X$ in this theorem is thought of as having been obtained by adjoining a \textbf{2-cell} to $A$, where the 2-cell refers to any space $B$ homeomorphic to $B^2$.
\end{remark}

\subsection{The Fundamental Groups of the Torus and the Dunce Cap}

\begin{theorem}
The fundamental group of the torus has a presentation consisting of two generators $\alpha, \beta$ and a single relation $\alpha\beta\alpha^{-1}\beta^{-1}$.
\end{theorem}

\begin{corollary}
The fundamental group of the torus is a free abelian group of rank $2$.
\end{corollary}

\begin{definition}[N-fold Dunce Cap]
Let $n$ be a positive integer greater than $1$. Let $r\colon S^1 \rightarrow S^1$ be rotation through the angle $\frac{2\pi }{n}$. \\

Form a quotient space $X$ from the unit ball $B^2$ by identifying the point $x$ of $S^1$ with the points $r(x), \dots, r^{n-1}(x)$. Then $X$ is a compact Hausdorff space, known as the
\textbf{n-fold dunce cap}.
\end{definition}

\begin{theorem}
The fundamental group of the $n$-fold dunce cap is a cyclic group of order $n$.
\end{theorem}

