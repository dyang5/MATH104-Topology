\section{Chapter 1: Set Theory and Logic}

\setcounter{subsection}{1}
\subsection{Functions}

\begin{definition}[Injective, Surjective, Bijection]
A function $f \colon A \rightarrow B$ is said to be \textbf{injective} (or \textbf{one-to-one}) if for each pair of distinct points of $A$, their images under $f$ are distinct. \\

It is said to be \textbf{surjective} if every element of $B$ is the image of some element of $A$ under $f$. \\

If $f$ is both \textbf{injective} and \textbf{surjective}, it said to be \textbf{bijective}.
\end{definition}

\subsection{Relations}

\begin{definition}[Relation]
A \textbf{relation} on a set $A$ is a subset $C$ of the Cartesian product $A \times A$. 
\end{definition}

\begin{definition}[Equivalence Relation]
An \textbf{equivalence relation} $\sim$ on a set $A$ is a relation $C$ on $A$ having the following three properties:
\begin{enumerate}
    \item (Reflexivity) $x \sim x$ for every $x$ in $A$.
    \item (Symmetry) If $x \sim y$, then $y \sim x$.
    \item (Transitivity) If $x \sim y$ and $y \sim z$, then $x \sim z$.
\end{enumerate}    
\end{definition}

\begin{definition}[Order Relation]\label{simple_order_relation}
    A relation $C$ on a set $A$ is an \textbf{order relation} (also simple order, or linear order) if it has the following properties: 
    \begin{enumerate}
        \item (Comparability) For every $x$ and $y$ in $A$ for which $x \neq y$, either $xCy$ or $yCx$.
        \item (Nonreflexivity) For no $x$ in $A$ does the relation $xCx$ hold.
        \item (Transitivity) If $xCy$ and $yCz$, then $xCz$.
    \end{enumerate}    
    \textit{Note: the relation $C$ is often replaced as $<$, just as how it is synonyous with $\sim$ in the case of an equivalence relation.}
\end{definition}

\begin{remark}
It follows that $xCy$ and $yCx$ cannot both be true. If so, then transitivity implies $xCx$, contradicting nonreflexivity.    
\end{remark}

\begin{eg}
Suppose that $A$ and $B$ are two sets with order relations $<_A$ and $<_B$ respectively. The order relation $<$ on $A \times B$ defined by
\[
    a_1 \times b_1 < a_2 \times b_2
\]
if $a_1 <_A a_2$ or if $a_1 = a_2$ and $b_1 <_B b_2$ is known as the \textbf{dictionary order relation} on $A \times B$.
\end{eg}

\begin{definition}[Immediate Predecessors and Successors]
If $X$ is a set and $<$ is an order relation on $X$, and if $a < b$, the \textbf{open interval} $(a, b)$ on $X$ is the set
\[
    (a, b) = \{ x \mid a < x < b \}. 
\] 
If this set is empty, $a$ is the \textbf{immediate predecessor} of $b$ and $b$ is the \textbf{immediate successor} of $a$.
\end{definition}

\begin{definition}[Order Type]
Suppose that $A$ and $b$ are two sets with order relations $<_A$ and $<_B$, respectively. $A$ and $B$ have the same \textbf{order type} 
if there is a bijective correspondence between them that preserves order. \\

That is, if there exists a bijective function $f \colon A \rightarrow B$ such that
\[
    a_1 <_A a_2 \Rightarrow f(a_1) <_B f(a_2).
\]
\end{definition}

\begin{eg}
The interval $(-1, 1)$ of real numbers has the same order type as $\R$. The function $f \colon (-1, 1) \rightarrow B$ such that
\[
    f(x) = \frac{x}{1-x^2}
\]
is an order-preserving bijective correspondence.
\end{eg}

\begin{definition}[Supremum and Infimum]
Let $A$ be an ordered set. The subset $A_0$ of $A$ is \textbf{bounded above} if there is an element $b$ of $A$ such that $x \leq b$ for every $x \in A_0$: 
$b$ is an \textbf{upper bound} for $A_0$. 
If the set of all upper bounds for $A_0$ has a smallest element, that elements is the \textbf{supremum} of $A_0$ (also the least upper bound). \\

The subset $A_0$ of $A$ is \textbf{bounded below} if there is an element $b$ of $A$ such that $b \leq x$ for every $x \in A_0$: 
$b$ is a \textbf{lower bound} for $A_0$. 
If the set of all lower bounds for $A_0$ has a largest element, that elements is the \textbf{infimum} of $A_0$ (also the greatest lower bound). \\
\end{definition}

\begin{definition}[Least Upper Bound and Greatest Lower Bound Properties]
An ordered set $A$ is said to have the \textbf{least upper bound property} if every nonempty subset $A_0$ of $A$ that is bounded above has a least upper bound. \\

An ordered set $A$ is said to have the \textbf{greatest lower bound property} if every nonempty subset $A_0$ of $A$ that is bounded below has a greatest lower bound.
\end{definition}

\subsection{The Integers and the Real Numbers}
\begin{theorem}[Well-Ordering Principle]
Every nonempty subset of $\Z_+$ has a smallest element.
\end{theorem}

\subsection{Cartesian Products}
This section contains definitions and examples of indexing functions (e.g. $\{ 1, \dots, n \}$, $\Z_+$), tuples, sequences, and Cartesian products.

\begin{definition}[$\omega$-tuple / Sequence]
Given a set $X$, a $\omega$-tuple of elements of $X$ is a function 
\[
    \mathbf{x} \colon \Z_+ \rightarrow X
\]
also known as a \textbf{sequence} (or infinite sequence) of elements of $X$.
\end{definition}

\subsection{Finite Sets}
This section contains basic definitions of finite sets, including cardinality and proof of a number of set axioms.

\subsection{Countable and Uncountable Sets}
\begin{definition}[Countably Infinite]
A set $A$ is infinite if it is not finite. It is \textbf{countably infinite} of there is a bijective correspondence
\[
    f \colon A \rightarrow \Z_+.
\] 
\end{definition}

\begin{eg}
The set $\Z$ of all integers is countably infinite. Similarly, $\Z \times \Z$ is countably infinite. 
\end{eg}

\begin{proof}[Proof (Countability of $\Z \times \Z$)]
\textit{Proof 1}. Consider the bijections $f \colon \Z_+ \times \Z_+ \rightarrow A$ and $g \colon A \rightarrow \Z_+$ defined as follows:
\[
    f(x, y) = (x + y - 1, y) \text{ and } g(x, y) = \frac{1}{2}(x-1)x + y.    
\]
    
The composition $g \circ f$ is also a bijection from $\Z \times \Z$ to $\Z$, so $\Z \times \Z$ is countably infinite. \\

\textit{Proof 2}. Consider $f(n, m) = 2^n 3^m$, an injective map from $\Z \times \Z$ to $\Z$.
\end{proof}

\begin{definition}[Countable and Uncountable Sets]
A set is \textbf{countable} if it is either finite or countably infinite. A set that is not countable is \textbf{uncountable}.
\end{definition}

\begin{eg}
$\{0, 1\}^\omega$, $\mathcal{P} (\Z_+)$, and $\R$ are examples of uncountable sets.
\end{eg}

\begin{theorem}
Let $B$ be a nonempty set. Then the following are equivalent:
\begin{enumerate}
    \item $B$ is countable.
    \item There is a surjective function $f\colon \Z_+ \rightarrow B$.
    \item There is an injective function $g\colon B \rightarrow \Z_+$.
\end{enumerate}    
\end{theorem}

\begin{theorem}[Countable Union of Countable Sets]
A countable union of countable sets is countable.
\end{theorem}

\subsection{Principle of Recursive Definition}
This section contains recursion axioms and the introduction of the principle of recursion/recursion formula.  

\subsection{Infinite Sets and the Axiom of Choice}

\begin{theorem}
Let $A$ be a set. The following statements about $A$ are equivalent:
\begin{enumerate}
    \item $A$ is infinite.
    \item There exists an injective function $f \colon \Z_+ \rightarrow A$
    \item There exists a bijection of $A$ with a proper subset of itself.
\end{enumerate}
\end{theorem}

\begin{theorem}[Axiom of Choice]
Given a collection $\mathcal{A}$ of disjoint nonempty sets, there exists a set $C$ consisting of exactly
one element from each element of $\mathcal{A}$; that is, a set $C$ such that $C$ is contained in the union of the
elements of $\mathcal{A}$, and for each $A \in \mathcal{A}$, the set $C \cap A$ contains a single element.
\end{theorem}

\subsection{Well-Ordered Sets}

\begin{definition}[Well-Ordered Sets]
A set $A$ with an order relation $<$ is \textbf{well-ordered} if every nonempty subset of $A$ has a smallest element.
\end{definition}

\begin{eg}
The set $\Z_+ \times \Z_+$ is a well-ordered set in the dictionary order. \\

However, the dictionary order does not give a well-ordering of the set $(\Z_+)^\omega$.
\end{eg}

\begin{theorem}[Well-Ordering Theorem; Zermelo, 1904]
If $A$ is a set, there exists an order relation on $A$ that is a well-ordering.
\end{theorem}

\begin{corollary}
There exists an uncountable well-ordered set.    
\end{corollary}

\begin{definition}[Section of a Set]
Let $X$ be a well-ordered set. Given $\alpha \in X$, let $S_\alpha$ denote the set
\[
    S_\alpha = \{ x \mid x \in X \text{ and } x < \alpha\}.
\]

$S_\alpha$ is the \textbf{section} of $X$ by $\alpha$.
\end{definition}

\begin{lemma}[First Uncountable Ordinal]
There exists a well-ordered set $A$ having a largest element $\Omega$ such that the section
$S_\Omega$ of $A$ by $\Omega$ is uncountable but every other section of $A$ is countable.   
\end{lemma}

\begin{theorem}
If $A$ has a countable subset of $S_\Omega$, then $A$ has an upper bound in $S_\Omega$.     
\end{theorem}

\subsection{The Maximum Principle}

\begin{definition}[Partial Order]
Given a set $A$, a relation $\prec $ on $A$ is a \textbf{strict partial order} on $A$ if it has the following properties:
\begin{enumerate}
    \item (Nonreflexivity) The relation $a \prec a$ never holds.
    \item (Transitivity) If $a \prec b$ and $b \prec c$, then $a \prec c$.
\end{enumerate}

If the relation $\prec$ is instead $\preceq$, where $a \preceq b$ implies $a = b$ or $a \prec b$, $\preceq$ is a \textbf{partial order} on $A$. 
\end{definition}


\begin{remark}
    These are the second and third properties of a simple order, defined in \cref{simple_order_relation}. 
    Consequently, a strict partial order behaves like a simple order except that it need not be true that for
    every pair of distinct $x$ and $y$ in the set, either $x \prec y$ or $y \prec x$.
\end{remark}

\begin{theorem}[The Maximum Principle]
Let $A$ be a set and let $\prec$ be a strict partial order on $A$. Then there exists a maximal simply ordered subset $B$ of $A$.
\end{theorem}

\begin{eg}
If $\mathscr{A}$ is the collection of all circular regions in the plane under the ``proper subset of'' relation, a maximal simply
ordered subcollection of $\mathscr{A}$ consists of all circular regions with centers at the origin.
\end{eg}

\begin{definition}
Let $A$ be a set and let $\prec$ be a strict partial order on $A$. If $B$ is a subset of $A$, an 
\textbf{upper bound} on $B$ is an element $c$ of $A$ such that for every $b$ in $B$, either $b = c$ or $b \prec c$. \\

A \textbf{maximal element} of $A$ is an element $m$ on $A$ such that for no element $a$ of $A$ does the relation $m \prec a$ hold. 
\end{definition}

\begin{theorem}[Zorn's Lemma]
Let $A$ be a set that is strictly partially ordered. If every simply ordered subset of $A$ has an upper bound in $A$, then $A$ has a maximal element.
\end{theorem}