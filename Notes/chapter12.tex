\newpage

\section{Classification of Surfaces}

\setcounter{subsection}{73}
\subsection{Fundamental Groups of Surfaces}

\begin{definition}[Polygonal Region, Labelling, Labelling Schemes]
A \textbf{polygonal region} can be thought of as the space that is the interaction of half-spaces of a set of line segments joining adjacent points on a circle. \\
    
A \textbf{labelling} of the edges is a map from the set of edges to a set of labels of $P$. For edges that have the same label, form an equivalence class between a point $x$ on one edge and $h(x)$ on the other, where $h$ is the positive linear map from one edge onto the other. \\
    
The quotient space $X$ arising from this equivalence relation on the polygonal region $P$ is obtained by \textbf{pasting the edges of $P$ together} according to the given orientations and labelling. \\

For each polygonal region, we have a notion of a \textbf{labelling scheme}, which is a sequence of labels corresponding to edges and their orientations.
\end{definition}

\begin{theorem}
Let $X$ be the space obtained from a finite collection of polygonal regions by pasting edges together according to some labelling scheme. Then $X$ is a compact Hausdorff space.
\end{theorem}

\begin{theorem}
    Let $P$ be a polygonal region; let 
    \[ 
        w = (a_{i_1})^{\epsilon_1} \cdots (a_{i_n})^{\epsilon_n}
    \]
    be a labelling scheme for the edges of $P$. Let $X$ be the resulting quotient space; let $\pi \colon P \rightarrow X$ be the quotient map. If $\pi$ maps all the vertices of $P$ to a single point $x_0$ of $X$, and if $a_1, \dots, a_k$ are the distinct labels that appear in the labelling scheme, then $\pi_1(X, x_0)$ is isomorphic to the quotient of the free group on $k$ generators $\alpha_1, \dots, \alpha_k$ by the least normal subgroup containing the element 
    \[
        (\alpha_{i_1})^{\epsilon_1} \cdots (\alpha_{i_n})^{\epsilon_n}.
    \]
\end{theorem}

\begin{definition}[$N$-fold Torus]
    Consider the space obtained from a $4n$-sided polygonal region $P$ by means of the labelling scheme
    \[
        (a_1b_1a_1^{-1}b_1^{-1}) \dots (a_nb_na_n^{-1}b_n^{-1}).
    \]
    This space is called the \textbf{$n$-fold connected sum of tori}, denoted $T \# \dots \# T$.
\end{definition}
    
\begin{theorem}[Fundamental Group of $n$-fold Torus]
Let $X$ denote the $n$-fold torus. Then $\pi_1(X, x_0)$ is isomorphic to the quotient of the free group on the $2n$ generators $\alpha_1, \beta_1, \dots, \alpha_n, \beta_n$ by the least normal subgroup containing the element
\[
    [\alpha_1, \beta_1] \cdots [\alpha_n, \beta_n]. 
\]
\end{theorem}

\begin{definition}[$M$-fold Projective Plane]
Let $m > 1$. Consider the space obtained from a $2m$-sided polygonal region $P$ by means of the labelling scheme
\[
    (a_1a_1) \dots (a_na_n).
\]
This space is called the \textbf{$m$-fold connected sum of projective planes}, denoted $P^2 \dots \# P^2$.
\end{definition}

\begin{theorem}[Fundamental Group of $m$-fold Projective Plane]
Let $X$ denote the $m$-fold projective plane. Then $\pi_1(X, x_0)$ is isomorphic to the quotient of the free group on the $m$ generators $\alpha_1, \dots, \alpha_m$ by the least normal subgroup containing the element
\[
    (\alpha_1)^2\cdots (\alpha_m)^2.
\]
\end{theorem}
    