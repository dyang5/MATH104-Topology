\newpage

\section{Classification of Surfaces}

\setcounter{subsection}{73}
\subsection{Fundamental Groups of Surfaces}

\begin{definition}[Polygonal Region, Labelling, Labelling Schemes]
A \textbf{polygonal region} can be thought of as the space that is the interaction of half-spaces of a set of line segments joining adjacent points on a circle. \\
    
A \textbf{labelling} of the edges is a map from the set of edges to a set of labels of $P$. For edges that have the same label, form an equivalence class between a point $x$ on one edge and $h(x)$ on the other, where $h$ is the positive linear map from one edge onto the other. \\
    
The quotient space $X$ arising from this equivalence relation on the polygonal region $P$ is obtained by \textbf{pasting the edges of $P$ together} according to the given orientations and labelling. \\

For each polygonal region, we have a notion of a \textbf{labelling scheme}, which is a sequence of labels corresponding to edges and their orientations.
\end{definition}

\begin{theorem}
Let $X$ be the space obtained from a finite collection of polygonal regions by pasting edges together according to some labelling scheme. Then $X$ is a compact Hausdorff space.
\end{theorem}

\begin{theorem}
    Let $P$ be a polygonal region; let 
    \[ 
        w = (a_{i_1})^{\epsilon_1} \cdots (a_{i_n})^{\epsilon_n}
    \]
    be a labelling scheme for the edges of $P$. Let $X$ be the resulting quotient space; let $\pi \colon P \rightarrow X$ be the quotient map. If $\pi$ maps all the vertices of $P$ to a single point $x_0$ of $X$, and if $a_1, \dots, a_k$ are the distinct labels that appear in the labelling scheme, then $\pi_1(X, x_0)$ is isomorphic to the quotient of the free group on $k$ generators $\alpha_1, \dots, \alpha_k$ by the least normal subgroup containing the element 
    \[
        (\alpha_{i_1})^{\epsilon_1} \cdots (\alpha_{i_n})^{\epsilon_n}.
    \]
\end{theorem}

\begin{definition}[$N$-fold Torus]
    Consider the space obtained from a $4n$-sided polygonal region $P$ by means of the labelling scheme
    \[
        (a_1b_1a_1^{-1}b_1^{-1}) \dots (a_nb_na_n^{-1}b_n^{-1}).
    \]
    This space is called the \textbf{$n$-fold connected sum of tori}, denoted $T \# \dots \# T$.
\end{definition}
    
\begin{theorem}[Fundamental Group of $n$-fold Torus]
Let $X$ denote the $n$-fold torus. Then $\pi_1(X, x_0)$ is isomorphic to the quotient of the free group on the $2n$ generators $\alpha_1, \beta_1, \dots, \alpha_n, \beta_n$ by the least normal subgroup containing the element
\[
    [\alpha_1, \beta_1] \cdots [\alpha_n, \beta_n]. 
\]
\end{theorem}

\begin{definition}[$M$-fold Projective Plane]
Let $m > 1$. Consider the space obtained from a $2m$-sided polygonal region $P$ by means of the labelling scheme
\[
    (a_1a_1) \dots (a_na_n).
\]
This space is called the \textbf{$m$-fold connected sum of projective planes}, denoted $P^2 \dots \# P^2$.
\end{definition}

\begin{theorem}[Fundamental Group of $m$-fold Projective Plane]
Let $X$ denote the $m$-fold projective plane. Then $\pi_1(X, x_0)$ is isomorphic to the quotient of the free group on the $m$ generators $\alpha_1, \dots, \alpha_m$ by the least normal subgroup containing the element
\[
    (\alpha_1)^2\cdots (\alpha_m)^2.
\]
\end{theorem}
    
\subsection{Homology of Surfaces}

\begin{definition}[First Homology Group]
If $X$ is a path-connected space, let 
\[
    H_1(X) = \pi_1(X, x_0) / [\pi_1(X, x_0), \pi_1(X, x_0)].
\]
$H_1(X)$ is the \textbf{first homology group} of $X$. (The base point is omitted from the notation because there is a unique path-induced isomorphism between
the abelianized fundamental groups based at two different points.)
\end{definition}

\begin{theorem}
Let $F$ be a group; let $N$ be a normal subgroup of $F$; let $q\colon F \rightarrow F/N$ be the projection. The projection homomorphism
\[
    p\colon F \rightarrow F/[F, F]
\]
induces an isomorphism 
\[
    \varphi\colon q(F)/[q(F), q(F)] \rightarrow p(F) / p(N).
\]
\end{theorem}

\begin{remark}
Roughly speaking, the above theorem tells us that if one divides $F$ by $N$ and then abelianizes the quotient, one obtains
the same result as if one first abelianizes $F$ and then divides by the image of $N$ in the abelianization.
\end{remark}

\begin{corollary}
Let $F$ be a free group with free generators $\alpha_1, \dots, \alpha_n$; let $N$ be the least normal subgroup of $F$ containing the element $x$ of $F$; let $G = F / N$.
Let $p\colon F \rightarrow F / [F, F]$ be projection. Then $G/[G, G]$ is isomorphic to the quotient of $F/[F, F]$, which is free abelian with basis
$p(\alpha_1), \dots, p(\alpha_n)$, by the subgroup generated by $p(x)$.
\end{corollary}

\begin{theorem}
If $X$ is the $n$-fold connected sum of tori, then $H_1(X)$ is a free abelian group of rank $2n$.
\end{theorem}

\begin{theorem}
If $X$ is the $m$-fold connected sum of projective planes, then the torsion subgroup $T(X)$ of $H_1(X)$ has order $2$, and
$H_1(X) / T(X)$ is a free abelian group of rank $m-1$.
\end{theorem}
\begin{remark}
$H_1(X)$ in the above theorem is isomorphic to the $m$-fold cartesian product $\Z \times \dots \times \Z$ by the subgroup $0 \times \dots \times 0 \times 2\Z$. 
\end{remark}

\begin{theorem}
Let $T_n$ and $P_m$ denote the $n$-fold connected sum of tori and $m$-fold connected sum of projective planes, respectively.
Then the surfaces
\[
    S^2; T_1, T_2, \dots ; P_1, P_2, \dots
\]
are topologically distinct.
\end{theorem}

\subsection{Cutting and Pasting}

\begin{theorem}[Elementary Operations on Schemes]
The following operations can be performed on a labelling scheme $w_1, \dots, w_m$ without affecting the resulting quotient space $X$:
\begin{enumerate}[i)]
    \item Cut
    \item Paste
    \item Relabel
    \item Permute
    \item Flip
    \item Cancel
\end{enumerate}
See Munkres page 460 for more detailed explanations on each.
\end{theorem}

\begin{definition}
Two labelling schemes for collections of polygonal regions are \textbf{equivalent} if one can be obtained from the other by a sequence of elementary scheme operations.
\end{definition}

\subsection{The Classification Theorem}

\begin{definition}[Proper Labelling Scheme]
A \textbf{proper labelling scheme} is one in which each label appears exactly twice in a scheme. 
\textit{Note: If one applies any elementary operation to a proper scheme, the result is another proper scheme.}
\end{definition}

\begin{definition}[Torus and Projective Type]
Let $w$ be a proper labelling scheme for a single polygonal region. We say $w$ is of \textbf{torus type} if each label in it appears once with exponent $+1$ and once with exponent $-1$.
Otherwise, $w$ is said to be of \textbf{projective type}.
\end{definition}

\begin{theorem}[Algorithm for Reducing Proper Labelling Schemes]
For proper labelling schemes, we have the following equivalences:
\begin{enumerate}[i)]
    \item $[y_0]a[y_1]a[y_2] \sim aa[y_0 y_1^{-1} y_2]$
    \item $[y_0]aa^{-1}[y_1] \sim [y_0y_1]$ if $y_0y_1$ has length at least $4$
    \item $w_0[y_1]a[y_2]b[y_3]a^{-1}[y_4]b^{-1}[y_5] \sim w_0aba^{-1}b^{-1}[y_1y_4y_3y_2y_5]$
    \item $w_0(cc)(aba^{-1}b^{-1})w_1 \sim w_0 aabbcc w_1$
\end{enumerate}
\end{theorem}

\begin{theorem}[The Classification Theorem]
Let $X$ be the quotient space obtained from a polygonal region in the plane by pasting its edges together in pairs. 
Then $X$ is homeomorphic either to $S^2$, to the $n$-fold torus $T_n$, or the $m$-fold projective plane $P_m$.
\end{theorem}

\subsection{Constructing Compact Surfaces}

\begin{definition}
Let $X$ be a compact Hausdorff space. A \textbf{curved triangle} in $X$ ia a subspace $A$ of $X$ and a homeomorphism $h \colon T \rightarrow a$, 
where $T$ is a closed triangular region in the plane. \\

If $e$ is an edge of $T$, then $h(e)$ is said to be an \textbf{edge} of $A$; if $v$ is a vertex of $T$, then $h(v)$ is said to be a \textbf{vertex} of $A$. \\

A \textbf{triangulation} of $X$ is a collection of curved triangles $A_1, \dots, A_n$ in $X$ whose union is $X$ such that for $i \neq j$, 
the intersection $A_i \cap A_j$ is an edge $e$ of both, or a vertex of both $A_i$ and $A_j$, or an edge of both. Furthermore, if $h_i \colon T_i \rightarrow A_i$ is the
homeomorphism associated with $A_i$, we require that when $A_i \cap A_j$ is an edge of both, then the map $h_j^{-1}h_i$ defines a linear homeomorphism
of the edge $h_i^{-1}(e)$ of $T_i$ with the edge $h_j^{-1}(e)$ of $T_j$. If $X$ has a triangulation, it is said to be \textbf{triangulable.}
\end{definition}

\begin{theorem}
If $X$ is a compact triangulable surface, then $X$ is homeomorphic to the quotient space obtained from a collection of disjoint triangular regions in the plane
by pasting their edges together in pairs.
\end{theorem}

\begin{theorem}
If $X$ is a compact connected triangulable surface, then $X$ is homeomorphic to a space obtained from a polygonal region in the plane by pasting the edges together in pairs.
\end{theorem}



