\section{Applications to Group Theory}

\setcounter{subsection}{82}
\subsection{Covering Spaces of a Graph}

\begin{definition}[Linear Graph]
A \textbf{linear graph} is a space $X$ that is written as the union of a collection of subspaces $A_\alpha$, each of which is an arc, such that
\begin{enumerate}[1)]
    \item The intersection $A_\alpha \cap A_\beta$ of two arcs is either empty or consists of a single point that is an endpoint of each.
    \item The topology of $X$ is coherent with the subspaces $A_\alpha$.
\end{enumerate}
The arcs $A_\alpha$ are the \textbf{edges} of $X$, and the endpoints are the \textbf{vertices} of $X$.
\end{definition}

\begin{lemma}
Every linear graph $X$ is Hausdorff and normal.
\end{lemma}

\begin{eg}
If $X$ is the wedge of the circles $S_\alpha$ with common point $p$, then $X$ can be expressed as a linear graph (each $S_\alpha$ is a graph with three edges.)
\end{eg}

\begin{definition}[Subgraph]
Let $X$ be a linear graph. Let $Y$ be a subspace of $X$ that is a union of edges of $X$. Then $Y$ is closed in $X$ and is itself a linear graph, known as a \textbf{subgraph} of $X$.
\end{definition}

\begin{theorem*}[83.4]
Let $p\colon E \rightarrow  X$ be a covering map, where $X$ is a linear graph. If $A_\alpha$ is an edge of $X$ and $B$ is a path component of $p^{-1}(A_\alpha)$, then $p$ maps $B$ homeomorphically
onto $A_\alpha$. Furthermore, the space $E$ is a linear graph, with the path components of the spaces $p^{-1}(A_\alpha)$ as its edges.
\end{theorem*}

\subsection{The Fundamental Group of a Graph}

\begin{definition}
An \textbf{oriented edge} $e$ oif a graph $X$ is an edge of $X$ together with an ordering of its vertices. \\

An \textbf{edge path} in $X$ is a sequence of oriented edges of $X$.
\end{definition}

\begin{definition}[Reduced Edge Path]
Let $e_1, \dots, e_n$ be an edge path in the linear graph $X$. It can happen that for some $i$, the oriented edges
$e_i$ and $e_{i+1}$ consist of the same edge of $X$, but with opposite orientations. If this does not occur,
then the edge path is said to be a \textbf{reduced edge path}.
\end{definition}

\begin{definition}
A subgraph $T$ of $X$ is said to be a \textbf{tree} in $X$ if $T$ is connected and contains no closed reduced edge paths.
\end{definition}

\begin{lemma*}[84.2]
If $T$ is a tree in $X$, and if $A$ is an edge of $X$ that intersects $T$ in a single vertex, then 
$T \cup A$ is a tree in $X$. Conversely, if $T$ is a finite tree in $X$ that consists of more than one edge, then there is 
a tree $T_0$ in $X$ and an edge $A$ of $X$ that intersects $T_0$ in a single vertex, so that $T = T_0 \cup A$. 
\end{lemma*}

\begin{definition}
A tree $T$ in $X$ is \textbf{maximal} if there is no tree in $X$ that properly contains $T$.
\end{definition}

\begin{theorem*}[84.7]
Let $X$ be a connected graph that is not a tree. Then the fundamental group of $X$ is a nontrivial free group.
Indeed, if $T$ is a maximal tree in $X$, then the fundamental group of $X$ has a system of free generators that is in
bijective correspondence with the collection of edges of $X$ that are not in $T$.
\end{theorem*}

\subsection{Subgroups of Free Groups}

\begin{theorem}
If $H$ is a subgroup of a free group $F$, then $H$ is free.
\end{theorem}

\begin{definition}
If $X$ is a finite linear graph, the \textbf{Euler number} of $X$, denoted $\chi(X)$ is the number of vertices of $X$ minus the number of edges.
\end{definition}

\begin{lemma}
If $X$ is a finite, connected linear graph, then the cardinality of a system of free generators 
for the fundamental group of $X$ is $1 - \chi(X)$.
\end{lemma}

\begin{theorem}
Let $F$ be a free group with $n + 1$ free generators; let $H$ be a subgroup of $F$. If $H$ has index $k$ in $F$, then $H$
has $kn+1$ free generators.
\end{theorem}