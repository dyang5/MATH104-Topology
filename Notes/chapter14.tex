\section{Applications to Group Theory}

\setcounter{subsection}{82}
\subsection{Covering Spaces of a Graph}

\begin{definition}[Linear Graph]
A \textbf{linear graph} is a space $X$ that is written as the union of a collection of subspaces $A_\alpha$, each of which is an arc, such that
\begin{enumerate}[1)]
    \item The intersection $A_\alpha \cap A_\beta$ of two arcs is either empty or consists of a single point that is an endpoint of each.
    \item The topology of $X$ is coherent with the subspaces $A_\alpha$.
\end{enumerate}
The arcs $A_\alpha$ are the \textbf{edges} of $X$, and the endpoints are the \textbf{vertices} of $X$.
\end{definition}

\begin{lemma}
Every linear graph $X$ is Hausdorff and normal.
\end{lemma}

\begin{eg}
If $X$ is the wedge of the circles $S_\alpha$ with common point $p$, then $X$ can be expressed as a linear graph (each $S_\alpha$ is a graph with three edges.)
\end{eg}

\begin{definition}[Subgraph]
Let $X$ be a linear graph. Let $Y$ be a subspace of $X$ that is a union of edges of $X$. Then $Y$ is closed in $X$ and is itself a linear graph, known as a \textbf{subgraph} of $X$.
\end{definition}

\begin{theorem*}[83.4]
Let $p\colon E \rightarrow  X$ be a covering map, where $X$ is a linear graph. If $A_\alpha$ is an edge of $X$ and $B$ is a path component of $p^{-1}(A_\alpha)$, then $p$ maps $B$ homeomorphically
onto $A_\alpha$. Furthermore, the space $E$ is a linear graph, with the path components of the spaces $p^{-1}(A_\alpha)$ as its edges.
\end{theorem*}

