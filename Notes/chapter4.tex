\newpage

\section{Countability and Separation Axioms}
\setcounter{subsection}{29}

\subsection{The Countability Axioms}

\begin{definition}[First Countability Axiom]
A space $X$ is said to have a \textbf{countable basis at $x$} if there is a countable collection $\mathscr{B}$ of neighborhoods of $x$ such
that each neighborhood of $x$ contains at least one of the elements of $\mathscr{B}$. \\

A space that has a countable basis at each of its points is said to satisfy the \textbf{first countability axiom}, or to be \textbf{first-countable}.
\end{definition}

\begin{theorem}
Let $X$ be a topological space. 
\begin{enumerate}[a)]
    \item Let $A$ be a subset of $X$. If there is a sequence of points of $A$ converging to $x$, then $x \in \overline{A}$; 
    the converse holds if $X$ is first-countable.
    \item let $f\colon X \rightarrow Y$. If $f$ is continuous, then for every convergent sequence $x_n \rightarrow x$ in $X$, the function
    $f(x_n)$ converges to $f(x)$. The converse holds if $X$ is first-countable.
\end{enumerate}
\end{theorem}

\begin{definition}[Second Countability Axiom]
If a space $X$ has a countable basis for its topology, then $X$ is said to satisfy the \textbf{second countability axiom}, or to be
\textbf{second-countable}.
\end{definition}

\begin{remark}
The second axiom implies the first; if $\mathscr{B}$ is a countable basis for $X$, then the elements of $\mathscr{B}$ containing $x$
form a countable basis at $x$.
\end{remark}

\begin{eg}
$\R$ and $\R^n$ have countable bases, constructed using the collection of (products) of intervals having rational end points. \\

Similarly, $\R^\omega$ has a countable basis consisting of all products
\[
    \prod_{n \in \Z_+} U_n
\]
where $U_n$ is an open interval with rational end points for finitely many values of $n$ and $U_n = \R$ for all other values of $n$.
\end{eg}

\begin{theorem}
A subspace of a first-countable space is first-countable, and a countable product of first-countable spaces is first-countable. \\

The same properties hold for second-countable spaces.
\end{theorem}

\begin{definition}[Dense]
A subset $A$ of a space is \textbf{dense} in $X$ if $\overline{A} = X$.
\end{definition}

\begin{theorem}
Suppose that $X$ has a countable basis. Then
\begin{enumerate}[a)]
    \item Every open covering of $X$ contains a countable subcollection covering $X$. 
    \item There exists a countable subset of $X$ that is dense in $X$.
\end{enumerate}
\end{theorem}

\begin{definition}
A space for which every open covering contains a countable subcovering is a \textbf{Lindelof space}. A space having a countable dense subset is often said to be \textbf{separable}. \\

Each of these properties is equivalent to the second countability axiom when the space is metrizable.
\end{definition}

\begin{eg}
The product of two Lindelof spaces need not be Lindelof. \\

$\R_l$ is Lindelof, whereas $\R_l^2$, the \textbf{Sorgenfrey plane}, is not.
\end{eg}

\subsection{The Separation Axioms}

As a reminder, a space is \textit{Hausdorff} if for each pair $x, y$ of distinct points of $X$, there exist disjoint 
open sets containing $x$ and $y$, respectively. 

\begin{definition}[Regular and Normal Spaces]
Suppose that one-point sets are closed in $X$. Then $X$ is \textbf{regular} if for each pair consisting of a point $x$ and a closed set $B$ disjoint from $x$,
there exist disjoint open sets containing $x$ and $B$, respectively. \\

The space $X$ is \textbf{normal} if for each pair $A, B$ of disjoint closed sets of $X$, there exist disjoint open sets containing $A$ and $B$, respectively.
\end{definition}

\begin{remark}
A regular space is Hausdorff and a normal space is regular. \\

These relations constitute the \textbf{separation axioms}.
\end{remark}

\begin{lemma}
Let $X$ be a topological space. Let one-point sets in $X$ be closed.
\begin{enumerate}[a)]
    \item $X$ is regular if and only if given a point $x$ of $X$ and a neighborhood $U$ of $x$, there is a neighborhood $V$ of $x$ such that $\overline{V} \subset U$.
    \item $X$ is normal if and only if given a closed set $A$ and an open set $U$ containing $A$, there is an open set $V$ containing $A$ such that $\overline{V} \subset U$.
\end{enumerate}
\end{lemma}

\subsection{Normal Spaces}

\begin{theorem}
Every regular space with a countable basis is normal.
\end{theorem}

\begin{theorem}
Every metrizable space is normal.
\end{theorem}

\begin{theorem}
Every compact Hausdorff space is normal.
\end{theorem}

\begin{theorem}
Every well-ordered set $X$ is normal in the order topology.
\end{theorem}

\begin{remark}
A stronger result is that \textit{every} order topology is in fact normal.
\end{remark}


\subsection{Urysohn Lemma}

\begin{theorem}[Urysohn Lemma]
Let $X$ be a normal space. let $A$ and $B$ be disjoint closed subsets of $X$. Let $[a, b]$ be a closed interval in the real line.
Then there exists a continuous map
\[
    f\colon X \rightarrow [a, b]
\]
such that $f(x) = a$ for every $x$ in $A$ and $f(x) = b$ for every $x$ in $B$.
\end{theorem}

\begin{definition}
If $A$ and $B$ are two subsets of the topological space $X$ and if there is a continuous function $f\colon X \rightarrow [0, 1]$ such that
$f(A) = \{ 0 \}$ and $f(B) = \{ 1 \}$, we say that $A$ and $B$ \textbf{can be separated by a continuous function}. 
\end{definition}

\begin{remark}
The Urysohn lemma says that if every pair of disjoint closed sets in $X$ can be separated by disjoint open sets,
then each such pair can be separated by a continuous function. The converse is trivial $(f^{-1}[0, \frac{1}{2})$ and $f^{-1}(\frac{1}{2}, 1]$ are disjoint
open sets containing $A$ and $B$, respectively.
\end{remark}

\begin{definition}
A space $X$ is \textbf{completely regular} if one-point sets are closed in $X$ and for each point $x_0$ and each closed set $A$ not containing $x_0$,
there is a continuous function $f\colon X \rightarrow [0, 1]$ such that $f(x_0) = 1$ and $f(A) = \{ 0 \}.$
\end{definition}

\begin{remark}
A normal space is completely regular by the Urysohn lemma. A completely regular space is regular.
\end{remark}

\begin{theorem}
A subspace of a completely regular space is completely regular. A product of completely regular spaces is completely regular.
\end{theorem}

\subsection{The Urysohn Metrization Theorem}

\begin{theorem}[Urysohn Metrization Theorem]
Every regular space $X$ with a countable basis is metrizable.
\end{theorem}

\begin{theorem}[Imbedding Theorem]
Let $X$ be a space in which one-point sets are closed. Suppose that $\{ f_\alpha \}_{\alpha \in J}$ is an indexed family of continuous functions
$f_\alpha \colon X \rightarrow \R$ satisfying the requirement that for each point $x_0$ of $X$ and each neighborhood $U$ of $x_0$, there is an index $\alpha$
such that $f_\alpha$ is positive at $x_0$ and vanishes outside $U$. Then the function $F\colon X \rightarrow \R^J$ defined by
\[
    F(x) = (f_\alpha(x))_{\alpha \in J}
\]
is an imbedding of $X$ in $\R^J$. If $f_\alpha$ maps $X$ into $[0, 1]$ for each $\alpha$, then $F$ imbeds $X$ in $[0, 1]^J$.
\end{theorem}

\begin{theorem}
A space $X$ is completely regular if and only if it is homeomorphic to a subspace of $[0, 1]^J$ for some $J$.
\end{theorem}

\subsection{The Tietze Extension Theorem}

\begin{theorem}[The Tietze Extension Theorem]
Let $X$ be a normal space; let $A$ be a closed subspace of $X$.
\begin{enumerate}[a)]
    \item Any continuous map of $A$ into the closed interval $[a, b]$ of $\R$ may be extended to a continuous map of all of $X$ into $[a, b]$.
    \item Any continuous map $A$ of $\R$ may be extended to a continuous map of all of $X$ into $\R$.
\end{enumerate}
\end{theorem}


